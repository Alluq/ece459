
\documentclass[letterpaper,hide notes,xcolor={table,svgnames},pdftex,10pt]{beamer}
\def\showexamples{t}


%\usepackage[svgnames]{xcolor}

%% Demo talk
%\documentclass[letterpaper,notes=show]{beamer}

\usecolortheme{crane}
\setbeamertemplate{navigation symbols}{}

\usetheme{MyPittsburgh}
%\usetheme{Frankfurt}

%\usepackage{tipa}

\usepackage{hyperref}
\usepackage{graphicx,xspace}
\usepackage[normalem]{ulem}
\usepackage{multicol}
\usepackage{amsmath,amssymb,amsthm,graphicx,xspace}
\newcommand\SF[1]{$\bigstar$\footnote{SF: #1}}

\usepackage[default]{sourcesanspro}
\usepackage[T1]{fontenc}
\usepackage[scaled]{beramono}
\usepackage{tikzpagenodes}

\newcounter{tmpnumSlide}
\newcounter{tmpnumNote}


% old question code
%\newcommand\question[1]{{$\bigstar$ \small \onlySlide{2}{#1}}}
% \newcommand\nquestion[1]{\ifdefined \presentationonly \textcircled{?} \fi \note{\par{\Large \textbf{?}} #1}}
% \newcommand\nanswer[1]{\note{\par{\Large \textbf{A}} #1}}


 \newcommand\mnote[1]{%
   \addtocounter{tmpnumSlide}{1}
   \ifdefined\showcues {~\tiny\fbox{\arabic{tmpnumSlide}}}\fi
   \note{\setlength{\parskip}{1ex}\addtocounter{tmpnumNote}{1}\textbf{\Large \arabic{tmpnumNote}:} {#1\par}}}

\newcommand\mmnote[1]{\note{\setlength{\parskip}{1ex}#1\par}}

%\newcommand\mnote[2][]{\ifdefined\handoutwithnotes {~\tiny\fbox{#1}}\fi
% \note{\setlength{\parskip}{1ex}\textbf{\Large #1:} #2\par}}

%\newcommand\mnote[2][]{{\tiny\fbox{#1}} \note{\setlength{\parskip}{1ex}\textbf{\Large #1:} #2\par}}

\newcommand\mquestion[2]{{~\color{red}\fbox{?}}\note{\setlength{\parskip}{1ex}\par{\Large \textbf{?}} #1} \note{\setlength{\parskip}{1ex}\par{\Large \textbf{A}} #2\par}\ifdefined \presentationonly \pause \fi}

\newcommand\blackboard[1]{%
\ifdefined   \showblackboard
  {#1}
  \else {\begin{center} \fbox{\colorbox{blue!30}{%
         \begin{minipage}{.95\linewidth}%
           \hspace{\stretch{1}} Some space intentionally left blank; done at the blackboard.%
         \end{minipage}}}\end{center}}%
         \fi%
}



%\newcommand\q{\tikz \node[thick,color=black,shape=circle]{?};}
%\newcommand\q{\ifdefined \presentationonly \textcircled{?} \fi}

\usepackage{listings}
\lstset{basicstyle=\footnotesize\ttfamily,
	breaklines=true,
	aboveskip=15pt,
  	belowskip=15pt,
	frame=lines,
	numbers=left, basicstyle=\scriptsize, numberstyle=\tiny, stepnumber=0, numbersep=2pt
}

\usepackage{siunitx}
\newcommand\sius[1]{\num[group-separator = {,}]{#1}\si{\micro\second}}
\newcommand\sims[1]{\num[group-separator = {,}]{#1}\si{\milli\second}}
\newcommand\sins[1]{\num[group-separator = {,}]{#1}\si{\nano\second}}
\sisetup{group-separator = {,}, group-digits = true}

%% -------------------- tikz --------------------
\usepackage{tikz}
\usetikzlibrary{positioning}
\usetikzlibrary{arrows,backgrounds,automata,decorations.shapes,decorations.pathmorphing,decorations.markings,decorations.text,decorations.pathreplacing}

\tikzstyle{place}=[circle,draw=blue!50,fill=blue!20,thick, inner sep=0pt,minimum size=6mm]
\tikzstyle{transition}=[rectangle,draw=black!50,fill=black!20,thick, inner sep=0pt,minimum size=4mm]

\tikzstyle{block}=[rectangle,draw=black, thick, inner sep=5pt]
\tikzstyle{bullet}=[circle,draw=black, fill=black, thin, inner sep=2pt]

\tikzstyle{pre}=[<-,shorten <=1pt,>=stealth',semithick]
\tikzstyle{post}=[->,shorten >=1pt,>=stealth',semithick]
\tikzstyle{bi}=[<->,shorten >=1pt,shorten <=1pt, >=stealth',semithick]

\tikzstyle{mut}=[-,>=stealth',semithick]

\tikzstyle{treereset}=[dashed,->, shorten >=1pt,>=stealth',thin]

\usepackage{ifmtarg}
\usepackage{xifthen}
\makeatletter
% new counter to now which frame it is within the sequence
\newcounter{multiframecounter}
% initialize buffer for previously used frame title
\gdef\lastframetitle{\textit{undefined}}
% new environment for a multi-frame
\newenvironment{multiframe}[1][]{%
\ifthenelse{\isempty{#1}}{%
% if no frame title was set via optional parameter,
% only increase sequence counter by 1
\addtocounter{multiframecounter}{1}%
}{%
% new frame title has been provided, thus
% reset sequence counter to 1 and buffer frame title for later use
\setcounter{multiframecounter}{1}%
\gdef\lastframetitle{#1}%
}%
% start conventional frame environment and
% automatically set frame title followed by sequence counter
\begin{frame}%
\frametitle{\lastframetitle~{\normalfont(\arabic{multiframecounter})}}%
}{%
\end{frame}%
}
\makeatother

\makeatletter
\newdimen\tu@tmpa%
\newdimen\ydiffl%
\newdimen\xdiffl%
\newcommand\ydiff[2]{%
    \coordinate (tmpnamea) at (#1);%
    \coordinate (tmpnameb) at (#2);%
    \pgfextracty{\tu@tmpa}{\pgfpointanchor{tmpnamea}{center}}%
    \pgfextracty{\ydiffl}{\pgfpointanchor{tmpnameb}{center}}%
    \advance\ydiffl by -\tu@tmpa%
}
\newcommand\xdiff[2]{%
    \coordinate (tmpnamea) at (#1);%
    \coordinate (tmpnameb) at (#2);%
    \pgfextractx{\tu@tmpa}{\pgfpointanchor{tmpnamea}{center}}%
    \pgfextractx{\xdiffl}{\pgfpointanchor{tmpnameb}{center}}%
    \advance\xdiffl by -\tu@tmpa%
}
\makeatother
\newcommand{\copyrightbox}[3][r]{%
\begin{tikzpicture}%
\node[inner sep=0pt,minimum size=2em](ciimage){#2};
\usefont{OT1}{phv}{n}{n}\fontsize{4}{4}\selectfont
\ydiff{ciimage.south}{ciimage.north}
\xdiff{ciimage.west}{ciimage.east}
\ifthenelse{\equal{#1}{r}}{%
\node[inner sep=0pt,right=1ex of ciimage.south east,anchor=north west,rotate=90]%
{\raggedleft\color{black!50}\parbox{\the\ydiffl}{\raggedright{}#3}};%
}{%
\ifthenelse{\equal{#1}{l}}{%
\node[inner sep=0pt,right=1ex of ciimage.south west,anchor=south west,rotate=90]%
{\raggedleft\color{black!50}\parbox{\the\ydiffl}{\raggedright{}#3}};%
}{%
\node[inner sep=0pt,below=1ex of ciimage.south west,anchor=north west]%
{\raggedleft\color{black!50}\parbox{\the\xdiffl}{\raggedright{}#3}};%
}
}
\end{tikzpicture}
}


%% --------------------

%\usepackage[excludeor]{everyhook}
%\PushPreHook{par}{\setbox0=\lastbox\llap{MUH}}\box0}

%\vspace*{\stretch{1}

%\setbox0=\lastbox \llap{\textbullet\enskip}\box0}

\setlength{\parskip}{\fill}

\newcommand\noskips{\setlength{\parskip}{1ex}}
\newcommand\doskips{\setlength{\parskip}{\fill}}

\newcommand\xx{\par\vspace*{\stretch{1}}\par}
\newcommand\xxs{\par\vspace*{2ex}\par}
\newcommand\tuple[1]{\langle #1 \rangle}
\newcommand\code[1]{{\sf \footnotesize #1}}
\newcommand\ex[1]{\uline{Example:} \ifdefined \presentationonly \pause \fi
  \ifdefined\showexamples#1\xspace\else{\uline{\hspace*{2cm}}}\fi}

\newcommand\ceil[1]{\lceil #1 \rceil}


\AtBeginSection[]
{
   \begin{frame}
       \frametitle{Outline}
       \tableofcontents[currentsection]
   \end{frame}
}



\pgfdeclarelayer{edgelayer}
\pgfdeclarelayer{nodelayer}
\pgfsetlayers{edgelayer,nodelayer,main}

\tikzstyle{none}=[inner sep=0pt]
\tikzstyle{rn}=[circle,fill=Red,draw=Black,line width=0.8 pt]
\tikzstyle{gn}=[circle,fill=Lime,draw=Black,line width=0.8 pt]
\tikzstyle{yn}=[circle,fill=Yellow,draw=Black,line width=0.8 pt]
\tikzstyle{empty}=[circle,fill=White,draw=Black]
\tikzstyle{bw} = [rectangle, draw, fill=blue!20, 
    text width=4em, text centered, rounded corners, minimum height=2em]
    
    \newcommand{\CcNote}[1]{% longname
	This work is licensed under the \textit{Creative Commons #1 3.0 License}.%
}
\newcommand{\CcImageBy}[1]{%
	\includegraphics[scale=#1]{creative_commons/cc_by_30.pdf}%
}
\newcommand{\CcImageSa}[1]{%
	\includegraphics[scale=#1]{creative_commons/cc_sa_30.pdf}%
}
\newcommand{\CcImageNc}[1]{%
	\includegraphics[scale=#1]{creative_commons/cc_nc_30.pdf}%
}
\newcommand{\CcGroupBySa}[2]{% zoom, gap
	\CcImageBy{#1}\hspace*{#2}\CcImageNc{#1}\hspace*{#2}\CcImageSa{#1}%
}
\newcommand{\CcLongnameByNcSa}{Attribution-NonCommercial-ShareAlike}

\newenvironment{changemargin}[1]{% 
  \begin{list}{}{% 
    \setlength{\topsep}{0pt}% 
    \setlength{\leftmargin}{#1}% 
    \setlength{\rightmargin}{1em}
    \setlength{\listparindent}{\parindent}% 
    \setlength{\itemindent}{\parindent}% 
    \setlength{\parsep}{\parskip}% 
  }% 
  \item[]}{\end{list}} 




\title{Lecture 24 --- Profiling }

\author{Patrick Lam \\ \small \texttt{patrick.lam@uwaterloo.ca}}
\institute{Department of Electrical and Computer Engineering \\
  University of Waterloo}
\date{\today}


\begin{document}

\begin{frame}
  \titlepage

 \end{frame}


\begin{frame}
\frametitle{Remember the Initial Quiz}

\Large
\begin{changemargin}{2cm}
Think back: what operations are fast and what operations are not?

Takeaway: our intuition  is often wrong. 

Not just at a macro level, \\
but at a micro level. 

You may be able to narrow down that this computation of $x$ is slow, \\
but if you examine it carefully\ldots what parts of it are slow?
\end{changemargin}

\end{frame}



\begin{frame}
\frametitle{Premature Optimization}

\vspace*{.5cm}
\Large
\begin{quote}
\textit{Programmers waste enormous amounts of time thinking about, or worrying about, the speed of noncritical parts of their programs, and these attempts at efficiency actually have a strong negative impact when debugging and maintenance are considered. We should forget about small efficiencies, say about 97\% of the time: premature optimization is the root of all evil. Yet we should not pass up our opportunities in that critical 3\%.}
\end{quote}
	\hfill -- Donald Knuth


\end{frame}



\begin{frame}
\frametitle{That Saying You Were Expecting}

\Large
\begin{changemargin}{1.5cm}
Feeling lucky? \\
Maybe you optimized a slow part. 

To make your programs or systems fast, \\
you need to find out what is currently slow and improve it (duh!). 

Up until now, it's mostly been about \\
\qquad ``let's speed this up''.\\
We haven't taken much time to decide what we should speed up.
\end{changemargin}

\end{frame}



\begin{frame}
\frametitle{Basic Overview}

\Large
\begin{changemargin}{1.5cm}
General idea:\\
\qquad collect data on what parts of the code\\
\qquad are taking up most of the time.

\begin{itemize}
\item What functions get called?
\item How long do functions take?
\item What's using memory?
\end{itemize}
\end{changemargin}

\end{frame}



\begin{frame}
\frametitle{\texttt{printf}}

\Large
\begin{changemargin}{1.5cm}
There is always the ``informal'' way:\\

\qquad ``debug'' by print statements.

On entry to \texttt{foo}, \\
log ``entering function foo'', plus timestamp.

On exit, \\
log ``exiting foo'', also with timestamp.
\end{changemargin}


\end{frame}



\begin{frame}
\frametitle{Console Profiling}

\Large
\begin{changemargin}{1cm}
Kind of works, and [JZ] has used it to figure out what blocks of a single large function are taking a long time (updating exchange rates\ldots yeah). 

But this approach is not necessarily good.  

This is ``invasive'' profiling---change the source code of the program to add instrumentation (log statements). 

Plus we have to do a lot of manual accounting.

Doesn't really scale.
\end{changemargin}

\end{frame}



\begin{frame}
\frametitle{Wizardry!}

\Large
\begin{changemargin}{2cm}
Like debugging:\\
\qquad if you get to be a wizard you can maybe do it by code inspection.

But speculative execution inside your head is harder for perf than debugging.

So: we want to use tools \\ 
and do this in a methodical way.
\end{changemargin}

\end{frame}


%%%%%%%%%%%%%%%%%%%%%%%%%%%%%%%%%%%%%%%%%%%%%%%%%%%%%%%%%%%%%%%%%%%%%%%%%%%%%%%%
\begin{frame}
  \frametitle{Introduction to Profiling}


\Large
\begin{changemargin}{1.5cm}
    So far we've been looking at small problems.\\[1em]
    Must {\bf profile} to see what's slow in a
      large program.\\[1em]
    Two main outputs:
      \begin{itemize}
        \item flat;
        \item call-graph.
      \end{itemize}



    Two main data gathering methods:
      \begin{itemize}
        \item statistical;
        \item instrumentation.
      \end{itemize}
   \end{changemargin}
\end{frame}
%%%%%%%%%%%%%%%%%%%%%%%%%%%%%%%%%%%%%%%%%%%%%%%%%%%%%%%%%%%%%%%%%%%%%%%%%%%%%%%%

%%%%%%%%%%%%%%%%%%%%%%%%%%%%%%%%%%%%%%%%%%%%%%%%%%%%%%%%%%%%%%%%%%%%%%%%%%%%%%%%
\begin{frame}
  \frametitle{Profiler Outputs}

\Large
\begin{changemargin}{.5cm}
  
  {\bf Flat Profiler:}

  \begin{itemize}
    \item Only computes average time in a particular function.
    \item Does not include other (useful) information (callees).
  \end{itemize}

~\\[1em]

  {\bf Call-graph Profiler:}

  \begin{itemize}
    \item Computes call times.
    \item Reports frequency of function calls.
    \item Gives a call graph: who called what function?
  \end{itemize}
  \end{changemargin}
\end{frame}
%%%%%%%%%%%%%%%%%%%%%%%%%%%%%%%%%%%%%%%%%%%%%%%%%%%%%%%%%%%%%%%%%%%%%%%%%%%%%%%%

%%%%%%%%%%%%%%%%%%%%%%%%%%%%%%%%%%%%%%%%%%%%%%%%%%%%%%%%%%%%%%%%%%%%%%%%%%%%%%%%
\begin{frame}
  \frametitle{Data Gathering Methods}

\large
\begin{changemargin}{1.5cm}
  {\bf Statistical:}

  Mostly, take samples of the system state, that is:
  \begin{itemize}
    \item every 100ms, check the system state.
    \item will cause some slowdown, but not much.
  \end{itemize}
~\\[1em]

  {\bf Instrumentation:}\\
  Add instructions at specified program points:
  \begin{itemize}
    \item can do this at compile time or run time (expensive);
    \item can instrument either manually or automatically;
    \item like conditional breakpoints.
  \end{itemize}
  \end{changemargin}
\end{frame}
%%%%%%%%%%%%%%%%%%%%%%%%%%%%%%%%%%%%%%%%%%%%%%%%%%%%%%%%%%%%%%%%%%%%%%%%%%%%%%%%

%%%%%%%%%%%%%%%%%%%%%%%%%%%%%%%%%%%%%%%%%%%%%%%%%%%%%%%%%%%%%%%%%%%%%%%%%%%%%%%%
\begin{frame}
  \frametitle{Guide to Profiling}

  
\large
\begin{changemargin}{1cm}
  When writing large software projects:\\[1em]

     $\bullet$ First, write clear and concise code. \\
      \qquad Don't do premature optimizations---\\
      \qquad focus on correctness.\\
     $\bullet$ Profile to get a baseline of your performance:\\[0em]
      \begin{itemize}
        \item allows you to easily track any performance changes;
        \item allows you to re-design your program before it's too late.
      \end{itemize}
  Focus your optimization efforts on the code that matters.
  \end{changemargin}
\end{frame}
%%%%%%%%%%%%%%%%%%%%%%%%%%%%%%%%%%%%%%%%%%%%%%%%%%%%%%%%%%%%%%%%%%%%%%%%%%%%%%%%

%%%%%%%%%%%%%%%%%%%%%%%%%%%%%%%%%%%%%%%%%%%%%%%%%%%%%%%%%%%%%%%%%%%%%%%%%%%%%%%%
\begin{frame}
  \frametitle{Things to Look For}

\large
\begin{changemargin}{1.5cm}
  
    Good signs:\\[0em]

    \begin{itemize}
      \item Time is spent in the right part of the system.
      \item Most time should not be spent handling errors;\\
 in non-critical code; or in exceptional cases.
      \item Time is not unnecessarily spent in OS.
    \end{itemize}
    \end{changemargin}
\end{frame}
%%%%%%%%%%%%%%%%%%%%%%%%%%%%%%%%%%%%%%%%%%%%%%%%%%%%%%%%%%%%%%%%%%%%%%%%%%%%%%%%

%%%%%%%%%%%%%%%%%%%%%%%%%%%%%%%%%%%%%%%%%%%%%%%%%%%%%%%%%%%%%%%%%%%%%%%%%%%%%%%%
\begin{frame}
  \frametitle{{\tt gprof} introduction}

  
\large
\begin{changemargin}{1.5cm}
\vspace*{-3em}
    Statistical profiler, plus some instrumentation for calls.\\[1em]
    Runs completely in user-space.\\[1em]
    Only requires a compiler.
    \end{changemargin}
\end{frame}
%%%%%%%%%%%%%%%%%%%%%%%%%%%%%%%%%%%%%%%%%%%%%%%%%%%%%%%%%%%%%%%%%%%%%%%%%%%%%%%%

%%%%%%%%%%%%%%%%%%%%%%%%%%%%%%%%%%%%%%%%%%%%%%%%%%%%%%%%%%%%%%%%%%%%%%%%%%%%%%%%
\begin{frame}
  \frametitle{{\tt gprof} usage}

  
\large
\begin{changemargin}{1.5cm}
\vspace*{-3em}
    Use the {\tt -pg} flag with {\tt clang}.\\[1em]
    Run your program as you normally would.
      \begin{itemize}
        \item Your program will now create a {\tt gmon.out} file.
      \end{itemize}

    Use gprof to interpret the results:\\
 \qquad {\tt gprof <executable>}.
    \end{changemargin}
\end{frame}
%%%%%%%%%%%%%%%%%%%%%%%%%%%%%%%%%%%%%%%%%%%%%%%%%%%%%%%%%%%%%%%%%%%%%%%%%%%%%%%%

%%%%%%%%%%%%%%%%%%%%%%%%%%%%%%%%%%%%%%%%%%%%%%%%%%%%%%%%%%%%%%%%%%%%%%%%%%%%%%%%
\begin{frame}[fragile]
  \frametitle{{\tt gprof} example}

  
\large
\begin{changemargin}{1.5cm}
    A program with 100 million calls to two math functions.

  \begin{lstlisting}
int main() {
    int i,x1=10,y1=3,r1=0;
    float x2=10,y2=3,r2=0;

    for(i=0;i<100000000;i++) {
        r1 += int_math(x1,y1);
        r2 += float_math(y2,y2);
    }
}
  \end{lstlisting}

  \begin{itemize}
    \item Looking at the code, we have no idea what takes longer.
    \item Probably would guess floating point math taking longer.
    \item (Overall, silly example.)
  \end{itemize}
  \end{changemargin}
\end{frame}
%%%%%%%%%%%%%%%%%%%%%%%%%%%%%%%%%%%%%%%%%%%%%%%%%%%%%%%%%%%%%%%%%%%%%%%%%%%%%%%%

%%%%%%%%%%%%%%%%%%%%%%%%%%%%%%%%%%%%%%%%%%%%%%%%%%%%%%%%%%%%%%%%%%%%%%%%%%%%%%%%
\begin{frame}[fragile]
  \frametitle{Example (Integer Math)}

  
\large
\begin{changemargin}{1.5cm}
  \begin{lstlisting}
int int_math(int x, int y){
    int r1;
    r1=int_power(x,y);
    r1=int_math_helper(x,y);
    return r1;
}

int int_math_helper(int x, int y){
    int r1;
    r1=x/y*int_power(y,x)/int_power(x,y);
    return r1;
}

int int_power(int x, int y){
    int i, r;
    r=x;
    for(i=1;i<y;i++){
        r=r*x;
    }
    return r;
}
  \end{lstlisting}
  \end{changemargin}
\end{frame}
%%%%%%%%%%%%%%%%%%%%%%%%%%%%%%%%%%%%%%%%%%%%%%%%%%%%%%%%%%%%%%%%%%%%%%%%%%%%%%%%


%%%%%%%%%%%%%%%%%%%%%%%%%%%%%%%%%%%%%%%%%%%%%%%%%%%%%%%%%%%%%%%%%%%%%%%%%%%%%%%%
\begin{frame}[fragile]
  \frametitle{Example (Float Math)}

\large
\begin{changemargin}{1.5cm}
  
  \begin{lstlisting}
float float_math(float x, float y) {
    float r1;
    r1=float_power(x,y);
    r1=float_math_helper(x,y);
    return r1;
}

float float_math_helper(float x, float y) {
    float r1;
    r1=x/y*float_power(y,x)/float_power(x,y);
    return r1;
}

float float_power(float x, float y){
    float i, r;
    r=x;
    for(i=1;i<y;i++) {
        r=r*x;
    }
    return r;
}
  \end{lstlisting}
  \end{changemargin}

\end{frame}
%%%%%%%%%%%%%%%%%%%%%%%%%%%%%%%%%%%%%%%%%%%%%%%%%%%%%%%%%%%%%%%%%%%%%%%%%%%%%%%%

%%%%%%%%%%%%%%%%%%%%%%%%%%%%%%%%%%%%%%%%%%%%%%%%%%%%%%%%%%%%%%%%%%%%%%%%%%%%%%%%
\begin{frame}[fragile]
  \frametitle{Flat Profile}

\large
\begin{changemargin}{1.5cm}

    When we run the program, profile says:

{
  \begin{lstlisting}[basicstyle=\tiny]
Flat profile:

Each sample counts as 0.01 seconds.
  %   cumulative   self              self     total           
 time   seconds   seconds    calls  ns/call  ns/call  name    
 32.58      4.69     4.69 300000000    15.64    15.64  int_power
 30.55      9.09     4.40 300000000    14.66    14.66  float_power
 16.95     11.53     2.44 100000000    24.41    55.68  int_math_helper
 11.43     13.18     1.65 100000000    16.46    45.78  float_math_helper
  4.05     13.76     0.58 100000000     5.84    77.16  int_math
  3.01     14.19     0.43 100000000     4.33    64.78  float_math
  2.10     14.50     0.30                             main
  \end{lstlisting}
}

  \begin{itemize}
    \item One function per line.
    \item {\bf \% time:} the percent of the total execution time in this function.
    \item {\bf self:} seconds in this function.
    \item {\bf cumulative:} sum of this function's time + any above it in table.
  \end{itemize}
  \end{changemargin}
\end{frame}
%%%%%%%%%%%%%%%%%%%%%%%%%%%%%%%%%%%%%%%%%%%%%%%%%%%%%%%%%%%%%%%%%%%%%%%%%%%%%%%%

%%%%%%%%%%%%%%%%%%%%%%%%%%%%%%%%%%%%%%%%%%%%%%%%%%%%%%%%%%%%%%%%%%%%%%%%%%%%%%%%
\begin{frame}[fragile]
  \frametitle{Flat Profile}


\large
\begin{changemargin}{1.5cm}
  \begin{lstlisting}[basicstyle=\tiny]
Flat profile:

Each sample counts as 0.01 seconds.
  %   cumulative   self              self     total           
 time   seconds   seconds    calls  ns/call  ns/call  name    
 32.58      4.69     4.69 300000000    15.64    15.64  int_power
 30.55      9.09     4.40 300000000    14.66    14.66  float_power
 16.95     11.53     2.44 100000000    24.41    55.68  int_math_helper
 11.43     13.18     1.65 100000000    16.46    45.78  float_math_helper
  4.05     13.76     0.58 100000000     5.84    77.16  int_math
  3.01     14.19     0.43 100000000     4.33    64.78  float_math
  2.10     14.50     0.30                             main
  \end{lstlisting}


  \begin{itemize}
    \item {\bf calls:} number of times this function was called
    \item {\bf self ns/call:} just self nanoseconds / calls
    \item {\bf total ns/call:} average time for function execution, including
      any other calls the function makes
  \end{itemize}
  \end{changemargin}
\end{frame}
%%%%%%%%%%%%%%%%%%%%%%%%%%%%%%%%%%%%%%%%%%%%%%%%%%%%%%%%%%%%%%%%%%%%%%%%%%%%%%%%

%%%%%%%%%%%%%%%%%%%%%%%%%%%%%%%%%%%%%%%%%%%%%%%%%%%%%%%%%%%%%%%%%%%%%%%%%%%%%%%%
\begin{frame}[fragile]
  \frametitle{Call Graph Example (1)}

\large
\begin{changemargin}{1.5cm}
  
    After the flat profile gives you a feel for which functions are costly, 
      can get a better story from the call graph.
  \vfill
  \begin{lstlisting}[basicstyle=\tiny]
index % time    self  children    called     name
                                                 <spontaneous>
[1]    100.0    0.30   14.19                 main [1]
                0.58    7.13 100000000/100000000     int_math [2]
                0.43    6.04 100000000/100000000     float_math [3]
-----------------------------------------------
                0.58    7.13 100000000/100000000     main [1]
[2]     53.2    0.58    7.13 100000000         int_math [2]
                2.44    3.13 100000000/100000000     int_math_helper [4]
                1.56    0.00 100000000/300000000     int_power [5]
-----------------------------------------------
                0.43    6.04 100000000/100000000     main [1]
[3]     44.7    0.43    6.04 100000000         float_math [3]
                1.65    2.93 100000000/100000000     float_math_helper [6]
                1.47    0.00 100000000/300000000     float_power [7]
  \end{lstlisting}
  \end{changemargin}
\end{frame}
%%%%%%%%%%%%%%%%%%%%%%%%%%%%%%%%%%%%%%%%%%%%%%%%%%%%%%%%%%%%%%%%%%%%%%%%%%%%%%%%

%%%%%%%%%%%%%%%%%%%%%%%%%%%%%%%%%%%%%%%%%%%%%%%%%%%%%%%%%%%%%%%%%%%%%%%%%%%%%%%%
\begin{frame}[fragile]
  \frametitle{Reading the Call Graph}

\large
\begin{changemargin}{1.5cm}
  
    The line with the index is the current function being looked at {\bf (primary line)}.\\
\begin{itemize}
    \item Lines above are functions which called this function.
    \item Lines below are functions which were called by this function
      (children).
  \end{itemize}
~\\
  {\bf Primary Line}

  \begin{itemize}  
    \item {\bf time:} total percentage of time spent in this function and its
      children
    \item {\bf self:} same as in flat profile
    \item {\bf children:} time spent in all calls made by the function
      \begin{itemize}
        \item should be equal to self + children of all functions below
      \end{itemize}
  \end{itemize}
  \end{changemargin}
\end{frame}
%%%%%%%%%%%%%%%%%%%%%%%%%%%%%%%%%%%%%%%%%%%%%%%%%%%%%%%%%%%%%%%%%%%%%%%%%%%%%%%%

%%%%%%%%%%%%%%%%%%%%%%%%%%%%%%%%%%%%%%%%%%%%%%%%%%%%%%%%%%%%%%%%%%%%%%%%%%%%%%%%
\begin{frame}[fragile]
  \frametitle{Reading Callers from Call Graph}

  
\large
\begin{changemargin}{1.5cm}
  {\bf Callers (functions above the primary line)}
  \begin{itemize}  
    \item {\bf self:} time spent in primary function, when called from current
      function.
    \item {\bf children:} time spent in primary function's children, when
      called from current function.
    \item {\bf called:} number of times primary function was called from current
      function / number of nonrecursive calls to primary function.
  \end{itemize}
  \end{changemargin}
\end{frame}
%%%%%%%%%%%%%%%%%%%%%%%%%%%%%%%%%%%%%%%%%%%%%%%%%%%%%%%%%%%%%%%%%%%%%%%%%%%%%%%%

%%%%%%%%%%%%%%%%%%%%%%%%%%%%%%%%%%%%%%%%%%%%%%%%%%%%%%%%%%%%%%%%%%%%%%%%%%%%%%%%
\begin{frame}[fragile]
  \frametitle{Reading Callees from Call Graph}

\large
\begin{changemargin}{1.5cm}
  
  {\bf Callees (functions below the primary line)}
  \begin{itemize}  
    \item {\bf self:} time spent in current function when called from primary.
    \item {\bf children:} time spent in current function's children calls when
      called from primary.
      \begin{itemize}
        \item self + children is an estimate of time spent in current function
          when called from primary function.
      \end{itemize}
    \item {\bf called:} number of times current function was called from primary
      function / number of nonrecursive calls to current function.
  \end{itemize}
  \end{changemargin}
\end{frame}
%%%%%%%%%%%%%%%%%%%%%%%%%%%%%%%%%%%%%%%%%%%%%%%%%%%%%%%%%%%%%%%%%%%%%%%%%%%%%%%%

%%%%%%%%%%%%%%%%%%%%%%%%%%%%%%%%%%%%%%%%%%%%%%%%%%%%%%%%%%%%%%%%%%%%%%%%%%%%%%%%
\begin{frame}[fragile]
  \frametitle{Call Graph Example (2)}

\large
\begin{changemargin}{1.5cm}
  
  \begin{lstlisting}[basicstyle=\tiny]
index % time    self  children    called     name

                2.44    3.13 100000000/100000000     int_math [2]
[4]     38.4    2.44    3.13 100000000         int_math_helper [4]
                3.13    0.00 200000000/300000000     int_power [5]
-----------------------------------------------
                1.56    0.00 100000000/300000000     int_math [2]
                3.13    0.00 200000000/300000000     int_math_helper [4]
[5]     32.4    4.69    0.00 300000000         int_power [5]
-----------------------------------------------
                1.65    2.93 100000000/100000000     float_math [3]
[6]     31.6    1.65    2.93 100000000         float_math_helper [6]
                2.93    0.00 200000000/300000000     float_power [7]
-----------------------------------------------
                1.47    0.00 100000000/300000000     float_math [3]
                2.93    0.00 200000000/300000000     float_math_helper [6]
[7]     30.3    4.40    0.00 300000000         float_power [7]
  \end{lstlisting}

  
 We can now see where most of the time comes from, and pinpoint 
      locations that make unexpected calls, etc.\\[1em]
 This example isn't too exciting; could simplify the math.
 \end{changemargin}
\end{frame}
%%%%%%%%%%%%%%%%%%%%%%%%%%%%%%%%%%%%%%%%%%%%%%%%%%%%%%%%%%%%%%%%%%%%%%%%%%%%%%%%

%%%%%%%%%%%%%%%%%%%%%%%%%%%%%%%%%%%%%%%%%%%%%%%%%%%%%%%%%%%%%%%%%%%%%%%%%%%%%%%%
\begin{frame}
  \frametitle{Summary (Profiling)}
\large
\begin{changemargin}{1.5cm}
  
  \begin{itemize}
    \item Saw how to use {\tt gprof}
    \item Profile early and often.
    \item Make sure your profiling shows what you expect. 
  \end{itemize}
  \end{changemargin}
\end{frame}
%%%%%%%%%%%%%%%%%%%%%%%%%%%%%%%%%%%%%%%%%%%%%%%%%%%%%%%%%%%%%%%%%%%%%%%%%%%%%%%%




\end{document}

