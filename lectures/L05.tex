\documentclass[letterpaper,10pt]{article}

\usepackage{enumitem}
\usepackage{titling}
\usepackage{listings,listings-rust}
\usepackage{url}
\usepackage{soul}
\usepackage{hyperref}
\usepackage{setspace}
\usepackage{subfig}
\usepackage{sectsty}
\usepackage{pdfpages}
\usepackage{colortbl}
\usepackage{multirow}
\usepackage{multicol}
\usepackage{relsize}
\usepackage{amsmath}
\usepackage{wasysym}
\usepackage{fancyvrb}
\usepackage[yyyymmdd]{datetime}
\usepackage{amsmath,amssymb,amsthm,graphicx,xspace}
\usepackage[titlenotnumbered,noend,noline]{algorithm2e}
\usepackage[compact]{titlesec}
\usepackage{XCharter}
\usepackage[T1]{fontenc}
\usepackage[scaled]{beramono}
\usepackage[normalem]{ulem}
\usepackage{booktabs}
\usepackage{tikz}
\usetikzlibrary{arrows.meta,automata,shapes,trees,matrix,chains,scopes,positioning,calc,decorations.pathreplacing}
\tikzstyle{block} = [rectangle, draw, fill=blue!20, 
    text width=2.5em, text centered, rounded corners, minimum height=2em]
\tikzstyle{bw} = [rectangle, draw, fill=blue!20, 
    text width=4em, text centered, rounded corners, minimum height=2em]

\definecolor{namerow}{cmyk}{.40,.40,.40,.40}
\definecolor{namecol}{cmyk}{.40,.40,.40,.40}
\renewcommand{\dateseparator}{-}

\let\LaTeXtitle\title
\renewcommand{\title}[1]{\LaTeXtitle{\textsf{#1}}}

\lstset{basicstyle=\footnotesize\ttfamily,breaklines=true}

\newcommand{\CPP}{C\nolinebreak\hspace{-.05em}\raisebox{.4ex}{\tiny\bf +}\nolinebreak\hspace{-.10em}\raisebox{.4ex}{\tiny\bf +}}
\def\CPP{{C\nolinebreak[4]\hspace{-.05em}\raisebox{.4ex}{\tiny\bf ++}}}

\newcommand{\handout}[5]{
  \noindent
  \begin{center}
  \framebox{
    \vbox{
      \hbox to 5.78in { {\bf ECE459: Programming for Performance } \hfill #2 }
      \vspace{4mm}
      \hbox to 5.78in { {\Large \hfill #4  \hfill} }
      \vspace{2mm}
      \hbox to 5.78in { {\em #3 \hfill \today} }
    }
  }
  \end{center}
  \vspace*{4mm}
}

\newcommand{\lecture}[3]{\handout{#1}{#2}{#3}{Lecture#1}}
\newcommand{\tuple}[1]{\ensuremath{\left\langle #1 \right\rangle}\xspace}

\addtolength{\oddsidemargin}{-1.000in}
\addtolength{\evensidemargin}{-0.500in}
\addtolength{\textwidth}{2.0in}
\addtolength{\topmargin}{-1.000in}
\addtolength{\textheight}{1.75in}
\addtolength{\parskip}{\baselineskip}
\setlength{\parindent}{0in}
\renewcommand{\baselinestretch}{1.5}
\newcommand{\term}{Winter 2020}

\singlespace


\begin{document}

\lecture{5 --- Working with Threads }{\term}{Patrick Lam and Jeff Zarnett}

\section*{Using Threads to Program for Performance}
We'll start by seeing how to use threads on ``embarrassingly parallel problems'', which have:
  \begin{itemize}
    \item mostly-independent sub-problems (little synchronization); and
    \item strong locality (little communication).
  \end{itemize}

Later, we'll see:
  \begin{itemize}
    \item which problems are amenable to parallelization (\emph{dependencies})
    \item alternative parallelization patterns\\(right now, just use one thread
          per sub-problem)
  \end{itemize}

\paragraph{About Pthreads.} Pthreads stands for POSIX threads. It's available
on most systems, including Pthreads Win32 (which I don't recommend).
Use Linux, and our provided server, for this course. C++ 11 also includes threads
in its specification.


Here's a quick {\tt pthreads} refresher~\cite{mos}:

\begin{itemize}
	\item \texttt{pthread\_create} -- Create a new thread. 
	\item \texttt{pthread\_exit} -- Terminate the calling thread. It ends execution and returns a value.
	\item \texttt{pthread\_join} -- Wait for a specific thread to exit. The caller cannot proceed until the thread it is waiting for calls \texttt{pthread\_exit}. Note that it is an error to join a thread that has already been joined.
	\item \texttt{pthread\_yield} -- Release the CPU and let another thread run. We expect that threads want to co-operate rather then compete for CPU time and threads can make decisions about when it would be ideal to let some other thread run instead.
	\item \texttt{pthread\_attr\_init} -- Create and initialize a thread's attributes. The attributes contain things like the priority of the thread.
	\item \texttt{pthread\_attr\_destroy} -- Clean up a thread's attributes. Free up the memory holding the thread's attributes. This does not terminate the threads.
	\item \texttt{pthread\_cancel} -- Signal cancellation to a thread; this can be asynchronous or deferred, depending on the thread's attributes.
\end{itemize}


To compile a C or C++ program
with pthreads, add the {\tt -pthread} parameter to the compiler
commandline. If you want C99 for C, add \texttt{-std=c99}. To ensure C++11 support in GCC, use \verb!-std=c++11!.

\paragraph{Starting a new thread.} You can start a thread with
\verb+pthread_create()+ or by creating a \verb+std::thread+:

{\small
  \begin{minipage}{.55\textwidth}
\begin{lstlisting}[language=C]
#include <pthread.h>
#include <stdio.h>

void* run(void*) {
  printf("In run\n");
}

int main() {
  pthread_t thread;
  pthread_create(&thread, NULL, run, NULL);
  printf("In main\n");
}
\end{lstlisting}
  \end{minipage} 
  \begin{minipage}{.4\textwidth}
\begin{lstlisting}[language=C]
#include <thread>
#include <iostream>

void run() {
  std::cout << "In run\n";
}

int main() {
  std::thread t1(run);
  std::cout << "In main\n";
  t1.join(); // see below
}
\end{lstlisting}
  \end{minipage}
}

From the man page, here's how you use \verb+pthread_create+:
\begin{lstlisting}[language=C]
int pthread_create(pthread_t* thread, 
                   const pthread_attr_t* attr,
                   void* (*start_routine)(void*),
                   void* arg);
\end{lstlisting}

\begin{itemize}
\item  {\bf thread}: creates a handle to a thread at pointer location

\item  {\bf attr}: thread attributes (NULL for defaults, more details later)

\item  {\bf start\_routine}: function to start execution

\item   {\bf arg}: value to pass to start\_routine
\end{itemize}

This function returns 0 on success and an error number otherwise (in
which case the contents of *thread are undefined).

\paragraph{Waiting for Threads to Finish.} If you want to join the threads
of execution, use the {\tt pthread\_join} call. Let's improve our example.

\begin{lstlisting}[language=C]
#include <pthread.h>
#include <stdio.h>

void* run(void*) {
  printf("In run\n");
}

int main() {
  pthread_t thread;
  pthread_create(&thread, NULL, &run, NULL);
  printf("In main\n");
  pthread_join(thread, NULL);
}
\end{lstlisting}

The main thread now waits for the newly created thread to terminate
before it terminates. (C++11 requires threads to be either joined or
detached when they go out of scope; we'll see the meaning of detach below.)

Here's the syntax for {\tt pthread\_join}:

\begin{lstlisting}[language=C]
int pthread_join(pthread_t thread, void** retval)
\end{lstlisting}
~\vspace*{-3em}
\begin{itemize}
\item  {\bf thread}: wait for this thread to terminate (thread must be~joinable).

\item  {\bf retval}: stores exit status of thread (set by {\tt pthread\_exit}) to
                 the location pointed by *retval. If cancelled, returns
                 {\tt PTHREAD\_CANCELED}. {\tt NULL} is ignored.
\end{itemize}

This function returns 0 on success, error number otherwise.

 {\bf Caveat: Only call this one time per thread!} Multiple calls to join on the same thread
  lead to undefined behaviour.

\paragraph{Inter-thread communication.} Recall that the {\tt pthread\_create} 
call allows you to pass data to the new thread. Let's see how we might do that\ldots

\begin{lstlisting}[language=C]
int i;
for (i = 0; i < 10; ++i) {
  pthread_create(&thread[i], NULL, run, &i);
}
\end{lstlisting}

{\bf Wrong!} This is a \emph{terrible} idea. Why?
\begin{enumerate}
    \item The value of {\tt i} will probably change before the thread executes.
    \item The memory for {\tt i} may be out of scope, and therefore invalid by
          the time the thread executes.
\end{enumerate}
The worst part is that in a simple program where you just write most of the pthread launch logic in main, then \texttt{i} will not go out of scope and it will seem like everything is fine... Instead we need to do this:

\begin{lstlisting}[language=C]
int i;
int* arg;
for (i = 0; i < 10; ++i) {
  arg = malloc( sizeof( int ) );
  *arg = i;
  pthread_create(&thread[i], NULL, run, arg);
}
\end{lstlisting}

And then \texttt{arg} needs to be deallocated at some point by the thread.

On the other hand, you can pull off something similar with C++11 threads:
\begin{lstlisting}[language=C]
int i;
for (i = 0; i < 10; ++i) {
  std::thread t(run, i);
  t.detach();
}
\end{lstlisting}
This is OK because we pass {\tt i} by value, which doesn't work for Pthreads.

In Pthreads-land, this is marginally acceptable:
\begin{lstlisting}[language=C]
int i;
for (i = 0; i < 10; ++i)
  pthread_create(&thread[i], NULL, &run, (void*)i);

...

void* run(void* arg) {
  int id = (int)arg;
\end{lstlisting}
It's not ideal, though.
  \begin{itemize}
    \item Beware size mismatches between arguments: you have
      no guarantee that a pointer is the same size as an int, so your data
      may overflow. (C only guarantees that the difference between two pointers is an int.)
    \item Sizes of data types change between systems. For maximum
      portability, just use pointers you got from {\tt malloc}.
  \end{itemize}

  The idiomatic way of returning data from threads in C++11 appears to be using
  futures. {\tt std::async} provides support for this:
\begin{lstlisting}[language=C]
#include <thread>
#include <iostream>
#include <future>

int run() {
  return 42;
}

int main() {
  std::future<int> t1_retval = std::async(std::launch::async, run);
  std::cout << t1_retval.get();
}
\end{lstlisting}
This launches your thread for you. The {\tt get()} call waits until the answer
is ready and returns it to you.

\paragraph{More on inter-thread synchronization.} There was a comment on {\tt pthread\_join}
only working if the target thread was joinable. Joinable threads
(which is the default on Linux) wait for someone to call {\tt pthread\_join}
before they release their resources (e.g. thread stacks). On the other
hand, you can also create \emph{detached} threads, which release
resources when they terminate, without being joined. We've seen C++11 detached
threads above.

\begin{lstlisting}[language=C]
int pthread_detach(pthread_t thread);
\end{lstlisting}
~\vspace*{-3em}
\begin{itemize}
\item  {\bf thread}: marks the thread as detached
\end{itemize}

This call returns 0 on success, error number otherwise.

Calling {\tt pthread\_detach} on an already detached thread results in undefined
behaviour.

\paragraph{Finishing a thread.} A thread finishes when its {\tt start\_routine}
returns. But it's also possible to explicitly end a thread from within:

\begin{lstlisting}[language=C]
void pthread_exit(void *retval);
\end{lstlisting}
~\vspace*{-3em}
\begin{itemize}
\item  {\bf retval}: return value passed to function which called {\tt pthread\_join}
\end{itemize}

Alternately, returning from the thread's {\tt start\_routine} is equivalent
to calling {\tt pthread\_exit}, and {\tt start\_routine}'s return value
is passed back to the {\tt pthread\_join} caller. There is no C++11 equivalent.

\paragraph{Attributes.} Beyond being detached/joinable, threads have additional
attributes. (Note, also, that even though being joinable rather than
detached is the default on Linux, it's not necessarily the default everywhere).
Here's a list.
  \begin{itemize}
    \item Detached or joinable state
    \item Scheduling inheritance
    \item Scheduling policy
    \item Scheduling parameters
    \item Scheduling contention scope
    \item Stack size
    \item Stack address
    \item Stack guard (overflow) size
  \end{itemize}

Basically, you create and destroy attributes objects with {\tt
  pthread\_attr\_init} and {\tt pthread\_attr\_destroy}
respectively. You can pass attributes objects to {\tt
  pthread\_create}. For instance,

  \begin{lstlisting}[language=C]
size_t stacksize;
pthread_attr_t attributes;
pthread_attr_init(&attributes);
pthread_attr_getstacksize(&attributes, &stacksize);
printf("Stack size = %i\n", stacksize);
pthread_attr_destroy(&attributes);
  \end{lstlisting}

Running this on a laptop produces:

  \begin{lstlisting}[language=C]
jon@riker examples master % ./stack_size 
Stack size = 8388608
  \end{lstlisting}

Once you have a thread attribute object, you can set the thread state to joinable:
  \begin{lstlisting}[language=C]
pthread_attr_setdetachstate(&attributes, PTHREAD_CREATE_JOINABLE);
  \end{lstlisting}

\paragraph{Warning about detached threads.} Consider the following code.

\begin{lstlisting}[language=C]
#include <pthread.h>
#include <stdio.h>

void* run(void*) {
  printf("In run\n");
}

int main() {
  pthread_t thread;
  pthread_create(&thread, NULL, &run, NULL);
  pthread_detach(thread);
  printf("In main\n");
}
\end{lstlisting}

  When I run it, it just prints ``In main''. Why?

\paragraph{Solution.} Use {\tt pthread\_exit} to quit if you have any detached threads.
  \begin{lstlisting}[language=C]
#include <pthread.h>
#include <stdio.h>

void* run(void*) {
  printf("In run\n");
}

int main() {
  pthread_t thread;
  pthread_create(&thread, NULL, &run, NULL);
  pthread_detach(thread);
  printf("In main\n");
  pthread_exit(NULL); // This waits for all detached
                      // threads to terminate
}
  \end{lstlisting}
(There is no C++11 equivalent.)

%% \paragraph{Three useful Pthread calls.} You may find these helpful.

%% \begin{lstlisting}[language=C]
%% pthread_t pthread_self(void);

%% int pthread_equal(pthread_t t1, pthread_t t2);

%% int pthread_once(pthread_once_t* once_control,
%%                  void (*init_routine)(void));
%% pthread_once_t once_control = PTHREAD_ONCE_INIT;
%% \end{lstlisting}

%% \begin{itemize}
%% \item {\tt pthread\_self} returns the handle of the currently running thread.
%% \item {\tt pthread\_equal} compares 2 threads for equality.
%% \item {\tt pthread\_once} runs a block of code just once, even across threads, based
%% on a (presumably-global) control variable. This is useful for initialization code.
%% \end{itemize}

\paragraph{Threading Challenges.}
  \begin{itemize}
    \item Be aware of scheduling (you can also set affinity with pthreads on
      Linux).

    \item Make sure the libraries you use are {\bf thread-safe}:
      \begin{itemize}
        \item Means that the library protects its shared data (we'll see how, below).
      \end{itemize}

    \item glibc reentrant functions are also safe: a program can have
      more than one thread calling these functions concurrently. For
      example, use {\tt rand\_r}, not {\tt
        rand}.
  \end{itemize}



\bibliographystyle{alphaurl}
\bibliography{459}


\end{document}
