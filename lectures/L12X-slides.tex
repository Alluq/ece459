% 60 minutes
\documentclass[aspectratio=43]{beamer}

% Text packages to stop warnings
\usepackage{lmodern}
\usepackage{textcomp}
\usepackage{listings}
\usepackage{alltt}
\usepackage{tikz}
\usetikzlibrary{arrows,decorations.pathreplacing,positioning}

% Themes
\usetheme{Boadilla}
\setbeamertemplate{footline}[page number]{}
\setbeamertemplate{navigation symbols}{}

% Suppress the navigation bar
\beamertemplatenavigationsymbolsempty

\lstset{basicstyle=\scriptsize, frame=single}

\newenvironment{changemargin}[1]{% 
  \begin{list}{}{% 
    \setlength{\topsep}{0pt}% 
    \setlength{\leftmargin}{#1}% 
    \setlength{\rightmargin}{1em}
    \setlength{\listparindent}{\parindent}% 
    \setlength{\itemindent}{\parindent}% 
    \setlength{\parsep}{\parskip}% 
  }% 
  \item[]}{\end{list}} 

\title{Lecture 10X---A Principled View of OpenMP (extra)}
\subtitle{ECE 459: Programming for Performance}
\date{February 11, 2013}

\begin{document}

%%%%%%%%%%%%%%%%%%%%%%%%%%%%%%%%%%%%%%%%%%%%%%%%%%%%%%%%%%%%%%%%%%%%%%%%%%%%%%%%
\begin{frame}[plain]
  \titlepage
\end{frame}
%%%%%%%%%%%%%%%%%%%%%%%%%%%%%%%%%%%%%%%%%%%%%%%%%%%%%%%%%%%%%%%%%%%%%%%%%%%%%%%%

%%%%%%%%%%%%%%%%%%%%%%%%%%%%%%%%%%%%%%%%%%%%%%%%%%%%%%%%%%%%%%%%%%%%%%%%%%%%%%%%
\begin{frame}
  \frametitle{What is OpenMP?}

  \begin{changemargin}{1.5cm}
    A portable, easy to use parallel programming API.\\[1em]
    Combines:
      \begin{itemize}
        \item Compiler directives;
        \item Runtime library routines; and
        \item Environment variables.
      \end{itemize}
      ~\\

    Compiling with OpenMP also defines {\tt \_OPENMP} for {\tt ifdef}s.\\[1em]

  {\bf Documentation:}

  \url{http://www.openmp.org/mp-documents/OpenMP3.1.pdf}

  \end{changemargin}

\end{frame}
%%%%%%%%%%%%%%%%%%%%%%%%%%%%%%%%%%%%%%%%%%%%%%%%%%%%%%%%%%%%%%%%%%%%%%%%%%%%%%%%

%%%%%%%%%%%%%%%%%%%%%%%%%%%%%%%%%%%%%%%%%%%%%%%%%%%%%%%%%%%%%%%%%%%%%%%%%%%%%%%%
\begin{frame}
  \frametitle{Directive Format}

  \begin{changemargin}{2.5cm}

  \begin{center}
    {\tt \#pragma omp }{\it directive-name [clause [[,] clause]*]}
  \end{center}

    ~\\[1em]
    There are {\bf 16} directives.\\[1em]
    Either a single statement or a compound statement {\bf \{ \}} goes
      after the directive.\\[1em]

    Most clauses have a {\bf list} as an argument.
      \begin{itemize}
        \item A {\bf list} is a comma-separated list of {\bf list items}.\\
            A {\bf list item} is simply a variable name (for C/C++)
      \end{itemize}
  \end{changemargin}
\end{frame}
%%%%%%%%%%%%%%%%%%%%%%%%%%%%%%%%%%%%%%%%%%%%%%%%%%%%%%%%%%%%%%%%%%%%%%%%%%%%%%%%

\part{Data Terminology}
\frame{\partpage}
%%%%%%%%%%%%%%%%%%%%%%%%%%%%%%%%%%%%%%%%%%%%%%%%%%%%%%%%%%%%%%%%%%%%%%%%%%%%%%%%
\begin{frame}
  \frametitle{Three Keywords for Variable Scope and Storage}

  \begin{changemargin}{2.5cm}
\Large
      \begin{itemize}
        \item private;
        \item shared; and
        \item threadprivate.
      \end{itemize}
    ~\\[1em]
  \end{changemargin}
\end{frame}
%%%%%%%%%%%%%%%%%%%%%%%%%%%%%%%%%%%%%%%%%%%%%%%%%%%%%%%%%%%%%%%%%%%%%%%%%%%%%%%%

%%%%%%%%%%%%%%%%%%%%%%%%%%%%%%%%%%%%%%%%%%%%%%%%%%%%%%%%%%%%%%%%%%%%%%%%%%%%%%%%
\begin{frame}[fragile]
  \frametitle{Private Variables}

  \begin{changemargin}{2cm}
    Declared with {\tt private} \structure{clause} in OpenMP.\\[1em]
    Creates new storage (does not copy values) for the variable.\\[1em]
    Scope extends from the start of the region to the end. \\Destroyed afterwards.\\[2em]

  Pthread pseudocode for private variables:

  \begin{lstlisting}[language=C]
void* run(void* arg) {
    int x;
    // use x
}
  \end{lstlisting}
  \end{changemargin}

\end{frame}
%%%%%%%%%%%%%%%%%%%%%%%%%%%%%%%%%%%%%%%%%%%%%%%%%%%%%%%%%%%%%%%%%%%%%%%%%%%%%%%%

%%%%%%%%%%%%%%%%%%%%%%%%%%%%%%%%%%%%%%%%%%%%%%%%%%%%%%%%%%%%%%%%%%%%%%%%%%%%%%%%
\begin{frame}[fragile]
  \frametitle{Shared Variables}

  \begin{changemargin}{2cm}
    Declared with {\tt shared} \structure{clause} in OpenMP.\\[1em]
    All threads have access to the same block of data.\\[2em]

    Pthread pseudocode:
  \begin{lstlisting}[language=C]
int x;

void* run(void* arg) {
    // use x
}
  \end{lstlisting}
  \end{changemargin}

\end{frame}
%%%%%%%%%%%%%%%%%%%%%%%%%%%%%%%%%%%%%%%%%%%%%%%%%%%%%%%%%%%%%%%%%%%%%%%%%%%%%%%%

%%%%%%%%%%%%%%%%%%%%%%%%%%%%%%%%%%%%%%%%%%%%%%%%%%%%%%%%%%%%%%%%%%%%%%%%%%%%%%%%
\begin{frame}[fragile]
  \frametitle{Thread-Private Variables}
  \begin{changemargin}{2.5cm}
    Declared with {\tt threadprivate} \structure{directive} in OpenMP.\\[1em]
    Each thread makes a copy of the variable.\\[1em]
    Variable accessible to the thread in any parallel region.\\[2em]

OpenMP code:
  \begin{lstlisting}[language=C]
int x;
#pragma omp threadprivate(x)
  \end{lstlisting}

  maps to this Pthread pseudocode:
  \begin{lstlisting}[language=C]
int x;
int x[NUM_THREADS];

void* run(void* arg) {
  // use x[pthread_self()]
}
  \end{lstlisting}
  \end{changemargin}

\end{frame}
%%%%%%%%%%%%%%%%%%%%%%%%%%%%%%%%%%%%%%%%%%%%%%%%%%%%%%%%%%%%%%%%%%%%%%%%%%%%%%%%

%%%%%%%%%%%%%%%%%%%%%%%%%%%%%%%%%%%%%%%%%%%%%%%%%%%%%%%%%%%%%%%%%%%%%%%%%%%%%%%%
\begin{frame}
  \frametitle{Contents of Clauses}

  \begin{changemargin}{2.5cm}
    A variable may not appear in {\bf more than one clause} on the same
      directive.\\[1em]
    There's an exception for {\tt firstprivate} and {\tt lastprivate}, which we'll
      see later.\\[1em]
    By default, variables declared in regions are \structure{private} and
      outside are \structure{shared} (exception: anything with dynamic
      storage is shared).
  \end{changemargin}

\end{frame}
%%%%%%%%%%%%%%%%%%%%%%%%%%%%%%%%%%%%%%%%%%%%%%%%%%%%%%%%%%%%%%%%%%%%%%%%%%%%%%%%

\part{Directives}
\frame{\partpage}
%%%%%%%%%%%%%%%%%%%%%%%%%%%%%%%%%%%%%%%%%%%%%%%%%%%%%%%%%%%%%%%%%%%%%%%%%%%%%%%%
\begin{frame}
  \frametitle{Parallel}

  \begin{changemargin}{2.5cm}
  \begin{center}
    {\tt \#pragma omp }{\bf parallel} {\it [clause [[,] clause]*]}
  \end{center}~\\
    This is the most basic directive in OpenMP.\\[1em]
    Forms a team of threads and starts parallel execution.\\[1em]
    The thread that enters the region becomes the {\bf master} (thread 0).\\[2em]
  Allowed Clauses: {\bf if, num\_threads, default, private, firstprivate,
    shared, copyin, reduction}.
  \end{changemargin}
\end{frame}
%%%%%%%%%%%%%%%%%%%%%%%%%%%%%%%%%%%%%%%%%%%%%%%%%%%%%%%%%%%%%%%%%%%%%%%%%%%%%%%%

%%%%%%%%%%%%%%%%%%%%%%%%%%%%%%%%%%%%%%%%%%%%%%%%%%%%%%%%%%%%%%%%%%%%%%%%%%%%%%%%
\begin{frame}
  \frametitle{Visual Explanation of Parallel}

  \tikzstyle{brace} = [decorate,decoration={brace},thick]
  \tikzstyle{transition} = [rectangle, draw=blue!50, fill=blue!20, thick,
                            text width=6.75cm, align=center]
  \tikzstyle{thread} = [->, >=stealth', very thick]
  \tikzstyle{master} = [->, >=stealth', very thick, red]

  \begin{center}
  \begin{tikzpicture}[decoration={brace}]

    \node[transition] (begin) {{\tt \#pragma omp parallel}\\begin block};
    \node[transition] (end) [below=of begin] {end block};

    \draw[master]   (   0,   1.55) to (   0,   0.55);

    \draw[master]   (-3.5, -0.525) to (-3.5, -1.525);
    \draw[thread]   (-2.5, -0.525) to (-2.5, -1.525);
    \draw[thread]   (-1.5, -0.525) to (-1.5, -1.525);
    \draw[thread]   (-0.5, -0.525) to (-0.5, -1.525);
    \draw[thread]   ( 0.5, -0.525) to ( 0.5, -1.525);
    \draw[thread]   ( 1.5, -0.525) to ( 1.5, -1.525);
    \draw[thread]   ( 2.5, -0.525) to ( 2.5, -1.525);
    \draw[thread]   ( 3.5, -0.525) to ( 3.5, -1.525);

    \draw[master]   (   0,  -2.08) to (   0,  -3.08);

    \draw[decorate, thick, red] (-4,  -3.08) -- (-4,  -2.08)
      node[red, anchor=east, midway, outer sep=0.1cm] {Master Thread};
    \draw[decorate, thick]      (-4, -1.525) -- (-4, -0.525)
      node[anchor=east, midway, outer sep=0.1cm] {Thread Team};
    \draw[decorate, thick, red] (-4,   0.55) -- (-4,   1.55)
      node[red, anchor=east, midway, outer sep=0.1cm] {Master Thread};
  \end{tikzpicture}
  \end{center}

  \begin{changemargin}{.5cm}
  \begin{itemize}
    \item By default, the number of threads used is set globally automatically
      or manually.
    \item After the parallel block, the thread team sleeps until it's needed.
  \end{itemize}
  \end{changemargin}

\end{frame}
%%%%%%%%%%%%%%%%%%%%%%%%%%%%%%%%%%%%%%%%%%%%%%%%%%%%%%%%%%%%%%%%%%%%%%%%%%%%%%%%

%%%%%%%%%%%%%%%%%%%%%%%%%%%%%%%%%%%%%%%%%%%%%%%%%%%%%%%%%%%%%%%%%%%%%%%%%%%%%%%%
\begin{frame}[fragile]
  \frametitle{Parallel Example}

  \begin{changemargin}{2.5cm}

  \begin{lstlisting}[langauge=C]
#pragma omp parallel
{
    printf("Hello!");
}
  \end{lstlisting}
  \vfill
  If the number of threads is 4, this produces:
  \vfill
  \begin{lstlisting}[langauge=C]
Hello!
Hello!
Hello!
Hello!
  \end{lstlisting}
  \end{changemargin}


\end{frame}
%%%%%%%%%%%%%%%%%%%%%%%%%%%%%%%%%%%%%%%%%%%%%%%%%%%%%%%%%%%%%%%%%%%%%%%%%%%%%%%%

%%%%%%%%%%%%%%%%%%%%%%%%%%%%%%%%%%%%%%%%%%%%%%%%%%%%%%%%%%%%%%%%%%%%%%%%%%%%%%%%
\begin{frame}[fragile]
  \frametitle{{\tt if} and {\tt num\_threads} Clauses}

  \begin{changemargin}{2.5cm}
  \begin{center}
    {\bf if}({\it primitive-expression})    
  \end{center}
  \begin{itemize}
    \item If primitive-expression {\tt false}, then only one thread will execute.
  \end{itemize}~\\

  If the parallel section is going to run multiple threads (e.g. {\bf if} expression
  is true), we can specify how many:
  \begin{center}
    {\bf num\_threads}({\it integer-expression})    
  \end{center}
  \begin{itemize}
    \item Spawns at most {\bf num\_threads}, depending on the number of
      threads available.
    \item Can only guarantee the number of threads requested if
      {\bf dynamic adjustment} for number of threads is off and 
      enough threads aren't busy.
  \end{itemize}
  \end{changemargin}

\end{frame}
%%%%%%%%%%%%%%%%%%%%%%%%%%%%%%%%%%%%%%%%%%%%%%%%%%%%%%%%%%%%%%%%%%%%%%%%%%%%%%%%

%%%%%%%%%%%%%%%%%%%%%%%%%%%%%%%%%%%%%%%%%%%%%%%%%%%%%%%%%%%%%%%%%%%%%%%%%%%%%%%%
\begin{frame}
  \frametitle{{\tt reduction} Clause}

  \begin{changemargin}{2.5cm}
  \begin{center}
    {\bf reduction}({\it operator:list})
  \end{center}
  ~\\

  {\bf Operators (Initial Value)}
  \begin{center}
    \begin{tabular}{l r | l r | l r | l r | l r}
      + & (0) & -  &  (0) &    $\mid$ & (0) & \&\& & (1) & max & MAX\\
      * & (1) & \& & (\~{}0) & \^{} & (0) &   $\mid\mid$ & (0) & min & MIN\\ 
    \end{tabular}
  \end{center}
  ~\\[1em]
  
  \begin{itemize}
    \item Each thread gets a \structure{private} copy of the variable.
    \item The variable is initialized by OpenMP (so you don't need to do anything else).
    \item At the end of the region, OpenMP updates your result using the operator.
  \end{itemize}
  \end{changemargin}

\end{frame}
%%%%%%%%%%%%%%%%%%%%%%%%%%%%%%%%%%%%%%%%%%%%%%%%%%%%%%%%%%%%%%%%%%%%%%%%%%%%%%%%

%%%%%%%%%%%%%%%%%%%%%%%%%%%%%%%%%%%%%%%%%%%%%%%%%%%%%%%%%%%%%%%%%%%%%%%%%%%%%%%%
\begin{frame}[fragile]
  \frametitle{{\tt reduction} Clause Pthreads Pseudocode}

  \begin{changemargin}{2cm}
  \begin{lstlisting}
void* run(void* arg) {
    variable = initial value;
    // code inside block---modifies variable
    return variable;
}

// ... later in master thread (sequentially):
variable = initial value
for t in threads {
    thread_variable
    pthread_join(t, &thread_variable);
    variable = variable (operator) thread_variable;
}
  \end{lstlisting}
  \end{changemargin}

\end{frame}
%%%%%%%%%%%%%%%%%%%%%%%%%%%%%%%%%%%%%%%%%%%%%%%%%%%%%%%%%%%%%%%%%%%%%%%%%%%%%%%%

%%%%%%%%%%%%%%%%%%%%%%%%%%%%%%%%%%%%%%%%%%%%%%%%%%%%%%%%%%%%%%%%%%%%%%%%%%%%%%%%
\begin{frame}[fragile]
  \frametitle{(For) Loop Clause}

  \begin{changemargin}{2.5cm}
  \begin{center}
    {\tt \#pragma omp }{\bf for} {\it [clause [[,] clause]*]}
  \end{center}

    Iterations of the loop will be distributed among the
      current team of threads.\\[1em]
    Only supports simple ``for'' loops with invariant bounds (bounds do
      not change during the loop).\\[1em]
    Loop variable is implicitly \structure{private}; OpenMP sets the
      correct values.\\[2em]

  Allowed Clauses: {\bf private, firstprivate, lastprivate, reduction, schedule,
    collapse, ordered, nowait}.
  \end{changemargin}

\end{frame}
%%%%%%%%%%%%%%%%%%%%%%%%%%%%%%%%%%%%%%%%%%%%%%%%%%%%%%%%%%%%%%%%%%%%%%%%%%%%%%%%

%%%%%%%%%%%%%%%%%%%%%%%%%%%%%%%%%%%%%%%%%%%%%%%%%%%%%%%%%%%%%%%%%%%%%%%%%%%%%%%%
\begin{frame}
  \frametitle{{\tt schedule} Clause}

  \begin{changemargin}{1.5cm}
  \begin{center}
    {\bf schedule}({\it kind[, chunk\_size]})
  \end{center}

    The {\bf chunk\_size} is the number of iterations a single thread
      should handle at a time.\\[1em]
    {\bf kind} is one of:
      \begin{itemize}
        \item {\bf static}
        \item {\bf dynamic}
        \item {\bf guided}
        \item {\bf auto}
        \item {\bf runtime}
      \end{itemize}~\\
    {\bf auto} is obvious (OpenMP decides what's best for you).\\
    {\bf runtime} is also obvious; we'll see how to adjust this later.\\
  \end{changemargin}

\end{frame}
%%%%%%%%%%%%%%%%%%%%%%%%%%%%%%%%%%%%%%%%%%%%%%%%%%%%%%%%%%%%%%%%%%%%%%%%%%%%%%%%

%%%%%%%%%%%%%%%%%%%%%%%%%%%%%%%%%%%%%%%%%%%%%%%%%%%%%%%%%%%%%%%%%%%%%%%%%%%%%%%%
\begin{frame}
  \frametitle{{\tt schedule} Clause {\tt kind}s}

  \begin{changemargin}{2.5cm}
    {\bf static}
    \begin{itemize}
      \item Divides the number of iterations into chunks and assigns each thread
        a chunk in round-robin fashion (before the loop executes).
    \end{itemize}

    {\bf dynamic}
    \begin{itemize}
      \item Divides the number of iterations into chunks and assigns each
        available thread a chunk, until there are no chunks left.
    \end{itemize}

    {\bf guided}
    \begin{itemize}
      \item Same as dynamic, except {\bf chunk\_size} represents the minimum
        size.
      \item Starts by dividing the loop into large chunks, and decreases the
        chunk size as fewer iterations remain.
    \end{itemize}
  \end{changemargin}

\end{frame}
%%%%%%%%%%%%%%%%%%%%%%%%%%%%%%%%%%%%%%%%%%%%%%%%%%%%%%%%%%%%%%%%%%%%%%%%%%%%%%%%

%%%%%%%%%%%%%%%%%%%%%%%%%%%%%%%%%%%%%%%%%%%%%%%%%%%%%%%%%%%%%%%%%%%%%%%%%%%%%%%%
\begin{frame}
  \frametitle{{\tt collapse} and {\tt ordered} Clauses}

  \begin{changemargin}{1cm}
  \begin{center}
    {\bf collapse}({\it n})
  \end{center}

    This collapses {\it n} levels of loops.\\[1em]
    $n$ should be at least 2, otherwise nothing happens.\\[1em]
    Collapsed loop variables are also made \structure{private}.\\[2em]

  \begin{center}
    {\bf ordered}
  \end{center}
  
    Enables the use of {\tt ordered} directives inside loop.
  \end{changemargin}

\end{frame}
%%%%%%%%%%%%%%%%%%%%%%%%%%%%%%%%%%%%%%%%%%%%%%%%%%%%%%%%%%%%%%%%%%%%%%%%%%%%%%%%

%%%%%%%%%%%%%%%%%%%%%%%%%%%%%%%%%%%%%%%%%%%%%%%%%%%%%%%%%%%%%%%%%%%%%%%%%%%%%%%%
\begin{frame}
  \frametitle{Ordered}

  \begin{changemargin}{2.5cm}
  \begin{center}
    {\tt \#pragma omp }{\bf ordered}
  \end{center}

    Containing loop must have an {\bf ordered} clause.\\[1em]
    OpenMP will ensure that the ordered directives are executed the same
      way the sequential loop would\\~~(one at a time).\\[1em]
    Each iteration of the loop may execute {\bf at most one} ordered
      directive.
  \end{changemargin}

\end{frame}
%%%%%%%%%%%%%%%%%%%%%%%%%%%%%%%%%%%%%%%%%%%%%%%%%%%%%%%%%%%%%%%%%%%%%%%%%%%%%%%%

%%%%%%%%%%%%%%%%%%%%%%%%%%%%%%%%%%%%%%%%%%%%%%%%%%%%%%%%%%%%%%%%%%%%%%%%%%%%%%%%
\begin{frame}[fragile]
  \frametitle{Invalid Use of Ordered}

  \begin{changemargin}{1.5cm}

  \begin{lstlisting}
void work(int i) {
  printf("i = %d\n", i);
}
...
int i;
#pragma omp for ordered
for (i = 0; i < 20; ++i) {

  #pragma omp ordered
  work(i);

  // Each iteration of the loop has 2 "ordered" clauses!
  #pragma omp ordered 
  work(i + 100);
}
  \end{lstlisting}
  \end{changemargin}


\end{frame}
%%%%%%%%%%%%%%%%%%%%%%%%%%%%%%%%%%%%%%%%%%%%%%%%%%%%%%%%%%%%%%%%%%%%%%%%%%%%%%%%

%%%%%%%%%%%%%%%%%%%%%%%%%%%%%%%%%%%%%%%%%%%%%%%%%%%%%%%%%%%%%%%%%%%%%%%%%%%%%%%%
\begin{frame}[fragile]
  \frametitle{Valid Use of Ordered}

  \begin{changemargin}{2.5cm}
  \begin{lstlisting}
void work(int i) {
  printf("i = %d\n", i);
}
...
int i;
#pragma omp for ordered
for (i = 0; i < 20; ++i) {
  if (i <= 10) {
    #pragma omp ordered
    work(i);
  }
  if (i > 10) {
    // two ordered clauses are mutually-exclusive
    #pragma omp ordered
    work(i+100);
  }
}
  \end{lstlisting}
  \end{changemargin}

  \begin{changemargin}{.5cm}
  \begin{itemize}
    \item <2-> {\bf Note:} if we change {\tt i \textgreater\- 10} to
      {\tt i \textgreater\- 9}, use becomes invalid
  \end{itemize}
  \end{changemargin}


\end{frame}
%%%%%%%%%%%%%%%%%%%%%%%%%%%%%%%%%%%%%%%%%%%%%%%%%%%%%%%%%%%%%%%%%%%%%%%%%%%%%%%%



%%%%%%%%%%%%%%%%%%%%%%%%%%%%%%%%%%%%%%%%%%%%%%%%%%%%%%%%%%%%%%%%%%%%%%%%%%%%%%%%
\begin{frame}[fragile]
  \frametitle{Tying It All Together}

  \begin{changemargin}{1cm}
  \begin{lstlisting}
#include <omp.h>
#include <stdio.h>

int main(int argc, char *argv[])
{
    int j, k, a;
    #pragma omp parallel num_threads(2)
    {
        #pragma omp for collapse(2) ordered private(j,k) \
                        schedule(static,3)
        for (k = 1; k <= 3; ++k)
            for (j = 1; j <= 2; ++j) {
                #pragma omp ordered
                printf("t[%d] k=%d j=%d\n",
                       omp_get_thread_num(),
                       k, j);
            }
    }
    return 0;
}
  \end{lstlisting}
  \end{changemargin}

\end{frame}
%%%%%%%%%%%%%%%%%%%%%%%%%%%%%%%%%%%%%%%%%%%%%%%%%%%%%%%%%%%%%%%%%%%%%%%%%%%%%%%%

%%%%%%%%%%%%%%%%%%%%%%%%%%%%%%%%%%%%%%%%%%%%%%%%%%%%%%%%%%%%%%%%%%%%%%%%%%%%%%%%
\begin{frame}[fragile]
  \frametitle{Output of Previous Example}

  \begin{changemargin}{2.5cm}

  \begin{lstlisting}
t[0] k=1 j=1
t[0] k=1 j=2
t[0] k=2 j=1
t[1] k=2 j=2
t[1] k=3 j=1
t[1] k=3 j=2
  \end{lstlisting}
  \vfill
  {\bf Note:} output is determinstic; program will run two threads as
  long as thread limit is at least 2.
  \end{changemargin}

\end{frame}
%%%%%%%%%%%%%%%%%%%%%%%%%%%%%%%%%%%%%%%%%%%%%%%%%%%%%%%%%%%%%%%%%%%%%%%%%%%%%%%%

%%%%%%%%%%%%%%%%%%%%%%%%%%%%%%%%%%%%%%%%%%%%%%%%%%%%%%%%%%%%%%%%%%%%%%%%%%%%%%%%
\begin{frame}[fragile]
  \frametitle{Parallel Loop}

  \begin{changemargin}{2.5cm}
  \begin{center}
    {\tt \#pragma omp }{\bf parallel for} {\it [clause [[,] clause]*]}
  \end{center}~\\

  Basically shorthand for:
  
  \begin{lstlisting}
#pragma omp parallel
{
    #pragma omp for
    {

    }
}
  \end{lstlisting}

  Allowed Clauses: everything allowed by {\tt parallel} and {\tt for}, except
  {\bf nowait}.
  \end{changemargin}

\end{frame}
%%%%%%%%%%%%%%%%%%%%%%%%%%%%%%%%%%%%%%%%%%%%%%%%%%%%%%%%%%%%%%%%%%%%%%%%%%%%%%%%

%%%%%%%%%%%%%%%%%%%%%%%%%%%%%%%%%%%%%%%%%%%%%%%%%%%%%%%%%%%%%%%%%%%%%%%%%%%%%%%%
\begin{frame}[fragile]
  \frametitle{Sections}

  \begin{changemargin}{2cm}
  \begin{center}
    {\tt \#pragma omp }{\bf sections} {\it [clause [[,] clause]*]}
  \end{center}

  Allowed Clauses: {\bf private, firstprivate, lastprivate, reduction, nowait}.
  \vfill
  Each {\bf sections} directive must contain one or more {\bf section} directive:
  \begin{center}
    {\tt \#pragma omp }{\bf section}
  \end{center}
  \vfill
  
  \begin{itemize}
    \item Sections distributed among current team of threads.
    \item {\bf Sections} statically limit parallelism to the number of
      sections lexically in the code.
  \end{itemize}
  \end{changemargin}

\end{frame}
%%%%%%%%%%%%%%%%%%%%%%%%%%%%%%%%%%%%%%%%%%%%%%%%%%%%%%%%%%%%%%%%%%%%%%%%%%%%%%%%

%%%%%%%%%%%%%%%%%%%%%%%%%%%%%%%%%%%%%%%%%%%%%%%%%%%%%%%%%%%%%%%%%%%%%%%%%%%%%%%%
\begin{frame}[fragile]
  \frametitle{Parallel Sections}

  \begin{changemargin}{2.5cm}

  \begin{center}
    {\tt \#pragma omp }{\bf parallel sections} {\it [clause [[,] clause]*]}
  \end{center}
  \vfill
  Again, basically shorthand for:
  
  \begin{lstlisting}
#pragma omp parallel
{
    #pragma omp sections
    {

    }
}
  \end{lstlisting}
  \vfill
  Allowed Clauses: everything allowed by {\tt parallel} and {\tt sections}, except
  {\bf nowait}.
  \end{changemargin}

\end{frame}
%%%%%%%%%%%%%%%%%%%%%%%%%%%%%%%%%%%%%%%%%%%%%%%%%%%%%%%%%%%%%%%%%%%%%%%%%%%%%%%%

%%%%%%%%%%%%%%%%%%%%%%%%%%%%%%%%%%%%%%%%%%%%%%%%%%%%%%%%%%%%%%%%%%%%%%%%%%%%%%%%
\begin{frame}[fragile]
  \frametitle{Single}

  \begin{changemargin}{1.5cm}
  \begin{center}
    {\tt \#pragma omp }{\bf single}
  \end{center}
    Only a single thread executes the region.\\[1em]
    Not guaranteed to be the master thread.\\[2em]
  Allowed Clauses: {\bf private, firstprivate, copyprivate, nowait}.
  \begin{itemize}
    \item Must not use {\bf copyprivate} with {\bf nowait}
  \end{itemize}
  \end{changemargin}

\end{frame}
%%%%%%%%%%%%%%%%%%%%%%%%%%%%%%%%%%%%%%%%%%%%%%%%%%%%%%%%%%%%%%%%%%%%%%%%%%%%%%%%

%%%%%%%%%%%%%%%%%%%%%%%%%%%%%%%%%%%%%%%%%%%%%%%%%%%%%%%%%%%%%%%%%%%%%%%%%%%%%%%%
\begin{frame}[fragile]
  \frametitle{Barrier}

  \begin{changemargin}{2.5cm}
  \begin{center}
    {\tt \#pragma omp }{\bf barrier}
  \end{center}

     Waits for all the threads in the team to reach the barrier before
      continuing.\\[1em]
     In other words---a synchronization point.\\[1em]
     Loops, Sections, Single have an implicit barrer at the end of their
      region (unless you use {\bf nowait}).\\[1em]
     Cannot appear naked as a then-clause or else-clause; wrap the barrier with a {\tt \{ \}}.\\[2em]
     Also available in pthreads as {\tt pthread\_barrier}.
  \end{changemargin}

\end{frame}
%%%%%%%%%%%%%%%%%%%%%%%%%%%%%%%%%%%%%%%%%%%%%%%%%%%%%%%%%%%%%%%%%%%%%%%%%%%%%%%%

%%%%%%%%%%%%%%%%%%%%%%%%%%%%%%%%%%%%%%%%%%%%%%%%%%%%%%%%%%%%%%%%%%%%%%%%%%%%%%%%
\begin{frame}[fragile]
  \frametitle{Master}

  \begin{changemargin}{2.5cm}
  \begin{center}
    {\tt \#pragma omp }{\bf master}
  \end{center}
~\\
    Similar to the {\bf single} directive.\\[1em]
    Master thread (and only the master thread) is guaranteed to enter this region.\\[1em]
    No implied barriers, no clauses.\\[1em]
  \end{changemargin}

\end{frame}
%%%%%%%%%%%%%%%%%%%%%%%%%%%%%%%%%%%%%%%%%%%%%%%%%%%%%%%%%%%%%%%%%%%%%%%%%%%%%%%%

%%%%%%%%%%%%%%%%%%%%%%%%%%%%%%%%%%%%%%%%%%%%%%%%%%%%%%%%%%%%%%%%%%%%%%%%%%%%%%%%
\begin{frame}[fragile]
  \frametitle{Critical}

  \begin{changemargin}{2.5cm}
  \begin{center}
    {\tt \#pragma omp }{\bf critical} {\it [(name)]}
  \end{center}~\\

    The enclosed region is guaranteed to only run one thread at a time
      (on a per-name basis).\\[1em]
    Same as a block of code in Pthreads surrounded by a {\tt mutex} lock
      and unlock.
  \end{changemargin}

\end{frame}
%%%%%%%%%%%%%%%%%%%%%%%%%%%%%%%%%%%%%%%%%%%%%%%%%%%%%%%%%%%%%%%%%%%%%%%%%%%%%%%%

%%%%%%%%%%%%%%%%%%%%%%%%%%%%%%%%%%%%%%%%%%%%%%%%%%%%%%%%%%%%%%%%%%%%%%%%%%%%%%%%
\begin{frame}[fragile]
  \frametitle{Atomic}

  \begin{changemargin}{1.5cm}
  \begin{center}
    {\tt \#pragma omp }{\bf atomic} [{\bf read $\mid$ write $\mid$ update $\mid$ capture}]\\
    {\it expression-stmt}
  \end{center}~\\
  
    Ensures a specific storage location is updated atomically.\\[1em]
    More efficient than using critical sections (or else why would they
      include it?)\\[2em]

  \end{changemargin}

\end{frame}
%%%%%%%%%%%%%%%%%%%%%%%%%%%%%%%%%%%%%%%%%%%%%%%%%%%%%%%%%%%%%%%%%%%%%%%%%%%%%%%%

%%%%%%%%%%%%%%%%%%%%%%%%%%%%%%%%%%%%%%%%%%%%%%%%%%%%%%%%%%%%%%%%%%%%%%%%%%%%%%%%
\begin{frame}[fragile]
  \frametitle{Atomic Capture}

  \begin{changemargin}{1cm}

  {\bf read expression:} {\tt v = x;}\\[1em]
  
  {\bf write expression:} {\tt x = expr;}\\[1em]

  {\bf update expression:} {\tt x++;} {\tt x{-}-;} {\tt ++x;} {\tt -{-}x;}\\
  \hspace{9em}{\tt x binop= expr;} {\tt x = x binop expr;}\\[1em]

    {\tt expr} must not access the same location as {\tt v} or {\tt x}.\\
    {\tt v} and {\tt x} must not access the same location; must be
      primitives.\\
    All operations to {\tt x} are atomic.\\[2em]

  {\bf capture expression:} {\tt v = x++;} {\tt v = x{-}-;} {\tt v = ++x;}
  {\tt v = -{-}x;}\\ \hspace{9.2em}{\tt v = x binop= expr;}

  Performs the indicated update. Also stores the original or
  final value computed.
  \end{changemargin}

\end{frame}
%%%%%%%%%%%%%%%%%%%%%%%%%%%%%%%%%%%%%%%%%%%%%%%%%%%%%%%%%%%%%%%%%%%%%%%%%%%%%%%%

%%%%%%%%%%%%%%%%%%%%%%%%%%%%%%%%%%%%%%%%%%%%%%%%%%%%%%%%%%%%%%%%%%%%%%%%%%%%%%%%
\begin{frame}[fragile]
  \frametitle{Atomic Capture}

  \begin{changemargin}{1cm}
  \begin{center}
    {\tt \#pragma omp }{\bf atomic capture}\\
    {\it structured-block}
  \end{center}

    Structured blocks are equivalent to the expanded expressions.
  \end{changemargin}

\end{frame}
%%%%%%%%%%%%%%%%%%%%%%%%%%%%%%%%%%%%%%%%%%%%%%%%%%%%%%%%%%%%%%%%%%%%%%%%%%%%%%%%

%%%%%%%%%%%%%%%%%%%%%%%%%%%%%%%%%%%%%%%%%%%%%%%%%%%%%%%%%%%%%%%%%%%%%%%%%%%%%%%%
\begin{frame}[fragile]
  \frametitle{Other Directives}

  \begin{changemargin}{2.5cm}

  \begin{itemize}
    \item {\bf task}
    \item {\bf taskyield}
    \item {\bf taskwait}
    \item {\bf flush}
  \end{itemize}
  ~\\[1em]
  We'll get into these next lecture.
  \end{changemargin}

\end{frame}
%%%%%%%%%%%%%%%%%%%%%%%%%%%%%%%%%%%%%%%%%%%%%%%%%%%%%%%%%%%%%%%%%%%%%%%%%%%%%%%%

%%%%%%%%%%%%%%%%%%%%%%%%%%%%%%%%%%%%%%%%%%%%%%%%%%%%%%%%%%%%%%%%%%%%%%%%%%%%%%%%
\begin{frame}[fragile]
  \frametitle{{\tt firstprivate} and {\tt lastprivate} Clauses}

  \begin{changemargin}{2.5cm}

  Pthreads pseudocode for {\bf firstprivate} clause:
  \begin{lstlisting}
int x;

void* run(void* arg) {
    int thread_x = x;
    // use thread_x
}
  \end{lstlisting}

  Pthread pseudocode for {\bf lastprivate} clause:
  \begin{lstlisting}
int x;

void* run(void* arg) {
    int thread_x;
    // use thread_x
    if (last_iteration) {
        x = thread_x;
    }
}
  \end{lstlisting}

  \begin{itemize}
    \item Same value as if the loop executed sequentially.
  \end{itemize}
  \end{changemargin}

\end{frame}
%%%%%%%%%%%%%%%%%%%%%%%%%%%%%%%%%%%%%%%%%%%%%%%%%%%%%%%%%%%%%%%%%%%%%%%%%%%%%%%%

%%%%%%%%%%%%%%%%%%%%%%%%%%%%%%%%%%%%%%%%%%%%%%%%%%%%%%%%%%%%%%%%%%%%%%%%%%%%%%%%
\begin{frame}[fragile]
  \frametitle{{\tt copyin}, {\tt copyprivate} and {\tt default} Clauses}

  \begin{changemargin}{2cm}
  \begin{itemize}
    \item {\bf copyin} like firstprivate, but for threadprivate variables.
  \end{itemize}

  Pthreads pseudocode for {\bf copyin}:
  \begin{lstlisting}[language=C]
int x;
int x[NUM_THREADS];

void* run(void* arg) {
  x[thread_num] = x;
  // use x[thread_num]
}
  \end{lstlisting}
~\\
    {\bf copyprivate} is only used with {\bf single}.
      \begin{itemize}
        \item Copies the specified private variables from the thread to all other
          threads.
        \item Cannot be used with {\bf nowait}.
      \end{itemize}
  ~\\

    {\bf default(shared)} makes all variables shared; \\{\bf default(none)} prevents sharing by default.
  \end{changemargin}

\end{frame}
%%%%%%%%%%%%%%%%%%%%%%%%%%%%%%%%%%%%%%%%%%%%%%%%%%%%%%%%%%%%%%%%%%%%%%%%%%%%%%%%

\part{Runtime Library Routines}
\frame{\partpage}
%%%%%%%%%%%%%%%%%%%%%%%%%%%%%%%%%%%%%%%%%%%%%%%%%%%%%%%%%%%%%%%%%%%%%%%%%%%%%%%%
\begin{frame}[fragile]
  \frametitle{Execution Environment}

  To use the runtime library you need to {\tt \#include <omp.h>}.\\[1em]

  \begin{itemize}
    \item {\tt int omp\_get\_num\_procs();}\\
         ~~number of processors in the system.
    \item {\tt int omp\_get\_thread\_num();}\\
         ~~thread number of the currently executing thread\\
         ~~(master thread will return 0).
    \item {\tt int omp\_in\_parallel();}\\
         ~~whether or not currently in a parallel region.
    \item {\tt int omp\_get\_num\_threads();}\\
         ~~number of threads in current team.
  \end{itemize}
\end{frame}
%%%%%%%%%%%%%%%%%%%%%%%%%%%%%%%%%%%%%%%%%%%%%%%%%%%%%%%%%%%%%%%%%%%%%%%%%%%%%%%%

%%%%%%%%%%%%%%%%%%%%%%%%%%%%%%%%%%%%%%%%%%%%%%%%%%%%%%%%%%%%%%%%%%%%%%%%%%%%%%%%
\begin{frame}
  \frametitle{Locks}

  \begin{changemargin}{2.5cm}

  Two types of locks:
  
  \begin{itemize}
    \item Simple: cannot be acquired if it is already held by the task trying to acquire it.
    \item Nested: can be acquired multiple times by the same task before being released (like Java).
  \end{itemize}

  ~\\
  Usage similar to Pthreads:

  \begin{center}
    \begin{tabular}{r | r}
      {\tt omp\_init\_lock} & {\tt omp\_init\_nest\_lock}\\
      {\tt omp\_destroy\_lock} & {\tt omp\_destroy\_nest\_lock}\\
      {\tt omp\_set\_lock} & {\tt omp\_set\_nest\_lock}\\
      {\tt omp\_unset\_lock} & {\tt omp\_unset\_nest\_lock}\\
      {\tt omp\_test\_lock} & {\tt omp\_test\_nest\_lock}\\
    \end{tabular}
  \end{center}
  \end{changemargin}


\end{frame}
%%%%%%%%%%%%%%%%%%%%%%%%%%%%%%%%%%%%%%%%%%%%%%%%%%%%%%%%%%%%%%%%%%%%%%%%%%%%%%%%

%%%%%%%%%%%%%%%%%%%%%%%%%%%%%%%%%%%%%%%%%%%%%%%%%%%%%%%%%%%%%%%%%%%%%%%%%%%%%%%%
\begin{frame}
  \frametitle{Timing}

  \begin{changemargin}{2.5cm}
  \begin{itemize}
    \item {\tt double omp\_get\_wtime();}\\
        ~~elapsed wall clock time in seconds\\ 
        ~~(since some time in the past).\\[1em]
    \item {\tt double omp\_get\_wtick();}\\
        ~~precision of the timer.
  \end{itemize}
  \end{changemargin}

\end{frame}
%%%%%%%%%%%%%%%%%%%%%%%%%%%%%%%%%%%%%%%%%%%%%%%%%%%%%%%%%%%%%%%%%%%%%%%%%%%%%%%%

%%%%%%%%%%%%%%%%%%%%%%%%%%%%%%%%%%%%%%%%%%%%%%%%%%%%%%%%%%%%%%%%%%%%%%%%%%%%%%%%
\begin{frame}
  \frametitle{Other Routines}

  \begin{changemargin}{2cm}
  Might see these in later lectures. Included for
  completeness:

  \begin{itemize}
    \item {\tt int omp\_get\_level();}
    \item {\tt int omp\_get\_active\_level();}
    \item {\tt int omp\_get\_ancestor\_thread\_num(int level);}
    \item {\tt int omp\_get\_team\_size(int level);}
    \item {\tt int omp\_in\_final();}
  \end{itemize}
\end{changemargin}

\end{frame}
%%%%%%%%%%%%%%%%%%%%%%%%%%%%%%%%%%%%%%%%%%%%%%%%%%%%%%%%%%%%%%%%%%%%%%%%%%%%%%%%

\part{Internal Control Variables}
\frame{\partpage}
%%%%%%%%%%%%%%%%%%%%%%%%%%%%%%%%%%%%%%%%%%%%%%%%%%%%%%%%%%%%%%%%%%%%%%%%%%%%%%%%
\begin{frame}
  \frametitle{Internal Control Variables}

  \begin{changemargin}{2cm}

    Control how OpenMP handles threads.\\[1em]
    Can be set with clauses, runtime routines, environment
      variables, or just from defaults.\\[1em]
    Routines will be represented as all-lower-case, environment variables
      as all-upper-case.\\[2em]

    Clause \textgreater\- Routine \textgreater\- Environment Variable \textgreater\- Default Value\\[1em]

    All values (except 1) are implementation defined.
  \end{changemargin}

\end{frame}
%%%%%%%%%%%%%%%%%%%%%%%%%%%%%%%%%%%%%%%%%%%%%%%%%%%%%%%%%%%%%%%%%%%%%%%%%%%%%%%%

%%%%%%%%%%%%%%%%%%%%%%%%%%%%%%%%%%%%%%%%%%%%%%%%%%%%%%%%%%%%%%%%%%%%%%%%%%%%%%%%
\begin{frame}
  \frametitle{Operation of Parallel Regions (1)}

  \begin{changemargin}{1.5cm}

  {\bf dyn-var}
  \begin{itemize}
    \item is dynamic adjustment of the number of threads enabled?
    \item {\bf Set by:} {\tt OMP\_DYNAMIC omp\_set\_dynamic}
    \item {\bf Get by:} {\tt omp\_get\_dynamic}
  \end{itemize}
  \vfill
  {\bf nest-var}
  \begin{itemize}
    \item is nested parallelism enabled?
    \item {\bf Set by:} {\tt OMP\_NESTED omp\_set\_nested}
    \item {\bf Get by:} {\tt omp\_get\_nested}
    \item Default value: {\tt false}
  \end{itemize}
  \end{changemargin}

\end{frame}
%%%%%%%%%%%%%%%%%%%%%%%%%%%%%%%%%%%%%%%%%%%%%%%%%%%%%%%%%%%%%%%%%%%%%%%%%%%%%%%%

%%%%%%%%%%%%%%%%%%%%%%%%%%%%%%%%%%%%%%%%%%%%%%%%%%%%%%%%%%%%%%%%%%%%%%%%%%%%%%%%
\begin{frame}
  \frametitle{Operation of Parallel Regions (2)}
  
  \begin{changemargin}{1.5cm}
  {\bf thread-limit-var}
  \begin{itemize}
    \item maximum number of threads in the program
    \item {\bf Set by:} {\tt OMP\_NUM\_THREADS omp\_set\_num\_threads}
    \item {\bf Get by:} {\tt omp\_get\_max\_threads}
  \end{itemize}
  \vfill
  {\bf max-active-levels-var}
  \begin{itemize}
    \item Maximum number of nested active parallel regions
    \item {\bf Set by:} {\tt OMP\_MAX\_ACTIVE\_LEVELS}\\
      \hspace{3.8em}{\tt omp\_set\_max\_active\_levels}
    \item {\bf Get by:} {\tt omp\_get\_max\_active\_levels}
  \end{itemize}
  \end{changemargin}

\end{frame}
%%%%%%%%%%%%%%%%%%%%%%%%%%%%%%%%%%%%%%%%%%%%%%%%%%%%%%%%%%%%%%%%%%%%%%%%%%%%%%%%

%%%%%%%%%%%%%%%%%%%%%%%%%%%%%%%%%%%%%%%%%%%%%%%%%%%%%%%%%%%%%%%%%%%%%%%%%%%%%%%%
\begin{frame}
  \frametitle{Operation of Parallel Regions/Loops}

  \begin{changemargin}{1.5cm}
  {\bf nthreads-var}
  \begin{itemize}
    \item number of threads requested for parallel regions.
    \item {\bf Set by:} {\tt OMP\_NUM\_THREADS omp\_set\_num\_threads}
    \item {\bf Get by:} {\tt omp\_get\_max\_threads}
  \end{itemize}
  \vfill
  {\bf run-sched-var}
  \begin{itemize}
    \item {\bf schedule} that the runtime schedule clause uses for loops.
    \item {\bf Set by:} {\tt OMP\_SCHEDULE omp\_set\_schedule}
    \item {\bf Get by:} {\tt omp\_get\_schedule}
  \end{itemize}
  \end{changemargin}

\end{frame}
%%%%%%%%%%%%%%%%%%%%%%%%%%%%%%%%%%%%%%%%%%%%%%%%%%%%%%%%%%%%%%%%%%%%%%%%%%%%%%%%

%%%%%%%%%%%%%%%%%%%%%%%%%%%%%%%%%%%%%%%%%%%%%%%%%%%%%%%%%%%%%%%%%%%%%%%%%%%%%%%%
\begin{frame}
  \frametitle{Program Execution}

  \begin{changemargin}{1.5cm}
  {\bf bind-var}
  \begin{itemize}
    \item Controls binding of threads to processors.
    \item {\bf Set by:} {\tt OMP\_PROC\_BIND}
  \end{itemize}
  \vfill
  {\bf stacksize-var}
  \begin{itemize}
    \item Controls stack size for threads.
    \item {\bf Set by:} {\tt OMP\_STACK\_SIZE}
  \end{itemize}
  \vfill
  {\bf wait-policy-var}
  \begin{itemize}
    \item Controls desired behaviour of waiting threads.
    \item {\bf Set by:} {\tt OMP\_WAIT\_POLICY}
  \end{itemize}
  \end{changemargin}

\end{frame}
%%%%%%%%%%%%%%%%%%%%%%%%%%%%%%%%%%%%%%%%%%%%%%%%%%%%%%%%%%%%%%%%%%%%%%%%%%%%%%%%

\part{Summary}
\frame{\partpage}
%%%%%%%%%%%%%%%%%%%%%%%%%%%%%%%%%%%%%%%%%%%%%%%%%%%%%%%%%%%%%%%%%%%%%%%%%%%%%%%%
\begin{frame}[fragile]
  \frametitle{Summary}

  \begin{changemargin}{1.5cm}
  \begin{itemize}
    \item Main concepts
      \begin{itemize}
        \item {\bf parallel}
        \item {\bf for} ({\bf ordered})
        \item {\bf sections}
        \item {\bf single}
        \item {\bf master}
      \end{itemize}
    \item Synchronization
      \begin{itemize}
        \item {\bf barrier}
        \item {\bf critical}
        \item {\bf atomic}
      \end{itemize}
    \item Data sharing: {\bf private}, {\bf shared}, {\bf threadprivate}
    \item Should be able to use OpenMP effectively with a reference.
  \end{itemize}~\\

  {\bf Reference Card}

  \url{http://openmp.org/mp-documents/OpenMP3.1-CCard.pdf}
  \end{changemargin}

\end{frame}
%%%%%%%%%%%%%%%%%%%%%%%%%%%%%%%%%%%%%%%%%%%%%%%%%%%%%%%%%%%%%%%%%%%%%%%%%%%%%%%%

%% _OPENMP

%% Clause > Runtime Library Routine > Environment Variable > Default Value

%%   \begin{itemize}
%%     \item By default, anything outside of the compound statement is {\bf shared}
%%       anything inside (and loop variables) is {\bf private}
%%   \end{itemize}


%% %%%%%%%%%%%%%%%%%%%%%%%%%%%%%%%%%%%%%%%%%%%%%%%%%%%%%%%%%%%%%%%%%%%%%%%%%%%%%%%%
%% \begin{frame}[fragile]
%%   \frametitle{Task}

%% \end{frame}
%% %%%%%%%%%%%%%%%%%%%%%%%%%%%%%%%%%%%%%%%%%%%%%%%%%%%%%%%%%%%%%%%%%%%%%%%%%%%%%%%%

%% %%%%%%%%%%%%%%%%%%%%%%%%%%%%%%%%%%%%%%%%%%%%%%%%%%%%%%%%%%%%%%%%%%%%%%%%%%%%%%%%
%% \begin{frame}[fragile]
%%   \frametitle{Taskyield}

%% \end{frame}
%% %%%%%%%%%%%%%%%%%%%%%%%%%%%%%%%%%%%%%%%%%%%%%%%%%%%%%%%%%%%%%%%%%%%%%%%%%%%%%%%%

%% %%%%%%%%%%%%%%%%%%%%%%%%%%%%%%%%%%%%%%%%%%%%%%%%%%%%%%%%%%%%%%%%%%%%%%%%%%%%%%%%
%% \begin{frame}[fragile]
%%   \frametitle{Taskwait}

%% \end{frame}
%% %%%%%%%%%%%%%%%%%%%%%%%%%%%%%%%%%%%%%%%%%%%%%%%%%%%%%%%%%%%%%%%%%%%%%%%%%%%%%%%%


%% %%%%%%%%%%%%%%%%%%%%%%%%%%%%%%%%%%%%%%%%%%%%%%%%%%%%%%%%%%%%%%%%%%%%%%%%%%%%%%%%
%% \begin{frame}[fragile]
%%   \frametitle{Flush}

%%   \begin{center}
%%     {\tt \#pragma omp }{\bf flush}
%%   \end{center}

  
%%   \begin{itemize}
%%     \item Makes the thread's temporary view of memory consistent with main
%%       memory
%%     \item Enforces an order on the memory operations of the variables
%%   \end{itemize}

%% \end{frame}
%% %%%%%%%%%%%%%%%%%%%%%%%%%%%%%%%%%%%%%%%%%%%%%%%%%%%%%%%%%%%%%%%%%%%%%%%%%%%%%%%%


\end{document}
