
\documentclass[letterpaper,hide notes,xcolor={table,svgnames},pdftex,10pt]{beamer}
\def\showexamples{t}


%\usepackage[svgnames]{xcolor}

%% Demo talk
%\documentclass[letterpaper,notes=show]{beamer}

\usecolortheme{crane}
\setbeamertemplate{navigation symbols}{}

\usetheme{MyPittsburgh}
%\usetheme{Frankfurt}

%\usepackage{tipa}

\usepackage{hyperref}
\usepackage{graphicx,xspace}
\usepackage[normalem]{ulem}
\usepackage{multicol}
\usepackage{amsmath,amssymb,amsthm,graphicx,xspace}
\newcommand\SF[1]{$\bigstar$\footnote{SF: #1}}

\usepackage[default]{sourcesanspro}
\usepackage[T1]{fontenc}
\usepackage[scaled]{beramono}
\usepackage{tikzpagenodes}

\newcounter{tmpnumSlide}
\newcounter{tmpnumNote}


% old question code
%\newcommand\question[1]{{$\bigstar$ \small \onlySlide{2}{#1}}}
% \newcommand\nquestion[1]{\ifdefined \presentationonly \textcircled{?} \fi \note{\par{\Large \textbf{?}} #1}}
% \newcommand\nanswer[1]{\note{\par{\Large \textbf{A}} #1}}


 \newcommand\mnote[1]{%
   \addtocounter{tmpnumSlide}{1}
   \ifdefined\showcues {~\tiny\fbox{\arabic{tmpnumSlide}}}\fi
   \note{\setlength{\parskip}{1ex}\addtocounter{tmpnumNote}{1}\textbf{\Large \arabic{tmpnumNote}:} {#1\par}}}

\newcommand\mmnote[1]{\note{\setlength{\parskip}{1ex}#1\par}}

%\newcommand\mnote[2][]{\ifdefined\handoutwithnotes {~\tiny\fbox{#1}}\fi
% \note{\setlength{\parskip}{1ex}\textbf{\Large #1:} #2\par}}

%\newcommand\mnote[2][]{{\tiny\fbox{#1}} \note{\setlength{\parskip}{1ex}\textbf{\Large #1:} #2\par}}

\newcommand\mquestion[2]{{~\color{red}\fbox{?}}\note{\setlength{\parskip}{1ex}\par{\Large \textbf{?}} #1} \note{\setlength{\parskip}{1ex}\par{\Large \textbf{A}} #2\par}\ifdefined \presentationonly \pause \fi}

\newcommand\blackboard[1]{%
\ifdefined   \showblackboard
  {#1}
  \else {\begin{center} \fbox{\colorbox{blue!30}{%
         \begin{minipage}{.95\linewidth}%
           \hspace{\stretch{1}} Some space intentionally left blank; done at the blackboard.%
         \end{minipage}}}\end{center}}%
         \fi%
}



%\newcommand\q{\tikz \node[thick,color=black,shape=circle]{?};}
%\newcommand\q{\ifdefined \presentationonly \textcircled{?} \fi}

\usepackage{listings}
\lstset{basicstyle=\footnotesize\ttfamily,
	breaklines=true,
	aboveskip=15pt,
  	belowskip=15pt,
	frame=lines,
	numbers=left, basicstyle=\scriptsize, numberstyle=\tiny, stepnumber=0, numbersep=2pt
}

\usepackage{siunitx}
\newcommand\sius[1]{\num[group-separator = {,}]{#1}\si{\micro\second}}
\newcommand\sims[1]{\num[group-separator = {,}]{#1}\si{\milli\second}}
\newcommand\sins[1]{\num[group-separator = {,}]{#1}\si{\nano\second}}
\sisetup{group-separator = {,}, group-digits = true}

%% -------------------- tikz --------------------
\usepackage{tikz}
\usetikzlibrary{positioning}
\usetikzlibrary{arrows,backgrounds,automata,decorations.shapes,decorations.pathmorphing,decorations.markings,decorations.text,decorations.pathreplacing}

\tikzstyle{place}=[circle,draw=blue!50,fill=blue!20,thick, inner sep=0pt,minimum size=6mm]
\tikzstyle{transition}=[rectangle,draw=black!50,fill=black!20,thick, inner sep=0pt,minimum size=4mm]

\tikzstyle{block}=[rectangle,draw=black, thick, inner sep=5pt]
\tikzstyle{bullet}=[circle,draw=black, fill=black, thin, inner sep=2pt]

\tikzstyle{pre}=[<-,shorten <=1pt,>=stealth',semithick]
\tikzstyle{post}=[->,shorten >=1pt,>=stealth',semithick]
\tikzstyle{bi}=[<->,shorten >=1pt,shorten <=1pt, >=stealth',semithick]

\tikzstyle{mut}=[-,>=stealth',semithick]

\tikzstyle{treereset}=[dashed,->, shorten >=1pt,>=stealth',thin]

\usepackage{ifmtarg}
\usepackage{xifthen}
\makeatletter
% new counter to now which frame it is within the sequence
\newcounter{multiframecounter}
% initialize buffer for previously used frame title
\gdef\lastframetitle{\textit{undefined}}
% new environment for a multi-frame
\newenvironment{multiframe}[1][]{%
\ifthenelse{\isempty{#1}}{%
% if no frame title was set via optional parameter,
% only increase sequence counter by 1
\addtocounter{multiframecounter}{1}%
}{%
% new frame title has been provided, thus
% reset sequence counter to 1 and buffer frame title for later use
\setcounter{multiframecounter}{1}%
\gdef\lastframetitle{#1}%
}%
% start conventional frame environment and
% automatically set frame title followed by sequence counter
\begin{frame}%
\frametitle{\lastframetitle~{\normalfont(\arabic{multiframecounter})}}%
}{%
\end{frame}%
}
\makeatother

\makeatletter
\newdimen\tu@tmpa%
\newdimen\ydiffl%
\newdimen\xdiffl%
\newcommand\ydiff[2]{%
    \coordinate (tmpnamea) at (#1);%
    \coordinate (tmpnameb) at (#2);%
    \pgfextracty{\tu@tmpa}{\pgfpointanchor{tmpnamea}{center}}%
    \pgfextracty{\ydiffl}{\pgfpointanchor{tmpnameb}{center}}%
    \advance\ydiffl by -\tu@tmpa%
}
\newcommand\xdiff[2]{%
    \coordinate (tmpnamea) at (#1);%
    \coordinate (tmpnameb) at (#2);%
    \pgfextractx{\tu@tmpa}{\pgfpointanchor{tmpnamea}{center}}%
    \pgfextractx{\xdiffl}{\pgfpointanchor{tmpnameb}{center}}%
    \advance\xdiffl by -\tu@tmpa%
}
\makeatother
\newcommand{\copyrightbox}[3][r]{%
\begin{tikzpicture}%
\node[inner sep=0pt,minimum size=2em](ciimage){#2};
\usefont{OT1}{phv}{n}{n}\fontsize{4}{4}\selectfont
\ydiff{ciimage.south}{ciimage.north}
\xdiff{ciimage.west}{ciimage.east}
\ifthenelse{\equal{#1}{r}}{%
\node[inner sep=0pt,right=1ex of ciimage.south east,anchor=north west,rotate=90]%
{\raggedleft\color{black!50}\parbox{\the\ydiffl}{\raggedright{}#3}};%
}{%
\ifthenelse{\equal{#1}{l}}{%
\node[inner sep=0pt,right=1ex of ciimage.south west,anchor=south west,rotate=90]%
{\raggedleft\color{black!50}\parbox{\the\ydiffl}{\raggedright{}#3}};%
}{%
\node[inner sep=0pt,below=1ex of ciimage.south west,anchor=north west]%
{\raggedleft\color{black!50}\parbox{\the\xdiffl}{\raggedright{}#3}};%
}
}
\end{tikzpicture}
}


%% --------------------

%\usepackage[excludeor]{everyhook}
%\PushPreHook{par}{\setbox0=\lastbox\llap{MUH}}\box0}

%\vspace*{\stretch{1}

%\setbox0=\lastbox \llap{\textbullet\enskip}\box0}

\setlength{\parskip}{\fill}

\newcommand\noskips{\setlength{\parskip}{1ex}}
\newcommand\doskips{\setlength{\parskip}{\fill}}

\newcommand\xx{\par\vspace*{\stretch{1}}\par}
\newcommand\xxs{\par\vspace*{2ex}\par}
\newcommand\tuple[1]{\langle #1 \rangle}
\newcommand\code[1]{{\sf \footnotesize #1}}
\newcommand\ex[1]{\uline{Example:} \ifdefined \presentationonly \pause \fi
  \ifdefined\showexamples#1\xspace\else{\uline{\hspace*{2cm}}}\fi}

\newcommand\ceil[1]{\lceil #1 \rceil}


\AtBeginSection[]
{
   \begin{frame}
       \frametitle{Outline}
       \tableofcontents[currentsection]
   \end{frame}
}



\pgfdeclarelayer{edgelayer}
\pgfdeclarelayer{nodelayer}
\pgfsetlayers{edgelayer,nodelayer,main}

\tikzstyle{none}=[inner sep=0pt]
\tikzstyle{rn}=[circle,fill=Red,draw=Black,line width=0.8 pt]
\tikzstyle{gn}=[circle,fill=Lime,draw=Black,line width=0.8 pt]
\tikzstyle{yn}=[circle,fill=Yellow,draw=Black,line width=0.8 pt]
\tikzstyle{empty}=[circle,fill=White,draw=Black]
\tikzstyle{bw} = [rectangle, draw, fill=blue!20, 
    text width=4em, text centered, rounded corners, minimum height=2em]
    
    \newcommand{\CcNote}[1]{% longname
	This work is licensed under the \textit{Creative Commons #1 3.0 License}.%
}
\newcommand{\CcImageBy}[1]{%
	\includegraphics[scale=#1]{creative_commons/cc_by_30.pdf}%
}
\newcommand{\CcImageSa}[1]{%
	\includegraphics[scale=#1]{creative_commons/cc_sa_30.pdf}%
}
\newcommand{\CcImageNc}[1]{%
	\includegraphics[scale=#1]{creative_commons/cc_nc_30.pdf}%
}
\newcommand{\CcGroupBySa}[2]{% zoom, gap
	\CcImageBy{#1}\hspace*{#2}\CcImageNc{#1}\hspace*{#2}\CcImageSa{#1}%
}
\newcommand{\CcLongnameByNcSa}{Attribution-NonCommercial-ShareAlike}

\newenvironment{changemargin}[1]{% 
  \begin{list}{}{% 
    \setlength{\topsep}{0pt}% 
    \setlength{\leftmargin}{#1}% 
    \setlength{\rightmargin}{1em}
    \setlength{\listparindent}{\parindent}% 
    \setlength{\itemindent}{\parindent}% 
    \setlength{\parsep}{\parskip}% 
  }% 
  \item[]}{\end{list}} 




\title{Lecture 20 --- Compiler Optimizations }

\author{Patrick Lam \\ \small \texttt{patrick.lam@uwaterloo.ca}}
\institute{Department of Electrical and Computer Engineering \\
  University of Waterloo}
\date{\today}


\begin{document}

\begin{frame}
  \titlepage

\end{frame}


\begin{frame}
\frametitle{Interprocedural Analysis and Link-Time Optimizations}

``Are economies of scale real?''

In this context, does a
whole-program optimization really improve your program?


We'll start by first talking about some information that is critical for
whole-program optimizations.

\end{frame}

\begin{frame}
\frametitle{Alias and Pointer Analysis}

Compiler optimizations often need
to know about what parts of memory each statement reads to.  

This is
easy when talking about scalar variables which are stored on the
stack. 

This is much harder when talking about pointers or arrays
(which can alias). 

\alert{Alias analysis} helps by declaring that a
given variable {\tt p} does not alias {\tt q}.

\alert{Pointer analysis} tracks what regions of the heap each variable points to.

\end{frame}

\begin{frame}
\frametitle{Alias Analysis}

When we know that two pointers don't alias, then we know that their
effects are independent, so it's correct to move things around.

We used {\tt restrict} so that the
compiler wouldn't have to do as much pointer analysis. 

Shape analysis
builds on pointer analysis to determine that data structures are indeed
trees rather than lists.


\end{frame}

\begin{frame}
\frametitle{Call Graph}
Many interprocedural analyses require accurate call graphs. 

A \alert{call graph} is a directed graph showing relationships between
functions. 

It's easy to compute a call graph when you have C-style
function calls. 

It's much harder when you have virtual methods, as in
C++ or Java, or even C function pointers. 

In particular, you need pointer
analysis information to construct the call graph.


\end{frame}


\begin{frame}
\frametitle{Devirtualization}

This optimization attempts to convert virtual function calls to direct calls.  

Virtual method calls have the potential to be slow, because there is effectively a branch to predict. 

(In general for C++, the program must read the object's vtable.) 

Plus, virtual calls impede other optimizations.

\end{frame}

\begin{frame}[fragile]
\frametitle{Devirtualization}

Compilers can help by doing sophisticated analyses to compute the call graph and by replacing virtual method calls with nonvirtual method calls.  


  \begin{lstlisting}[language=C]
class A {
    virtual void m();
};

class B : public A {
    virtual void m();
}

int main(int argc, char *argv[]) {

    std::unique_ptr<A> t(new B);
    t.m();
}
  \end{lstlisting}

Devirtualization could eliminate vtable access; instead, we could just call B's {\tt m} method
directly.

\end{frame}

\begin{frame}
\frametitle{Devirtualization}

`Rapid Type Analysis'' analyzes the entire program, observes that
only {\tt B} objects are ever instantiated, and enables devirtualization
of the {\tt b.m()} call.

\emph{Enabled with {\tt -O2}, {\tt -O3}, or with {\tt -fdevirtualize}.}

\end{frame}

\begin{frame}
\frametitle{Inlining}

 Compilers can inline following compiler directives, but usually more based on heuristics. 
 
 Devirtualization enables more inlining. 
 
 The compiler always inlines functions marked with the {\tt always\_inline} attribute.

\noindent \emph{Enabled with {\tt -O2} and {\tt -O3}.}



\end{frame}

\begin{frame}[fragile]
\frametitle{Devirtualization and Inlining}

Obviously, inlining and devirtualization require call graphs. 

But so
does any analysis that needs to know about the heap effects of
functions that get called.


\begin{lstlisting}[language=C]
  int n;

  int f() { /* opaque */ }

  int main() {
    n = 5;
    f();
    printf("%d\n", n);
  }
\end{lstlisting}

We could propagate the constant value 5, as long as we know that {\tt
  f()} does not write to {\tt n}.


\end{frame}

\begin{frame}[fragile]
\frametitle{Tail Recursion Elimination}

This optimization is mandatory in some functional languages; we replace a call by a {\tt goto} at the compiler level.

{\small
\begin{lstlisting}[language=C]
  int bar(int N) {
    if (A(N))
      return B(N);
    else
      return bar(N);
  }
\end{lstlisting}
}

For both calls, to {\tt B} and {\tt bar}, we don't need to return control
to the calling {\tt bar()} before returning to its caller (because {\tt bar()}
is done anyway). 

This avoids function call overhead and reduces call stack use.

\noindent \emph{Enabled with {\tt -foptimize-sibling-calls}.} Also supports
sibling calls as well as tail-recursive calls.

\end{frame}

\begin{frame}
\frametitle{Link-Time Optimization}

The biggest challenge for interprocedural optimizations is scalability, so 
it fits right in as a topic of discussion for this course.

Here's an outline of how it works:
\begin{itemize}
\item local generation (parallelizable)
\item whole-program analysis (hard to parallelize!)
\item local transformations (parallelizable)
\end{itemize}

\end{frame}


\begin{frame}
\frametitle{Transformations in Code}

The transformations look like this:
\begin{itemize}
\item global decisions, local transformations:
\begin{itemize}
\item devirtualization
\item dead variable elimination/dead function elimination
\item field reordering, struct splitting/reorganization
\end{itemize}
\item global decisions, global transformations:
\begin{itemize}
\item cross-module inlining
\item virtual function inlining
\item interprocedural constant propagation
\end{itemize}
\end{itemize}


\end{frame}


\begin{frame}
\frametitle{Whole Program Analysis}

The interesting issues arise from making the whole-program analysis scalable. 

Firefox, the Linux kernel, Chromium contain tens of millions of lines of code. 

Whole-program analysis requires that all of 
this code (in IR) be available to the analysis; some summary of the code be in memory,  along with the call graph.

\end{frame}


\begin{frame}
\frametitle{Whole Program Analysis}


Since it's a whole-program analysis, any part of the program may affect other parts. 

The first problem
is getting it into memory; loading the IR for tens of millions of lines of code is a non-starter.


Clearly, anything that is more expensive than linear time can cause problems. 

Partitioning the program can help.

\end{frame}


\begin{frame}
\frametitle{gcc Whole Program Analysis}

How did gcc get better?  Avoiding unnecessary work.

\begin{itemize}
\item gcc 4.5: initial version of LTO;
\item gcc 4.6: parallelization; partitioning of the call graph (put closely-related functions together, approximate functions in other partitions); the bottleneck: streaming in types and declarations;
\item gcc 4.7--4.9: improve build times, memory usage [``chasing unnecessary data away''.]
\end{itemize}

Today's gcc, with {\tt -flto}, does work and includes
optimizations including constant propagation and function
specialization.


\end{frame}


\begin{frame}
\frametitle{It Works!}

gcc CTO appears to give 3--5\% improvements in performance, which compiler experts consider good.


Like we discussed last time, this allows developers to shift their attention from 
manual factoring of translation units to letting the compiler do it.

\end{frame}





\end{document}

