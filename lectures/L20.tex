\documentclass[letterpaper,10pt]{article}

\usepackage{enumitem}
\usepackage{titling}
\usepackage{listings,listings-rust}
\usepackage{url}
\usepackage{soul}
\usepackage{hyperref}
\usepackage{setspace}
\usepackage{subfig}
\usepackage{sectsty}
\usepackage{pdfpages}
\usepackage{colortbl}
\usepackage{multirow}
\usepackage{multicol}
\usepackage{relsize}
\usepackage{amsmath}
\usepackage{wasysym}
\usepackage{fancyvrb}
\usepackage[yyyymmdd]{datetime}
\usepackage{amsmath,amssymb,amsthm,graphicx,xspace}
\usepackage[titlenotnumbered,noend,noline]{algorithm2e}
\usepackage[compact]{titlesec}
\usepackage{XCharter}
\usepackage[T1]{fontenc}
\usepackage[scaled]{beramono}
\usepackage[normalem]{ulem}
\usepackage{booktabs}
\usepackage{tikz}
\usetikzlibrary{arrows.meta,automata,shapes,trees,matrix,chains,scopes,positioning,calc,decorations.pathreplacing}
\tikzstyle{block} = [rectangle, draw, fill=blue!20, 
    text width=2.5em, text centered, rounded corners, minimum height=2em]
\tikzstyle{bw} = [rectangle, draw, fill=blue!20, 
    text width=4em, text centered, rounded corners, minimum height=2em]

\definecolor{namerow}{cmyk}{.40,.40,.40,.40}
\definecolor{namecol}{cmyk}{.40,.40,.40,.40}
\renewcommand{\dateseparator}{-}

\let\LaTeXtitle\title
\renewcommand{\title}[1]{\LaTeXtitle{\textsf{#1}}}

\lstset{basicstyle=\footnotesize\ttfamily,breaklines=true}

\newcommand{\CPP}{C\nolinebreak\hspace{-.05em}\raisebox{.4ex}{\tiny\bf +}\nolinebreak\hspace{-.10em}\raisebox{.4ex}{\tiny\bf +}}
\def\CPP{{C\nolinebreak[4]\hspace{-.05em}\raisebox{.4ex}{\tiny\bf ++}}}

\newcommand{\handout}[5]{
  \noindent
  \begin{center}
  \framebox{
    \vbox{
      \hbox to 5.78in { {\bf ECE459: Programming for Performance } \hfill #2 }
      \vspace{4mm}
      \hbox to 5.78in { {\Large \hfill #4  \hfill} }
      \vspace{2mm}
      \hbox to 5.78in { {\em #3 \hfill \today} }
    }
  }
  \end{center}
  \vspace*{4mm}
}

\newcommand{\lecture}[3]{\handout{#1}{#2}{#3}{Lecture#1}}
\newcommand{\tuple}[1]{\ensuremath{\left\langle #1 \right\rangle}\xspace}

\addtolength{\oddsidemargin}{-1.000in}
\addtolength{\evensidemargin}{-0.500in}
\addtolength{\textwidth}{2.0in}
\addtolength{\topmargin}{-1.000in}
\addtolength{\textheight}{1.75in}
\addtolength{\parskip}{\baselineskip}
\setlength{\parindent}{0in}
\renewcommand{\baselinestretch}{1.5}
\newcommand{\term}{Winter 2020}

\singlespace


\begin{document}

\lecture{20 --- Compiler Optimizations}{\term}{Patrick Lam}

\section*{Compiler Optimizations: Interprocedural Analysis and Link-Time Optimizations}
\hfill ``Are economies of scale real?''

In this context, does a
whole-program optimization really improve your program?
We'll start by first talking about some information that is critical for
whole-program optimizations.

\subsection*{Alias and Pointer Analysis}
As we've seen in the above analyses, compiler optimizations often need
to know about what parts of memory each statement reads to.  This is
easy when talking about scalar variables which are stored on the
stack. This is much harder when talking about pointers or arrays
(which can alias). \emph{Alias analysis} helps by declaring that a
given variable {\tt p} does not alias another variable {\tt q}; that
is, they point to different heap locations. \emph{Pointer analysis}
abstractly tracks what regions of the heap each variable points to.
A region of the heap may be the memory allocated at a particular
program point.

When we know that two pointers don't alias, then we know that their
effects are independent, so it's correct to move things around.
This also helps in reasoning about side effects and enabling reordering.

We've talked about automatic parallelization previously in this course.
At this point, I'll remind you that we used {\tt restrict} so that the
compiler wouldn't have to do as much pointer analysis. Shape analysis
builds on pointer analysis to determine that data structures are indeed
trees rather than lists.

\paragraph{Call Graphs.} Many interprocedural analyses require accurate
call graphs. A call graph is a directed graph showing relationships between
functions. It's easy to compute a call graph when you have C-style
function calls. It's much harder when you have virtual methods, as in
C++ or Java, or even C function pointers. In particular, you need pointer
analysis information to construct the call graph.

\paragraph{Devirtualization.} This optimization attempts to convert
virtual function calls to direct calls.  Virtual method calls have the
potential to be slow, because there is effectively a branch to
predict. If the branch prediction goes well, then it doesn't impose
more runtime cost. However, the branch prediction might go poorly.  (In
general for C++, the program must read the object's vtable.) Plus, virtual
calls impede other optimizations. Compilers can help by doing
sophisticated analyses to compute the call graph and by replacing
virtual method calls with nonvirtual method calls.  Consider the
following code:
  \begin{lstlisting}[language=C]
class A {
    virtual void m();
};

class B : public A {
    virtual void m();
}

int main(int argc, char *argv[]) {

    std::unique_ptr<A> t(new B);
    t.m();
}
  \end{lstlisting}
Devirtualization could eliminate vtable access; instead, we could just call B's {\tt m} method
directly. By the way, ``Rapid Type Analysis'' analyzes the entire program, observes that
only {\tt B} objects are ever instantiated, and enables devirtualization
of the {\tt b.m()} call.

\noindent \emph{Enabled with {\tt -O2}, {\tt -O3}, or with {\tt -fdevirtualize}.}

\paragraph{Inlining.} We talked about inlining in Lecture 18. Compilers can inline following compiler directives, but usually more based on heuristics. Devirtualization enables more inlining. The compiler always inlines functions marked with the {\tt always\_inline} attribute, as seen in passing in Lecture 3.

\noindent \emph{Enabled with {\tt -O2} and {\tt -O3}.}

Obviously, inlining and devirtualization require call graphs. But so
does any analysis that needs to know about the heap effects of
functions that get called; for instance, consider this code:

{\small
\begin{lstlisting}[language=C]
  int n;

  int f() { /* opaque */ }

  int main() {
    n = 5;
    f();
    printf("%d\n", n);
  }
\end{lstlisting}
}
We could propagate the constant value 5, as long as we know that {\tt
  f()} does not write to {\tt n}.

\paragraph{Tail Recursion Elimination.} This optimization is mandatory
in some functional languages; we replace a call by a {\tt goto} at the
compiler level. Consider this example, courtesy of Wikipedia:

{\small
\begin{lstlisting}[language=C]
  int bar(int N) {
    if (A(N))
      return B(N);
    else
      return bar(N);
  }
\end{lstlisting}
}

For both calls, to {\tt B} and {\tt bar}, we don't need to return control
to the calling {\tt bar()} before returning to its caller (because {\tt bar()}
is done anyway). This avoids
function call overhead and reduces call stack use.

\noindent \emph{Enabled with {\tt -foptimize-sibling-calls}.} Also supports
sibling calls as well as tail-recursive calls.

\section*{Link-Time Optimizations}
Next up: mechanics of interprocedural optimizations in modern open-source
compilers. Conceptually, interprocedural optimizations have been well-understood
for a while. But practical implementations in open-source compilers are still
relatively new; Hubi\v{c}ka~\cite{hubicka14:_linkt_gcc} summarizes recent history.
In 2004, the only real interprocedural optimization in gcc was inlining, and it was
quite ad-hoc.

The biggest challenge for interprocedural optimizations is scalability, so 
it fits right in as a topic of discussion for this course.
Here's an outline of how it works:
\begin{itemize}[noitemsep]
\item local generation (parallelizable): compile to Intermediate Representation. Must generate compact
IR for whole-program analysis phase.
\item whole-program analysis (hard to parallelize!): create call graph, make transformation decisions. Possibly partition
the program.
\item local transformations (parallelizable): carry out transformations to local IRs, generate object code.
Perhaps use call graph partitions to decide optimizations. 
\end{itemize}
There were a number of conceptually-uninteresting implementation
challenges to be overcome before gcc could have its intermediate code available for
interprocedural analysis (i.e. there was no stable on-disk IR format). The transformations look like this:
\begin{itemize}[noitemsep]
\item global decisions, local transformations:
\begin{itemize}[noitemsep]
\item devirtualization
\item dead variable elimination/dead function elimination
\item field reordering, struct splitting/reorganization
\end{itemize}
\item global decisions, global transformations:
\begin{itemize}[noitemsep]
\item cross-module inlining
\item virtual function inlining
\item interprocedural constant propagation
\end{itemize}
\end{itemize}
The interesting issues arise from making the whole-program analysis scalable. Firefox, the Linux kernel,
and Chromium contain tens of millions of lines of code. Whole-program analysis requires that all of 
this code (in IR) be available to the analysis and that at least some summary of the code be in memory, 
along with the call graph.
(Since it's a whole-program analysis, any part of the program may affect other parts). The first problem
is getting it into memory; loading the IR for tens of millions of lines of code is a non-starter.
Clearly, anything that is more expensive than linear time can cause problems. Partitioning the program
can help.

How did gcc get better?
Hubi\v{c}ka~\cite{hubicka15:_link_gcc} explains how. In line with what I've said earlier, it's
avoiding unnecessary work.
\begin{itemize}[noitemsep]
\item gcc 4.5: initial version of LTO;
\item gcc 4.6: parallelization; partitioning of the call graph (put closely-related functions together, approximate functions in other partitions); the bottleneck: streaming in types and declarations;
\item gcc 4.7--4.9: improve build times, memory usage [``chasing unnecessary data away''.]
\end{itemize}
As far as I can tell, today's gcc, with {\tt -flto}, does work and includes
optimizations including constant propagation and function
specialization.

\paragraph{Impact.} gcc CTO appears to give 3--5\% improvements in performance, which compiler experts consider good.
Like we discussed last time, this allows developers to shift their attention from 
manual factoring of translation units to letting the compiler do it. (This is kind of like going
from manual transmissions to automatic transmissions for cars\ldots).

The LLVM project provides more details at~\cite{project17:_llvm_link_time_optim}, while gcc details
can be found at~\cite{novillo09:_linkt}.

%\url{https://gcc.gnu.org/wiki/LightweightIpo}

\bibliographystyle{alphaurl}
\bibliography{459}


\end{document}
