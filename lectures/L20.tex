\documentclass[letterpaper,10pt]{article}

\usepackage{enumitem}
\usepackage{titling}
\usepackage{listings,listings-rust}
\usepackage{url}
\usepackage{soul}
\usepackage{hyperref}
\usepackage{setspace}
\usepackage{subfig}
\usepackage{sectsty}
\usepackage{pdfpages}
\usepackage{colortbl}
\usepackage{multirow}
\usepackage{multicol}
\usepackage{relsize}
\usepackage{amsmath}
\usepackage{wasysym}
\usepackage{fancyvrb}
\usepackage[yyyymmdd]{datetime}
\usepackage{amsmath,amssymb,amsthm,graphicx,xspace}
\usepackage[titlenotnumbered,noend,noline]{algorithm2e}
\usepackage[compact]{titlesec}
\usepackage{XCharter}
\usepackage[T1]{fontenc}
\usepackage[scaled]{beramono}
\usepackage[normalem]{ulem}
\usepackage{booktabs}
\usepackage{tikz}
\usetikzlibrary{arrows.meta,automata,shapes,trees,matrix,chains,scopes,positioning,calc,decorations.pathreplacing}
\tikzstyle{block} = [rectangle, draw, fill=blue!20, 
    text width=2.5em, text centered, rounded corners, minimum height=2em]
\tikzstyle{bw} = [rectangle, draw, fill=blue!20, 
    text width=4em, text centered, rounded corners, minimum height=2em]

\definecolor{namerow}{cmyk}{.40,.40,.40,.40}
\definecolor{namecol}{cmyk}{.40,.40,.40,.40}
\renewcommand{\dateseparator}{-}

\let\LaTeXtitle\title
\renewcommand{\title}[1]{\LaTeXtitle{\textsf{#1}}}

\lstset{basicstyle=\footnotesize\ttfamily,breaklines=true}

\newcommand{\CPP}{C\nolinebreak\hspace{-.05em}\raisebox{.4ex}{\tiny\bf +}\nolinebreak\hspace{-.10em}\raisebox{.4ex}{\tiny\bf +}}
\def\CPP{{C\nolinebreak[4]\hspace{-.05em}\raisebox{.4ex}{\tiny\bf ++}}}

\newcommand{\handout}[5]{
  \noindent
  \begin{center}
  \framebox{
    \vbox{
      \hbox to 5.78in { {\bf ECE459: Programming for Performance } \hfill #2 }
      \vspace{4mm}
      \hbox to 5.78in { {\Large \hfill #4  \hfill} }
      \vspace{2mm}
      \hbox to 5.78in { {\em #3 \hfill \today} }
    }
  }
  \end{center}
  \vspace*{4mm}
}

\newcommand{\lecture}[3]{\handout{#1}{#2}{#3}{Lecture#1}}
\newcommand{\tuple}[1]{\ensuremath{\left\langle #1 \right\rangle}\xspace}

\addtolength{\oddsidemargin}{-1.000in}
\addtolength{\evensidemargin}{-0.500in}
\addtolength{\textwidth}{2.0in}
\addtolength{\topmargin}{-1.000in}
\addtolength{\textheight}{1.75in}
\addtolength{\parskip}{\baselineskip}
\setlength{\parindent}{0in}
\renewcommand{\baselinestretch}{1.5}
\newcommand{\term}{Winter 2020}

\singlespace


\begin{document}

\lecture{20 --- Reentrancy versus Thread-Safety}{\term}{Patrick Lam}

\section*{Reentrancy versus Thread-Safety}
On a different note, we're going to discuss the distinction between reentrant functions
and thread-safe functions. There is overlap, but these terms are actually different.

A function is \emph{reentrant} if it can be suspended in the middle
and re-entered, or called again, before the previous execution
returns.
    
Reentrant does not always mean {\bf thread-safe} (although it usually is).\
Recall: {\bf thread-safe} is essentially ``no data races''.

The distinction is moot if the function only modifies local data, e.g. {\tt sin()}.
Those functions are both reentrant and thread-safe.

\paragraph{Example.} Courtesy of Wikipedia (with modifications), here's a program (to the left) and its trace (to right):
\begin{center}
\begin{tabular}{ll}
\begin{minipage}{.5\textwidth}
  \begin{lstlisting}[basicstyle=\scriptsize]
int t;
 
void swap(int *x, int *y) {
    t = *x;
    *x = *y;
    // interrupt might invoke isr() here!
    *y = t;
}
 
void isr() {
    int x = 1, y = 2;
    swap(&x, &y);
}
...
int a = 3, b = 4;
...
    swap(&a, &b);
  \end{lstlisting}
\end{minipage}
&
\begin{minipage}{.4\textwidth}
  \begin{lstlisting}[basicstyle=\scriptsize]
call swap(&a, &b);
 t = *x;                // t = 3 (a)
 *x = *y;               // a = 4 (b)
 call isr();
    x = 1; y = 2;
    call swap(&x, &y)
    t = *x;             // t = 1 (x)
    *x = *y;            // x = 2 (y)
    *y = t;             // y = 1
 *y = t;                // b = 1

Final values:
a = 4, b = 1

Expected values:
a = 4, b = 3
  \end{lstlisting}
\end{minipage}
\end{tabular}
\end{center}


We can fix the example by storing the global variable in stack variable
{\tt s}, as follows.
\begin{center}
\begin{tabular}{ll}
\begin{minipage}{.5\textwidth}
  \begin{lstlisting}[escapechar=@,basicstyle=\scriptsize]
int t;
 
void swap(int *x, int *y) {
    int s;
 
    @{\bf s = t}@;  // save global variable
    t = *x;
    *x = *y;
    // interrupt might invoke isr() here!
    *y = t;
    @{\bf t = s}@;  // restore global variable
}
 
void isr() {
    int x = 1, y = 2;
    swap(&x, &y);
}
...
int a = 3, b = 4;
...
    swap(&a, &b);
  \end{lstlisting}
\end{minipage} &
\begin{minipage}{.5\textwidth}
  \begin{lstlisting}[basicstyle=\scriptsize]
call swap(&a, &b);
s = t;                 // s = UNDEFINED
t = *x;                // t = 3 (a)
*x = *y;               // a = 4 (b)
call isr();
    x = 1; y = 2;
    call swap(&x, &y)
    s = t;             // s = 3
    t = *x;            // t = 1 (x)
    *x = *y;           // x = 2 (y)
    *y = t;            // y = 1
    t = s;             // t = 3
*y = t;                // b = 3
t = s;                 // t = UNDEFINED

Final values:
a = 4, b = 3

Expected values:
a = 4, b = 3
  \end{lstlisting}
\end{minipage}
\end{tabular}
\end{center}

The obvious question: is the previous reentrant code also thread-safe?
(This is more what we're concerned about in this course.)

  Let's see. Consider two calls to the reentrant swap:\\ \verb+    swap(a, b)+, {\tt swap(c, d)} with {\tt a = 1, b = 2, c = 3, d = 4}.

  \begin{lstlisting}[basicstyle=\scriptsize]
    global: t

    /* thread 1 */                       /* thread 2 */
    a = 1, b = 2;
    s = t;    // s = UNDEFINED
    t = a;    // t = 1
                                         c = 3, d = 4;
                                         s = t;    // s = 1
                                         t = c;    // t = 3
                                         c = d;    // c = 4
                                         d = t;    // d = 3
    a = b;    // a = 2
    b = t;    // b = 3
    t = s;    // t = UNDEFINED
                                         t = s;    // t = 1

    Final values:
    a = 2, b = 3, c = 4, d = 3, t = 1

    Expected values:
    a = 2, b = 1, c = 4, d = 3, t = UNDEFINED
  \end{lstlisting}

To recap what we know so far: re-entrant does not always mean thread-safe. (But, for most sane implementations, reentrant is also thread-safe.)

But, are {\bf thread-safe} functions reentrant?
Nope! Consider:

  \begin{lstlisting}[basicstyle=\scriptsize]
int f() {
    lock();
    // protected code
    unlock();
}
  \end{lstlisting}

Recall:  Reentrant functions can be suspended in the middle of execution
    and called again before the previous execution completes.

{\tt f()} obviously isn't reentrant. Plus, it will deadlock\footnote{I'm talking about lock implementations which don't maintain a lock counter; you can request Java locks multiple times in the same thread and everything will be fine, but not pthreads locks.}.

Interrupt handling is more for systems programming, so the topic of reentrancy may or may not come
up again.

To sum up, here's the difference between reentrant and thread-safe functions:

  {\bf Reentrancy.}
  \begin{itemize}
    \item Has nothing to do with threads---assumes a {\bf single thread}.
    \item Reentrant means the execution can context switch at any point in
      in a function, call the {\bf same function}, and {\bf complete} before
      returning to the original function call.
    \item Function's result does not depend on where the context switch happens.
  \end{itemize}

  {\bf Thread-safety.}
  \begin{itemize}
    \item Result does not depend on any interleaving of threads from
      concurrency or parallelism.
    \item No unexpected results from multiple concurrent executions of the function.
  \end{itemize}

If it helps, here's another definition of thread-safety.
\begin{quote}
  ``A function whose effect, when called by two or more threads, is guaranteed to
  be as if the threads each executed the function one after another, in an
  undefined order, even if the actual execution is interleaved.''
\end{quote}

\paragraph{Good Example of an Exam Question.} Consider the following function.

  \begin{lstlisting}
void swap(int *x, int *y) {
    int t;
    t = *x;
    *x = *y;
    *y = t;
}  
  \end{lstlisting}

  \begin{itemize}
    \item Is the above code thread-safe?
    \item Write some expected results for running two calls in parallel.
    \item Argue these expected results always hold, or show an example where
      they do not.
  \end{itemize}

\section*{Good Programming Practices: Inlining}
We have seen the notion of inlining:
  \begin{itemize}
    \item Instructs the compiler to just insert the function code in-place,
      instead of calling the function.
    \item Hence, no function call overhead!
    \item Compilers can also do better---context-sensitive---operations they couldn't
      have done before.
  \end{itemize}

OK, so inlining removes overhead. Sounds like better performance! Let's inline everything!
There are two ways of inlining in C++.

\paragraph{Implicit inlining.} (defining a function inside a class definition):
  \begin{lstlisting}
class P {
public:
    int get_x() const { return x; }
...
private:
    int x;
};
  \end{lstlisting}

\paragraph{Explicit inlining.} Or, we can be explicit:
  \begin{lstlisting}
inline max(const int& x, const int& y) {
    return x < y ? y : x;
}
  \end{lstlisting}

\paragraph{The Other Side of Inlining.}
Inlining has one big downside:
  \begin{itemize}
    \item Your program size is going to increase.
  \end{itemize}
   This is worse than you think:
      \begin{itemize}
        \item Fewer cache hits.
        \item More trips to memory.
      \end{itemize}
   Some inlines can grow very rapidly (C++ extended constructors).
  Just from this your performance may go down easily.

  Note also that inlining is merely a suggestion to compilers.
  They may ignore you.
  For example:
  \begin{itemize}
    \item taking the address of an ``inline'' function and using it; or
    \item virtual functions (in C++),
  \end{itemize}
  will get you ignored quite fast.

\paragraph{Implications of inlining.} Inlining can make your life worse in two ways.
First, debugging is more difficult (e.g. you can't set a breakpoint in a function that
  doesn't actually exist).
 Most compilers simply won't inline code with debugging symbols on.
 Some do, but typically it's more of a pain.

Second, it can be a problem for library design:
  \begin{itemize}
    \item If you change any inline function in your library, any users
      of that library have to {\bf recompile} their program if the
      library updates. (Congratulations, you made a non-binary-compatible change!)
  \end{itemize}
This would not be a problem for non-inlined functions---programs execute the new function
dynamically at runtime.

\section*{High-Level Language Performance Tweaks}
So far, we've only seen C---we haven't seen anything complex, and C is
low level, which is good for learning what's really going on.

 Writing compact, readable code in C is hard, especially when \#define
macros and {\tt void *} beckon.

    C++11 has made major strides towards readability and
    efficiency---it provides light-weight abstractions. We'll look at
    a couple of examples.

\paragraph{Sorting.} Our goal is simple: we'd like to sort a bunch of integers.
In C, you would usually just use qsort from {\tt stdlib.h}.

  \begin{lstlisting}
void qsort (void* base, size_t num, size_t size,
            int (*comparator) (const void*, const void*));
  \end{lstlisting}

This is a fairly ugly definition (as usual, for generic C functions). How ugly is it?
Let's look at a usage example.
  \begin{lstlisting}[basicstyle=\small]
#include <stdlib.h>

int compare(const void* a, const void* b)
{
    return (*((int*)a) - *((int*)b));
}

int main(int argc, char* argv[])
{
    int array[] = {4, 3, 5, 2, 1};
    qsort(array, 5, sizeof(int), compare);
}
  \end{lstlisting}
This looks like a nightmare, and is more likely to have bugs than what we'll see next.


C++ has a sort with a much nicer interface\footnote{\ldots well, nicer to use, after you get over templates.}:

  \begin{lstlisting}[basicstyle=\small]
template <class RandomAccessIterator>
void sort (
    RandomAccessIterator first,
    RandomAccessIterator last
);

template <class RandomAccessIterator, class Compare>
void sort (
    RandomAccessIterator first,
    RandomAccessIterator last,
    Compare comp
);
  \end{lstlisting}
It is, in fact, easier to use:
  \begin{lstlisting}[basicstyle=\small]
#include <vector>
#include <algorithm>

int main(int argc, char* argv[])
{
    std::vector<int> v = {4, 3, 5, 2, 1};
    std::sort(v.begin(), v.end());
}
  \end{lstlisting}

{\bf Note:} Your compare function can be a function or a functor. (Don't know what functors
are? In C++, they're functions with state.) By default,
  {\tt sort} uses {\tt operator$<$} on the objects being sorted.

  \begin{itemize}
    \item Which is less error prone?
    \item Which is {\bf faster}?
  \end{itemize}

The second question is empirical. Let's see. We generate an array of 2 million ints
and sort it (10 times, taking the average).

\begin{itemize}
\item qsort: 0.49 seconds
\item C++ sort: 0.21 seconds
\end{itemize}

The C++ version is {\bf twice} as fast. Why?
      \begin{itemize}
        \item The C version just operates on memory---it has no clue about the
          data.
        \item We're throwing away useful information about what's being sorted.
        \item A C function-pointer call prevents inlining of the compare function.
      \end{itemize}
OK. What if we write our own sort in C, specialized for the data?

\begin{itemize}
\item Custom C sort: 0.29 seconds
\end{itemize}

Now the C++ version is still faster (but it's close). But, this is
quickly going to become a maintainability nightmare.
      \begin{itemize}
        \item Would you rather read a custom sort or 1 line?
        \item What (who) do you trust more?
      \end{itemize}

\subsection*{Lesson}
Abstractions will not make your program slower. 

\noindent
They allow speedups and are much easier to maintain and read.

\subsection*{Vectors vs Lists}
Consider two
problems.

\begin{enumerate}
\item Generate {\bf N} random integers and insert them into (sorted)
      sequence.
      
      {\bf Example:} 3 4 2 1
      
      \begin{itemize}
        \item 3
        \item 3 4
        \item 2 3 4
        \item 1 2 3 4
      \end{itemize}

\item Remove {\bf N} elements one-at-a-time by going to a random position
      and removing the element.

      {\bf Example:} 2 0 1 0
      
      \begin{itemize}
        \item 1 2 4
        \item 2 4
        \item 2
        \item 
      \end{itemize}
\end{enumerate}

For which {\bf N} is it better to use a list than a vector (or array)?

 
\paragraph{Complexity analysis.} As good computer scientists, let's analyze
the complexity.  

{\bf Vector}:\\[-2em]
      \begin{itemize}
        \item Inserting\\[-2em]
          \begin{itemize}
            \item $O(\log n)$ for binary search
            \item $O(n)$ for insertion (on average, move half the elements)
          \end{itemize}
        \item Removing\\[-2em]
          \begin{itemize}
            \item $O(1)$ for accessing
            \item $O(n)$ for deletion (on average, move half the elements)
          \end{itemize}
      \end{itemize}

{\bf List}:\\[-2em]
      \begin{itemize}
        \item Inserting\\[-2em]
          \begin{itemize}
            \item $O(n)$ for linear search
            \item $O(1)$ for insertion
          \end{itemize}
        \item Removing\\[-2em]
          \begin{itemize}
            \item $O(n)$ for accessing
            \item $O(1)$ for deletion
          \end{itemize}
      \end{itemize}

Therefore, based on their complexity, lists should be better.

\paragraph{Reality.} OK, here's what happens. 
\begin{verbatim}
$ ./vector_vs_list 50000
Test 1
======
vector: insert 0.1s   remove 0.1s   total 0.2s
list:   insert 19.44s   remove 5.93s   total 25.37s
Test 2
======
vector: insert 0.11s   remove 0.11s   total 0.22s
list:   insert 19.7s   remove 5.93s   total 25.63s
Test 3
======
vector: insert 0.11s   remove 0.1s   total 0.21s
list:   insert 19.59s   remove 5.9s   total 25.49s
\end{verbatim}

{\bf Vectors} dominate lists, performance wise. Why?
  \begin{itemize}
    \item Binary search vs. linear search complexity dominates.
    \item Lists use far more memory.
      {\bf On 64 bit machines:}
      \begin{itemize}
        \item Vector: 4 bytes per element.
        \item List: At least 20 bytes per element.
      \end{itemize}
    \item Memory access is slow, and results arrive in blocks:
      \begin{itemize}
        \item Lists' elements are all over memory, hence many
          cache misses.
        \item A cache miss for a vector will bring a lot more usable data.
      \end{itemize}
  \end{itemize}

So, here are some tips for getting better performance.
  \begin{itemize}
    \item Don't store unnecessary data in your program.
    \item Keep your data as compact as possible.
    \item Access memory in a predictable manner.
    \item Use vectors instead of lists by default.
    \item Programming abstractly can save a lot of time.
    \item Often, telling the compiler more gives you better code.
    \item Data structures can be critical, sometimes more than complexity.
    \item {\bf Low-level code != Efficient}.
    \item Think at a low level if you need to optimize anything.
    \item Readable code is good code---different hardware needs different
      optimizations.
  \end{itemize}


\bibliographystyle{alphaurl}
\bibliography{459}


\end{document}
