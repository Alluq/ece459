\documentclass[letterpaper,10pt]{article}

\usepackage{titling}
\usepackage{listings}
\usepackage{url}
\usepackage{setspace}
\usepackage{subfig}
\usepackage{sectsty}
\usepackage{pdfpages}
\usepackage{colortbl}
\usepackage{multirow}
\usepackage{multicol}
\usepackage{relsize}
\usepackage{amsmath}
\usepackage{fancyvrb}
\usepackage{amsmath,amssymb,amsthm,graphicx,xspace}
\usepackage[titlenotnumbered,noend,noline]{algorithm2e}
\usepackage[compact]{titlesec}
\usepackage{XCharter}
\usepackage[T1]{fontenc}
\usepackage{tikz}
\usetikzlibrary{arrows,automata,shapes,trees,matrix,chains,scopes,positioning,calc}
\tikzstyle{block} = [rectangle, draw, fill=blue!20, 
    text width=2.5em, text centered, rounded corners, minimum height=2em]
\tikzstyle{bw} = [rectangle, draw, fill=blue!20, 
    text width=4em, text centered, rounded corners, minimum height=2em]

\newcommand{\CPP}{C\nolinebreak\hspace{-.05em}\raisebox{.4ex}{\tiny\bf +}\nolinebreak\hspace{-.10em}\raisebox{.4ex}{\tiny\bf +}}
\def\CPP{{C\nolinebreak[4]\hspace{-.05em}\raisebox{.4ex}{\tiny\bf ++}}}

\let\LaTeXtitle\title
\renewcommand{\title}[1]{\LaTeXtitle{\textsf{#1}}}


\addtolength{\oddsidemargin}{-1.000in}
\addtolength{\evensidemargin}{-0.500in}
\addtolength{\textwidth}{2.0in}
\addtolength{\topmargin}{-1.000in}
\addtolength{\textheight}{1.75in}
\addtolength{\parskip}{\baselineskip}
\setlength{\parindent}{0in}
\renewcommand{\baselinestretch}{1.5}

\singlespace

\begin{document}

\lecture{33 --- More Advanced Queueing Theory }{\term}{Jeff Zarnett}

\section*{New Considerations}

There are a few new considerations, or, if you prefer, complications to queueing theory that we haven't yet covered in the fairly simple discussion of queueing theory we have had so far. But real life is considerably more complicated than a simple model. We'll talk about two settings, one that I love and one that I hate: food halls and Service Ontario. No points for guessing which is which.

If you're not familiar with them, a quick refresher. A food hall, or food court, is a place where there are a number of counter-service restaurants that frequently offer different cuisines. Service Ontario is a place of interaction with the administrative state, specifically the services administered by the provincial government of Ontario.

\paragraph{Multiple Services.} The way we have discussed the idea of services is that all the offered services are the same, or at the very least, every server can deliver every kind of service. That's sometimes reasonable -- at Service Ontario, there are many different services available (drivers license renewal, health card renewal, vehicle registration update, etc) and any one of the staff members there can help you with any one of those services.

There are, of course, other situations where an individual member of staff cannot provide all services and therefore you must queue for the specific thing that you want, even if the other queues are shorter or empty. A food hall works like this. The Mexican restaurant may be very popular because they have excellent tacos, resulting in a long queue for that particular place. There may be no queue at the Gelato stand, but that doesn't help; the Gelato place cannot provide tacos. And while both of these are food, you will probably agree that they're not (always) interchangeable. If you really want tacos, gelato won't do.

\paragraph{Interchangeability.} We just hit on the idea of interchangeability when talking about the food hall example. It's a spectrum: sometimes things are totally interchangeable, sometimes partly, and sometimes not at all. In the food hall example, total interchangeability is what happens if you're just extremely hungry and you would be completely happy whether you had tacos, pizza, or shawarma.  Partial interchangeability can happen when you have preferences but would accept something else: you want tacos but would choose shawarma under certain circumstances. And no interchangeability happens when you have your heart truly set on pizza and will accept nothing else.

Whether things are interchangeable depends both on the nature of the services and the needs of the people requesting them. Just as there are different services in this more complicated world, requesters can also behave differently.

All of the above food hall examples are about food, so there's at least a possibility of interchangeability because every option will have some nutritional value (in the sense of, it's edible food with calories in it; a nutritionist might say the gelato has no nutritional value because it's not a ``healthy'' choice). If you went to Service Ontario to renew your drivers' license, there are really no alternatives that get you the same result: a new health card just isn't the same thing.

Then there are the needs of the person requesting the thing:  If someone is a vegetarian, they might be okay with tacos or pizza, but not shawarma. If they are a vegan, maybe the only option for them is the taco stand, no matter how good the pizza place is or how long the line for tacos is.

\paragraph{Too Long.} 
While we're on the subject, when you're in the food hall and the length of a particular queue is too long, we may experience one of two behaviours: balking or reneging. 

Balking is what happens when you look at the line and you decide it's too long and there's no point in even getting into the line. That's very common: if I want to go to Service Ontario and it's so busy that there is a long line out the door, I'm not going to bother. I'll come back another time. In the food hall, if the queue for tacos is really long, I'll line up for the shawarma. 

Reneging is what happens if I enter the queue for tacos, but before I get to the front of the queue, I give up (leave the queue). If it's taking too long and I'm really hungry (or just impatient), this could happen. It could also happen if I'm at Service Ontario in between things (e.g., on a lunch break) and if time runs out I need to go back to work.

In both cases, there's an implicit or explicit estimation of the waiting time provided. When I choose not to enter the queue at all, it's because I've looked at the queue length, and maybe done some assessment of the service time, and calculated that the wait time. It's also possible that the establishment is kind enough to put out signs that say that the expected waiting time from this point is $X$ minutes. Not very common in a food hall or Service Ontario, but maybe the case in other places.

Obviously, my initial estimate can be wrong if I guessed incorrectly about the service time, or if the line is so long I can't see all of it and I don't get a good estimate. But another reason why my estimate might be wrong is if people can join the line ahead of me. Wait, what?

\paragraph{Priority.}
When I was last at Service Ontario, an elderly person with mobility restrictions showed up while I was waiting in line and that person was permitted to go to the front of the queue. It's sensible that priority would be given to this person -- asking them to stand in line a very long time is not nice. But of course, when they go into the line ahead of me, it increases my wait time. This increase may result in my decision to leave the line as the time has increased to the point that I can no longer, or at least no longer wish to, wait. 

Priorities for some groups over others are actually really interesting, because they have interesting effects and open up questions:

\begin{itemize}
	\item How much, if any, does giving priority to one group over another help the group being given priority?
	\item How much, if any, does giving priority to one group disadvantage the group not being given priority?
	\item Can the priority system incentivize people to choose things that are less popular?
	\item Recognizing that if everyone has priority, nobody has priority, how many requests can have priority before all benefit is lost?
\end{itemize}


\section*{Laboratory Study with a Mouse}

This section relies on a rather informative video by the channel Defunctland, about the history of the Disney Fastpass system~\cite{dldisney}. The actual video contains a lot of discussion about the history of Walt Disney World and other things that are not relevant here. But there are a few interesting things we can learn from it.

If we abstract away some of the details of the Mouse and his friends, 

\bibliographystyle{alphaurl}
\bibliography{459}


\end{document}
