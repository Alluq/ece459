
\documentclass[letterpaper,hide notes,xcolor={table,svgnames},pdftex,10pt]{beamer}
\def\showexamples{t}


%\usepackage[svgnames]{xcolor}

%% Demo talk
%\documentclass[letterpaper,notes=show]{beamer}

\usecolortheme{crane}
\setbeamertemplate{navigation symbols}{}

\usetheme{MyPittsburgh}
%\usetheme{Frankfurt}

%\usepackage{tipa}

\usepackage{hyperref}
\usepackage{graphicx,xspace}
\usepackage[normalem]{ulem}
\usepackage{multicol}
\usepackage{amsmath,amssymb,amsthm,graphicx,xspace}
\newcommand\SF[1]{$\bigstar$\footnote{SF: #1}}

\usepackage[default]{sourcesanspro}
\usepackage[T1]{fontenc}
\usepackage[scaled]{beramono}
\usepackage{tikzpagenodes}

\newcounter{tmpnumSlide}
\newcounter{tmpnumNote}


% old question code
%\newcommand\question[1]{{$\bigstar$ \small \onlySlide{2}{#1}}}
% \newcommand\nquestion[1]{\ifdefined \presentationonly \textcircled{?} \fi \note{\par{\Large \textbf{?}} #1}}
% \newcommand\nanswer[1]{\note{\par{\Large \textbf{A}} #1}}


 \newcommand\mnote[1]{%
   \addtocounter{tmpnumSlide}{1}
   \ifdefined\showcues {~\tiny\fbox{\arabic{tmpnumSlide}}}\fi
   \note{\setlength{\parskip}{1ex}\addtocounter{tmpnumNote}{1}\textbf{\Large \arabic{tmpnumNote}:} {#1\par}}}

\newcommand\mmnote[1]{\note{\setlength{\parskip}{1ex}#1\par}}

%\newcommand\mnote[2][]{\ifdefined\handoutwithnotes {~\tiny\fbox{#1}}\fi
% \note{\setlength{\parskip}{1ex}\textbf{\Large #1:} #2\par}}

%\newcommand\mnote[2][]{{\tiny\fbox{#1}} \note{\setlength{\parskip}{1ex}\textbf{\Large #1:} #2\par}}

\newcommand\mquestion[2]{{~\color{red}\fbox{?}}\note{\setlength{\parskip}{1ex}\par{\Large \textbf{?}} #1} \note{\setlength{\parskip}{1ex}\par{\Large \textbf{A}} #2\par}\ifdefined \presentationonly \pause \fi}

\newcommand\blackboard[1]{%
\ifdefined   \showblackboard
  {#1}
  \else {\begin{center} \fbox{\colorbox{blue!30}{%
         \begin{minipage}{.95\linewidth}%
           \hspace{\stretch{1}} Some space intentionally left blank; done at the blackboard.%
         \end{minipage}}}\end{center}}%
         \fi%
}



%\newcommand\q{\tikz \node[thick,color=black,shape=circle]{?};}
%\newcommand\q{\ifdefined \presentationonly \textcircled{?} \fi}

\usepackage{listings}
\lstset{basicstyle=\footnotesize\ttfamily,
	breaklines=true,
	aboveskip=15pt,
  	belowskip=15pt,
	frame=lines,
	numbers=left, basicstyle=\scriptsize, numberstyle=\tiny, stepnumber=0, numbersep=2pt
}

\usepackage{siunitx}
\newcommand\sius[1]{\num[group-separator = {,}]{#1}\si{\micro\second}}
\newcommand\sims[1]{\num[group-separator = {,}]{#1}\si{\milli\second}}
\newcommand\sins[1]{\num[group-separator = {,}]{#1}\si{\nano\second}}
\sisetup{group-separator = {,}, group-digits = true}

%% -------------------- tikz --------------------
\usepackage{tikz}
\usetikzlibrary{positioning}
\usetikzlibrary{arrows,backgrounds,automata,decorations.shapes,decorations.pathmorphing,decorations.markings,decorations.text,decorations.pathreplacing}

\tikzstyle{place}=[circle,draw=blue!50,fill=blue!20,thick, inner sep=0pt,minimum size=6mm]
\tikzstyle{transition}=[rectangle,draw=black!50,fill=black!20,thick, inner sep=0pt,minimum size=4mm]

\tikzstyle{block}=[rectangle,draw=black, thick, inner sep=5pt]
\tikzstyle{bullet}=[circle,draw=black, fill=black, thin, inner sep=2pt]

\tikzstyle{pre}=[<-,shorten <=1pt,>=stealth',semithick]
\tikzstyle{post}=[->,shorten >=1pt,>=stealth',semithick]
\tikzstyle{bi}=[<->,shorten >=1pt,shorten <=1pt, >=stealth',semithick]

\tikzstyle{mut}=[-,>=stealth',semithick]

\tikzstyle{treereset}=[dashed,->, shorten >=1pt,>=stealth',thin]

\usepackage{ifmtarg}
\usepackage{xifthen}
\makeatletter
% new counter to now which frame it is within the sequence
\newcounter{multiframecounter}
% initialize buffer for previously used frame title
\gdef\lastframetitle{\textit{undefined}}
% new environment for a multi-frame
\newenvironment{multiframe}[1][]{%
\ifthenelse{\isempty{#1}}{%
% if no frame title was set via optional parameter,
% only increase sequence counter by 1
\addtocounter{multiframecounter}{1}%
}{%
% new frame title has been provided, thus
% reset sequence counter to 1 and buffer frame title for later use
\setcounter{multiframecounter}{1}%
\gdef\lastframetitle{#1}%
}%
% start conventional frame environment and
% automatically set frame title followed by sequence counter
\begin{frame}%
\frametitle{\lastframetitle~{\normalfont(\arabic{multiframecounter})}}%
}{%
\end{frame}%
}
\makeatother

\makeatletter
\newdimen\tu@tmpa%
\newdimen\ydiffl%
\newdimen\xdiffl%
\newcommand\ydiff[2]{%
    \coordinate (tmpnamea) at (#1);%
    \coordinate (tmpnameb) at (#2);%
    \pgfextracty{\tu@tmpa}{\pgfpointanchor{tmpnamea}{center}}%
    \pgfextracty{\ydiffl}{\pgfpointanchor{tmpnameb}{center}}%
    \advance\ydiffl by -\tu@tmpa%
}
\newcommand\xdiff[2]{%
    \coordinate (tmpnamea) at (#1);%
    \coordinate (tmpnameb) at (#2);%
    \pgfextractx{\tu@tmpa}{\pgfpointanchor{tmpnamea}{center}}%
    \pgfextractx{\xdiffl}{\pgfpointanchor{tmpnameb}{center}}%
    \advance\xdiffl by -\tu@tmpa%
}
\makeatother
\newcommand{\copyrightbox}[3][r]{%
\begin{tikzpicture}%
\node[inner sep=0pt,minimum size=2em](ciimage){#2};
\usefont{OT1}{phv}{n}{n}\fontsize{4}{4}\selectfont
\ydiff{ciimage.south}{ciimage.north}
\xdiff{ciimage.west}{ciimage.east}
\ifthenelse{\equal{#1}{r}}{%
\node[inner sep=0pt,right=1ex of ciimage.south east,anchor=north west,rotate=90]%
{\raggedleft\color{black!50}\parbox{\the\ydiffl}{\raggedright{}#3}};%
}{%
\ifthenelse{\equal{#1}{l}}{%
\node[inner sep=0pt,right=1ex of ciimage.south west,anchor=south west,rotate=90]%
{\raggedleft\color{black!50}\parbox{\the\ydiffl}{\raggedright{}#3}};%
}{%
\node[inner sep=0pt,below=1ex of ciimage.south west,anchor=north west]%
{\raggedleft\color{black!50}\parbox{\the\xdiffl}{\raggedright{}#3}};%
}
}
\end{tikzpicture}
}


%% --------------------

%\usepackage[excludeor]{everyhook}
%\PushPreHook{par}{\setbox0=\lastbox\llap{MUH}}\box0}

%\vspace*{\stretch{1}

%\setbox0=\lastbox \llap{\textbullet\enskip}\box0}

\setlength{\parskip}{\fill}

\newcommand\noskips{\setlength{\parskip}{1ex}}
\newcommand\doskips{\setlength{\parskip}{\fill}}

\newcommand\xx{\par\vspace*{\stretch{1}}\par}
\newcommand\xxs{\par\vspace*{2ex}\par}
\newcommand\tuple[1]{\langle #1 \rangle}
\newcommand\code[1]{{\sf \footnotesize #1}}
\newcommand\ex[1]{\uline{Example:} \ifdefined \presentationonly \pause \fi
  \ifdefined\showexamples#1\xspace\else{\uline{\hspace*{2cm}}}\fi}

\newcommand\ceil[1]{\lceil #1 \rceil}


\AtBeginSection[]
{
   \begin{frame}
       \frametitle{Outline}
       \tableofcontents[currentsection]
   \end{frame}
}



\pgfdeclarelayer{edgelayer}
\pgfdeclarelayer{nodelayer}
\pgfsetlayers{edgelayer,nodelayer,main}

\tikzstyle{none}=[inner sep=0pt]
\tikzstyle{rn}=[circle,fill=Red,draw=Black,line width=0.8 pt]
\tikzstyle{gn}=[circle,fill=Lime,draw=Black,line width=0.8 pt]
\tikzstyle{yn}=[circle,fill=Yellow,draw=Black,line width=0.8 pt]
\tikzstyle{empty}=[circle,fill=White,draw=Black]
\tikzstyle{bw} = [rectangle, draw, fill=blue!20, 
    text width=4em, text centered, rounded corners, minimum height=2em]
    
    \newcommand{\CcNote}[1]{% longname
	This work is licensed under the \textit{Creative Commons #1 3.0 License}.%
}
\newcommand{\CcImageBy}[1]{%
	\includegraphics[scale=#1]{creative_commons/cc_by_30.pdf}%
}
\newcommand{\CcImageSa}[1]{%
	\includegraphics[scale=#1]{creative_commons/cc_sa_30.pdf}%
}
\newcommand{\CcImageNc}[1]{%
	\includegraphics[scale=#1]{creative_commons/cc_nc_30.pdf}%
}
\newcommand{\CcGroupBySa}[2]{% zoom, gap
	\CcImageBy{#1}\hspace*{#2}\CcImageNc{#1}\hspace*{#2}\CcImageSa{#1}%
}
\newcommand{\CcLongnameByNcSa}{Attribution-NonCommercial-ShareAlike}

\newenvironment{changemargin}[1]{% 
  \begin{list}{}{% 
    \setlength{\topsep}{0pt}% 
    \setlength{\leftmargin}{#1}% 
    \setlength{\rightmargin}{1em}
    \setlength{\listparindent}{\parindent}% 
    \setlength{\itemindent}{\parindent}% 
    \setlength{\parsep}{\parskip}% 
  }% 
  \item[]}{\end{list}} 




\title{Lecture 15 --- OpenMP Memory Model }

\author{Patrick Lam \\ \small \texttt{p.lam@ece.uwaterloo.ca}}
\institute{Department of Electrical and Computer Engineering \\
  University of Waterloo}
\date{\today}


\begin{document}

\begin{frame}
  \titlepage

 \end{frame}

\begin{frame}
  \frametitle{Memory Model}

  

  OpenMP uses a {\bf relaxed-consistency, shared-memory} model:

  \begin{itemize}
    \item All threads share a single store called
      \structure{memory}.\\ ~~(may not actually represent RAM)\\[1em]
    \item Each thread can have its own {\it temporary} view of memory.\\[1em]
    \item A thread's {\it temporary} view of memory is not required to be
      consistent with memory.
  \end{itemize}~\\

  We'll talk more about memory models later.
  
\end{frame}
%%%%%%%%%%%%%%%%%%%%%%%%%%%%%%%%%%%%%%%%%%%%%%%%%%%%%%%%%%%%%%%%%%%%%%%%%%%%%%%%

%%%%%%%%%%%%%%%%%%%%%%%%%%%%%%%%%%%%%%%%%%%%%%%%%%%%%%%%%%%%%%%%%%%%%%%%%%%%%%%%
\begin{frame}[fragile]
  \frametitle{Preventing Simultaneous Execution?}

  \begin{lstlisting}
                    a = b = 0
/* thread 1 */                      /* thread 2 */

atomic(b = 1) // [1]                atomic(a = 1) // [3]
atomic(tmp = a) // [2]              atomic(tmp = b) // [4]
if (tmp == 0) then                  if (tmp == 0) then
    // protected section                // protected section
end if                              end if
  \end{lstlisting}

  
 Does this code actually prevent simultaneous execution?
  
\end{frame}
%%%%%%%%%%%%%%%%%%%%%%%%%%%%%%%%%%%%%%%%%%%%%%%%%%%%%%%%%%%%%%%%%%%%%%%%%%%%%%%%

%%%%%%%%%%%%%%%%%%%%%%%%%%%%%%%%%%%%%%%%%%%%%%%%%%%%%%%%%%%%%%%%%%%%%%%%%%%%%%%%
\begin{frame}[fragile]
  \frametitle{Possible States for Example}

  \begin{lstlisting}
                    a = b = 0
/* thread 1 */                      /* thread 2 */

atomic(b = 1) // [1]                atomic(a = 1) // [3]
atomic(tmp = a) // [2]              atomic(tmp = b) // [4]
if (tmp == 0) then                  if (tmp == 0) then
    // protected section                // protected section
end if                              end if
  \end{lstlisting}

  \begin{center}
  \begin{tabular}{r r r r | r r}
    \multicolumn{4}{c|}{Order} & t1 tmp & t2 tmp\\
    \hline
    1 & 2 & 3 & 4 & 0 & 1\\
    1 & 3 & 2 & 4 & 1 & 1\\
    1 & 3 & 4 & 2 & 1 & 1\\
    3 & 4 & 1 & 2 & 1 & 0\\
    3 & 1 & 2 & 4 & 1 & 1\\
    3 & 1 & 4 & 2 & 1 & 1\\
  \end{tabular}
  \end{center}

  
    Looks like it (at least intuitively).
  
\end{frame}
%%%%%%%%%%%%%%%%%%%%%%%%%%%%%%%%%%%%%%%%%%%%%%%%%%%%%%%%%%%%%%%%%%%%%%%%%%%%%%%%

%%%%%%%%%%%%%%%%%%%%%%%%%%%%%%%%%%%%%%%%%%%%%%%%%%%%%%%%%%%%%%%%%%%%%%%%%%%%%%%%
\begin{frame}[fragile]
  \frametitle{Memory Model Gotcha}

  \begin{lstlisting}
                    a = b = 0
/* thread 1 */                      /* thread 2 */

atomic(b = 1) // [1]                atomic(a = 1) // [3]
atomic(tmp = a) // [2]              atomic(tmp = b) // [4]
if (tmp == 0) then                  if (tmp == 0) then
    // protected section                // protected section
end if                              end if
  \end{lstlisting}

  
    Sorry! With OpenMP's memory model, no guarantees:\\
    the update from one thread may not be seen by the other.
  
\end{frame}
%%%%%%%%%%%%%%%%%%%%%%%%%%%%%%%%%%%%%%%%%%%%%%%%%%%%%%%%%%%%%%%%%%%%%%%%%%%%%%%%

%%%%%%%%%%%%%%%%%%%%%%%%%%%%%%%%%%%%%%%%%%%%%%%%%%%%%%%%%%%%%%%%%%%%%%%%%%%%%%%%
\begin{frame}[fragile]
  \frametitle{Flush: Ensuring Consistent Views of Memory}

  \begin{center}
    {\tt \#pragma omp }{\bf flush} {\it[(list)]}
  \end{center}

  
    Makes the thread's temporary view of memory consistent with main
      memory; hence:\\[1em]
    $\bullet$ enforces an order on the memory operations of the variables.\\[1em]
    The variables in the list are called the {\it flush-set}. 
    If no variables given, the compiler will determine them for you.
  
\end{frame}
%%%%%%%%%%%%%%%%%%%%%%%%%%%%%%%%%%%%%%%%%%%%%%%%%%%%%%%%%%%%%%%%%%%%%%%%%%%%%%%%

%%%%%%%%%%%%%%%%%%%%%%%%%%%%%%%%%%%%%%%%%%%%%%%%%%%%%%%%%%%%%%%%%%%%%%%%%%%%%%%%
\begin{frame}
  \frametitle{Explaining Flush}

  
  Enforcing an order on the memory operations means:

  \begin{itemize}
    \item All read/write operations on the {\it flush-set} which happen
      before the {\bf flush} complete before the flush executes.
    \item All read/write operations on the {\it flush-set} which happen
      after the {\bf flush} complete after the flush executes.
    \item Flushes with overlapping {\it flush-sets} can not be reordered.
  \end{itemize}
  
\end{frame}
%%%%%%%%%%%%%%%%%%%%%%%%%%%%%%%%%%%%%%%%%%%%%%%%%%%%%%%%%%%%%%%%%%%%%%%%%%%%%%%%

%%%%%%%%%%%%%%%%%%%%%%%%%%%%%%%%%%%%%%%%%%%%%%%%%%%%%%%%%%%%%%%%%%%%%%%%%%%%%%%%
\begin{frame}
  \frametitle{Flush Correctness}

  
  To show a consistent value for a variable between two threads, 
  OpenMP must run statements in this order:

  \begin{enumerate}
    \item $t_1$ writes the value to $v$;
    \item $t_1$ flushes $v$; 
    \item $t_2$ flushes $v$ also;
    \item $t_2$ reads the consistent value from $v$.
  \end{enumerate}
  
\end{frame}
%%%%%%%%%%%%%%%%%%%%%%%%%%%%%%%%%%%%%%%%%%%%%%%%%%%%%%%%%%%%%%%%%%%%%%%%%%%%%%%%

%%%%%%%%%%%%%%%%%%%%%%%%%%%%%%%%%%%%%%%%%%%%%%%%%%%%%%%%%%%%%%%%%%%%%%%%%%%%%%%%
\begin{frame}[fragile]
  \frametitle{Take 2: Same Example, now improved with Flush}

  \begin{lstlisting}
                    a = b = 0
/* thread 1 */                      /* thread 2 */

atomic(b = 1)                       atomic(a = 1)
flush(b)                            flush(a)
flush(a)                            flush(b)
atomic(tmp = a)                     atomic(tmp = b)
if (tmp == 0) then                  if (tmp == 0) then
    // protected section                // protected section
end if                              end if
  \end{lstlisting}

  \begin{itemize}
    \item OK. Will this now prevent simultaneous access?
  \end{itemize}
\end{frame}
%%%%%%%%%%%%%%%%%%%%%%%%%%%%%%%%%%%%%%%%%%%%%%%%%%%%%%%%%%%%%%%%%%%%%%%%%%%%%%%%

%%%%%%%%%%%%%%%%%%%%%%%%%%%%%%%%%%%%%%%%%%%%%%%%%%%%%%%%%%%%%%%%%%%%%%%%%%%%%%%%
\begin{frame}[fragile]
  \frametitle{No Luck Yet: Why Flush Fails}

  

  \begin{center}
    \alert{\LARGE No.}
  \end{center}

  \begin{itemize}
    \item The compiler can reorder the {\tt flush(b)} in thread 1 or
      {\tt flush(a)} in thread 2.

    \item If {\tt flush(b)} gets reordered to after the protected
      section, we will not get our intended operation.
  \end{itemize}
  
\end{frame}
%%%%%%%%%%%%%%%%%%%%%%%%%%%%%%%%%%%%%%%%%%%%%%%%%%%%%%%%%%%%%%%%%%%%%%%%%%%%%%%%

%%%%%%%%%%%%%%%%%%%%%%%%%%%%%%%%%%%%%%%%%%%%%%%%%%%%%%%%%%%%%%%%%%%%%%%%%%%%%%%%
\begin{frame}[fragile]
  \frametitle{Should you use flush?}

  
  \Large
  Probably not, but now you know what it does.
  

\end{frame}
%%%%%%%%%%%%%%%%%%%%%%%%%%%%%%%%%%%%%%%%%%%%%%%%%%%%%%%%%%%%%%%%%%%%%%%%%%%%%%%%


%%%%%%%%%%%%%%%%%%%%%%%%%%%%%%%%%%%%%%%%%%%%%%%%%%%%%%%%%%%%%%%%%%%%%%%%%%%%%%%%
\begin{frame}[fragile]
  \frametitle{Same Example, Finally Correct}

  \begin{lstlisting}
                    a = b = 0
/* thread 1 */                      /* thread 2 */

atomic(b = 1)                       atomic(a = 1)
flush(a, b)                         flush(a, b)
atomic(tmp = a)                     atomic(tmp = b)
if (tmp == 0) then                  if (tmp == 0) then
    // protected section                // protected section
end if                              end if
  \end{lstlisting}
\end{frame}
%%%%%%%%%%%%%%%%%%%%%%%%%%%%%%%%%%%%%%%%%%%%%%%%%%%%%%%%%%%%%%%%%%%%%%%%%%%%%%%%

%%%%%%%%%%%%%%%%%%%%%%%%%%%%%%%%%%%%%%%%%%%%%%%%%%%%%%%%%%%%%%%%%%%%%%%%%%%%%%%%
\begin{frame}
  \frametitle{OpenMP Directives Where Flush Isn't Implied}

  
  \begin{itemize}
    \item at entry to {\bf for};
    \item at entry to, or exit from, {\bf master};
    \item at entry to {\bf sections}; 
    \item at entry to {\bf single};
    \item at exit from {\bf for}, {\bf single} or {\bf sections} with a {\bf nowait}
      \begin{itemize}
        \item {\bf nowait} removes implicit flush along with the implicit barrier
      \end{itemize}
  \end{itemize}

  This is not true for OpenMP versions before 2.5, so be careful.
  
\end{frame}
%%%%%%%%%%%%%%%%%%%%%%%%%%%%%%%%%%%%%%%%%%%%%%%%%%%%%%%%%%%%%%%%%%%%%%%%%%%%%%%%

%%%%%%%%%%%%%%%%%%%%%%%%%%%%%%%%%%%%%%%%%%%%%%%%%%%%%%%%%%%%%%%%%%%%%%%%%%%%%%%%
\begin{frame}
  \frametitle{OpenMP Task Directive}

  
  \begin{center}
    {\tt \#pragma omp }{\bf task} {\it [clause [[,] clause]*]}
  \end{center}~\\

     Generates a task for a thread in the team to run.\\[1em]
     When a thread enters the region it may:
      \begin{itemize}
        \item immediately execute the task; or
        \item defer its execution.\\\qquad  (any other thread may be assigned the task)
      \end{itemize}

  Allowed Clauses: {\bf if, final, untied, default, mergeable, private,
  firstprivate, shared}
  
\end{frame}
%%%%%%%%%%%%%%%%%%%%%%%%%%%%%%%%%%%%%%%%%%%%%%%%%%%%%%%%%%%%%%%%%%%%%%%%%%%%%%%%

%%%%%%%%%%%%%%%%%%%%%%%%%%%%%%%%%%%%%%%%%%%%%%%%%%%%%%%%%%%%%%%%%%%%%%%%%%%%%%%%
\begin{frame}
  \frametitle{Tasks: {\tt if} and {\tt final} Clauses}

  
  \begin{center}
  {\bf if} {\it(scalar-logical-expression)}
  \end{center}

    When expression is {\tt false}, generates an undeferred task---\\
    the generating task region is suspended until execution of the
      undeferred task finishes.\\[1em]

  \begin{center}
  {\bf final} {\it(scalar-logical-expression)}
  \end{center}

    When expression is {\tt true}, generates a final task.\\
    All tasks within a final task are {\it included}.\\
    Included tasks are undeferred and also execute immediately in the same thread.
  
\end{frame}
%%%%%%%%%%%%%%%%%%%%%%%%%%%%%%%%%%%%%%%%%%%%%%%%%%%%%%%%%%%%%%%%%%%%%%%%%%%%%%%%

%%%%%%%%%%%%%%%%%%%%%%%%%%%%%%%%%%%%%%%%%%%%%%%%%%%%%%%%%%%%%%%%%%%%%%%%%%%%%%%%
\begin{frame}[fragile]
  \frametitle{{\tt if} and {\tt final} Clauses: Examples}

  \begin{lstlisting}
void foo () {
    int i;
    #pragma omp task if(0) // This task is undeferred
    {
        #pragma omp task
        // This task is a regular task
        for (i = 0; i < 3; i++) {
            #pragma omp task
            // This task is a regular task
            bar();
        }
    }
    #pragma omp task final(1) // This task is a regular task
    {
        #pragma omp task // This task is included
        for (i = 0; i < 3; i++) {
            #pragma omp task
            // This task is also included
            bar();
        }
    }
}
  \end{lstlisting}
\end{frame}
%%%%%%%%%%%%%%%%%%%%%%%%%%%%%%%%%%%%%%%%%%%%%%%%%%%%%%%%%%%%%%%%%%%%%%%%%%%%%%%%

%%%%%%%%%%%%%%%%%%%%%%%%%%%%%%%%%%%%%%%%%%%%%%%%%%%%%%%%%%%%%%%%%%%%%%%%%%%%%%%%
\begin{frame}
  \frametitle{{\tt untied} and {\tt mergeable} Clauses}

  
\begin{center}
  {\bf untied}
\end{center}
  \begin{itemize}
    \item A suspended task can be resumed by any thread.
    \item ``untied'' is ignored if used with {\bf final}.
    \item Interacts poorly with thread-private variables and {\tt gettid()}.
  \end{itemize}

\begin{center}
  {\bf mergeable}
\end{center}

  \begin{itemize}
    \item For an undeferred or included task,
    allows the implementation to generate a merged task instead.
    \item In a merged task, the implementation may re-use the environment from its generating task (as if there was no task directive).
  \end{itemize}

  For more: \url{docs.oracle.com/cd/E24457_01/html/E21996/gljyr.html}
  
\end{frame}
%%%%%%%%%%%%%%%%%%%%%%%%%%%%%%%%%%%%%%%%%%%%%%%%%%%%%%%%%%%%%%%%%%%%%%%%%%%%%%%%

%%%%%%%%%%%%%%%%%%%%%%%%%%%%%%%%%%%%%%%%%%%%%%%%%%%%%%%%%%%%%%%%%%%%%%%%%%%%%%%%
\begin{frame}[fragile]
  \frametitle{Bad {\tt mergeable} Example}

  
  \begin{lstlisting}
#include <stdio.h>
void foo () {
    int x = 2;
    #pragma omp task mergeable
    {
        x++; // x is by default firstprivate
    }
    #pragma omp taskwait
    printf("%d\n",x); // prints 2 or 3
}
  \end{lstlisting}
  
    This is an incorrect usage of {\bf mergeable}: the output depends
      on whether or not the task got merged.\\[1em]
    Merging tasks (when safe) produces more efficient code.

  
\end{frame}
%%%%%%%%%%%%%%%%%%%%%%%%%%%%%%%%%%%%%%%%%%%%%%%%%%%%%%%%%%%%%%%%%%%%%%%%%%%%%%%%

%%%%%%%%%%%%%%%%%%%%%%%%%%%%%%%%%%%%%%%%%%%%%%%%%%%%%%%%%%%%%%%%%%%%%%%%%%%%%%%%
\begin{frame}[fragile]
  \frametitle{Taskyield}

  
  \begin{center}
    {\tt \#pragma omp }{\bf taskyield}
  \end{center}

    Specifies that the current task can be suspended in favour of another task.\\[1em]

  Here's a good use of {\bf taskyield}.

  \begin{lstlisting}
void foo (omp_lock_t * lock, int n) {
    int i;
    for ( i = 0; i < n; i++ )
    #pragma omp task
    {
        something_useful();
        while (!omp_test_lock(lock)) {
            #pragma omp taskyield
        }
        something_critical();
        omp_unset_lock(lock);
    }
}
  \end{lstlisting}
  
\end{frame}
%%%%%%%%%%%%%%%%%%%%%%%%%%%%%%%%%%%%%%%%%%%%%%%%%%%%%%%%%%%%%%%%%%%%%%%%%%%%%%%%

%%%%%%%%%%%%%%%%%%%%%%%%%%%%%%%%%%%%%%%%%%%%%%%%%%%%%%%%%%%%%%%%%%%%%%%%%%%%%%%%
\begin{frame}[fragile]
  \frametitle{Taskwait}

  

  \begin{center}
    {\tt \#pragma omp }{\bf taskwait}
  \end{center}~\\[1em]

     Waits for the completion of the current task's child tasks.
  

\end{frame}
%%%%%%%%%%%%%%%%%%%%%%%%%%%%%%%%%%%%%%%%%%%%%%%%%%%%%%%%%%%%%%%%%%%%%%%%%%%%%%%%

%%%%%%%%%%%%%%%%%%%%%%%%%%%%%%%%%%%%%%%%%%%%%%%%%%%%%%%%%%%%%%%%%%%%%%%%%%%%%%%%
\begin{frame}[fragile]
  \frametitle{OpenMP Example: Tree Traversal}

  
  \begin{lstlisting}
struct node {
    struct node *left;
    struct node *right;
};
extern void process(struct node *);

void traverse(struct node *p) {
    if (p->left) {
        #pragma omp task
        // p is firstprivate by default
        traverse(p->left);
    }
    if (p->right) {
        #pragma omp task
        // p is firstprivate by default
        traverse(p->right);
    }
    process(p);
}    
  \end{lstlisting}
  
\end{frame}
%%%%%%%%%%%%%%%%%%%%%%%%%%%%%%%%%%%%%%%%%%%%%%%%%%%%%%%%%%%%%%%%%%%%%%%%%%%%%%%%

%%%%%%%%%%%%%%%%%%%%%%%%%%%%%%%%%%%%%%%%%%%%%%%%%%%%%%%%%%%%%%%%%%%%%%%%%%%%%%%%
\begin{frame}[fragile]
  \frametitle{OpenMP Example 2: Post-order Tree Traversal}

  
  \begin{lstlisting}
struct node {
    struct node *left;
    struct node *right;
};
extern void process(struct node *);

void traverse(struct node *p) {
    if (p->left) {
        #pragma omp task
        // p is firstprivate by default
        traverse(p->left);
    }
    if (p->right) {
        #pragma omp task
        // p is firstprivate by default
        traverse(p->right);
    }
    #pragma omp taskwait
    process(p);
}    
  \end{lstlisting}
  
  Note: Used an explicit {\bf taskwait} before processing.
  
\end{frame}
%%%%%%%%%%%%%%%%%%%%%%%%%%%%%%%%%%%%%%%%%%%%%%%%%%%%%%%%%%%%%%%%%%%%%%%%%%%%%%%%


%%%%%%%%%%%%%%%%%%%%%%%%%%%%%%%%%%%%%%%%%%%%%%%%%%%%%%%%%%%%%%%%%%%%%%%%%%%%%%%%
\begin{frame}[fragile]
  \frametitle{OpenMP Example: Parallel Linked List Processing}

  
  \begin{lstlisting}
// node struct with data and pointer to next
extern void process(node* p);

void increment_list_items(node* head) {
    #pragma omp parallel
    {
        #pragma omp single
        {
            node * p = head;
            while (p) {
                #pragma omp task
                {
                    process(p);
                }
                p = p->next;
            }
        }
    }
}
  \end{lstlisting}
  
\end{frame}
%%%%%%%%%%%%%%%%%%%%%%%%%%%%%%%%%%%%%%%%%%%%%%%%%%%%%%%%%%%%%%%%%%%%%%%%%%%%%%%%


%%%%%%%%%%%%%%%%%%%%%%%%%%%%%%%%%%%%%%%%%%%%%%%%%%%%%%%%%%%%%%%%%%%%%%%%%%%%%%%%
\begin{frame}[fragile]
  \frametitle{OpenMP Example: Lots of Tasks}

  
  \begin{lstlisting}
#define LARGE_NUMBER 10000000
double item[LARGE_NUMBER];
extern void process(double);

int main() {
    #pragma omp parallel
    {
        #pragma omp single
        {
            int i;
            for (i=0; i<LARGE_NUMBER; i++) {
                #pragma omp task
                // i is firstprivate, item is shared
                process(item[i]);
            }
        }
    }
}
  \end{lstlisting}


  The main loop generates tasks, which are all assigned to the executing thread as it becomes available.\\[1em]
  When too many tasks generated: suspends main thread, runs some tasks, then resumes the loop in main thread.
  
\end{frame}
%%%%%%%%%%%%%%%%%%%%%%%%%%%%%%%%%%%%%%%%%%%%%%%%%%%%%%%%%%%%%%%%%%%%%%%%%%%%%%%%

%%%%%%%%%%%%%%%%%%%%%%%%%%%%%%%%%%%%%%%%%%%%%%%%%%%%%%%%%%%%%%%%%%%%%%%%%%%%%%%%
\begin{frame}[fragile]
  \frametitle{OpenMP Example: Improved version of Lots of Tasks}

  
  \begin{lstlisting}
#define LARGE_NUMBER 10000000
double item[LARGE_NUMBER];
extern void process(double);

int main() {
    #pragma omp parallel
    {
        #pragma omp single
        {
            int i;
            #pragma omp task untied
            {
                for (i=0; i<LARGE_NUMBER; i++) {
                    #pragma omp task
                    process(item[i]);
                }
            }
        }
    }
}
  \end{lstlisting}
  
  \begin{itemize}
    \item {\bf untied} lets another thread take on tasks.
  \end{itemize}
  
\end{frame}
%%%%%%%%%%%%%%%%%%%%%%%%%%%%%%%%%%%%%%%%%%%%%%%%%%%%%%%%%%%%%%%%%%%%%%%%%%%%%%%%

%%%%%%%%%%%%%%%%%%%%%%%%%%%%%%%%%%%%%%%%%%%%%%%%%%%%%%%%%%%%%%%%%%%%%%%%%%%%%%%%
\begin{frame}[fragile]
  \frametitle{About Nesting: Restrictions}

  
  \begin{itemize}
    \item You cannot nest {\bf for} regions.
    \item You cannot nest {\bf single} inside a {\bf for}.
    \item You cannot nest {\bf barrier} inside a\\~~~~~ {\bf critical/single/master/for}.
  \end{itemize}

  \begin{lstlisting}
void good_nesting(int n)
{
    int i, j;
    #pragma omp parallel default(shared)
    {
        #pragma omp for
        for (i=0; i<n; i++) {
            #pragma omp parallel shared(i, n)
            {
                #pragma omp for
                for (j=0; j < n; j++)
                    work(i, j);
            }
        }
    }
}
  \end{lstlisting}
  
\end{frame}
%%%%%%%%%%%%%%%%%%%%%%%%%%%%%%%%%%%%%%%%%%%%%%%%%%%%%%%%%%%%%%%%%%%%%%%%%%%%%%%%

%%%%%%%%%%%%%%%%%%%%%%%%%%%%%%%%%%%%%%%%%%%%%%%%%%%%%%%%%%%%%%%%%%%%%%%%%%%%%%%%
\begin{frame}
  \frametitle{Why Your Code is Slow}

  
  Want it to run faster? Avoid these pitfalls:
  \begin{enumerate}
    \item Unnecessary flush directives.
    \item Using critical sections or locks instead of atomic.
    \item Unnecessary concurrent-memory-writing protection:
      \begin{itemize}
        \item No need to protect local thread variables.
        \item No need to protect if only accessed in {\bf single} or
          {\bf master}.
      \end{itemize}
    \item Too much work in a critical section.
    \item Too many entries into critical sections.
  \end{enumerate}
  
\end{frame}
%%%%%%%%%%%%%%%%%%%%%%%%%%%%%%%%%%%%%%%%%%%%%%%%%%%%%%%%%%%%%%%%%%%%%%%%%%%%%%%%

%%%%%%%%%%%%%%%%%%%%%%%%%%%%%%%%%%%%%%%%%%%%%%%%%%%%%%%%%%%%%%%%%%%%%%%%%%%%%%%%
\begin{frame}[fragile]
  \frametitle{Example: Too Many Entries into Critical Sections}

  
  \begin{lstlisting}
#pragma omp parallel for
for (i = 0; i < N; ++i) { 
    #pragma omp critical
    {
        if (arr[i] > max) max = arr[i];
    } 
}
  \end{lstlisting}

would be better as:

  \begin{lstlisting}
#pragma omp parallel for
for (i = 0 ; i < N; ++i) { 
    #pragma omp flush(max)
    if (arr[i] > max) {
          #pragma omp critical
          {
                if (arr[i] > max) max = arr[i];
          }
    }
}
  \end{lstlisting}
  
\end{frame}
%%%%%%%%%%%%%%%%%%%%%%%%%%%%%%%%%%%%%%%%%%%%%%%%%%%%%%%%%%%%%%%%%%%%%%%%%%%%%%%%

%%%%%%%%%%%%%%%%%%%%%%%%%%%%%%%%%%%%%%%%%%%%%%%%%%%%%%%%%%%%%%%%%%%%%%%%%%%%%%%%
\begin{frame}
  \frametitle{Summary}

  
    Finished exploring OpenMP. Key points:
  \begin{itemize}
    \item How to use {\bf flush} correctly.
    \item How to use OpenMP {\bf tasks} to parallelize unstructured problems.
  \end{itemize}
  
\end{frame}
%%%%%%%%%%%%%%%%%%%%%%%%%%%%%%%%%%%%%%%%%%%%%%%%%%%%%%%%%%%%%%%%%%%%%%%%%%%%%%%%


\end{document}

