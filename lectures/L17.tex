\documentclass[letterpaper,10pt]{article}

\usepackage{enumitem}
\usepackage{titling}
\usepackage{listings,listings-rust}
\usepackage{url}
\usepackage{soul}
\usepackage{hyperref}
\usepackage{setspace}
\usepackage{subfig}
\usepackage{sectsty}
\usepackage{pdfpages}
\usepackage{colortbl}
\usepackage{multirow}
\usepackage{multicol}
\usepackage{relsize}
\usepackage{amsmath}
\usepackage{wasysym}
\usepackage{fancyvrb}
\usepackage[yyyymmdd]{datetime}
\usepackage{amsmath,amssymb,amsthm,graphicx,xspace}
\usepackage[titlenotnumbered,noend,noline]{algorithm2e}
\usepackage[compact]{titlesec}
\usepackage{XCharter}
\usepackage[T1]{fontenc}
\usepackage[scaled]{beramono}
\usepackage[normalem]{ulem}
\usepackage{booktabs}
\usepackage{tikz}
\usetikzlibrary{arrows.meta,automata,shapes,trees,matrix,chains,scopes,positioning,calc,decorations.pathreplacing}
\tikzstyle{block} = [rectangle, draw, fill=blue!20, 
    text width=2.5em, text centered, rounded corners, minimum height=2em]
\tikzstyle{bw} = [rectangle, draw, fill=blue!20, 
    text width=4em, text centered, rounded corners, minimum height=2em]

\definecolor{namerow}{cmyk}{.40,.40,.40,.40}
\definecolor{namecol}{cmyk}{.40,.40,.40,.40}
\renewcommand{\dateseparator}{-}

\let\LaTeXtitle\title
\renewcommand{\title}[1]{\LaTeXtitle{\textsf{#1}}}

\lstset{basicstyle=\footnotesize\ttfamily,breaklines=true}

\newcommand{\CPP}{C\nolinebreak\hspace{-.05em}\raisebox{.4ex}{\tiny\bf +}\nolinebreak\hspace{-.10em}\raisebox{.4ex}{\tiny\bf +}}
\def\CPP{{C\nolinebreak[4]\hspace{-.05em}\raisebox{.4ex}{\tiny\bf ++}}}

\newcommand{\handout}[5]{
  \noindent
  \begin{center}
  \framebox{
    \vbox{
      \hbox to 5.78in { {\bf ECE459: Programming for Performance } \hfill #2 }
      \vspace{4mm}
      \hbox to 5.78in { {\Large \hfill #4  \hfill} }
      \vspace{2mm}
      \hbox to 5.78in { {\em #3 \hfill \today} }
    }
  }
  \end{center}
  \vspace*{4mm}
}

\newcommand{\lecture}[3]{\handout{#1}{#2}{#3}{Lecture#1}}
\newcommand{\tuple}[1]{\ensuremath{\left\langle #1 \right\rangle}\xspace}

\addtolength{\oddsidemargin}{-1.000in}
\addtolength{\evensidemargin}{-0.500in}
\addtolength{\textwidth}{2.0in}
\addtolength{\topmargin}{-1.000in}
\addtolength{\textheight}{1.75in}
\addtolength{\parskip}{\baselineskip}
\setlength{\parindent}{0in}
\renewcommand{\baselinestretch}{1.5}
\newcommand{\term}{Winter 2020}

\singlespace


\begin{document}

\lecture{17 --- Mostly Data Parallelism}{\term}{Patrick Lam}


\section*{Data and Task Parallelism}
There are two broad categories of paralellism: data parallelism and
task parallelism. An analogy to data parallelism is hiring a call
center to (incompetently) handle large volumes of support calls,
\emph{all in the same way}. Assembly lines are an analogy to task
parallelism: each worker does a \emph{different} thing.

More precisely, in data parallelism, multiple threads perform the
\emph{same} operation on separate data items. For instance, you have a
big array and want to double all of the elements. Assign part of the
array to each thread. Each thread does the same thing: double array
elements.

In task parallelism, multiple threads perform \emph{different}
operations on separate data items. So you might have a thread that
renders frames and a thread that compresses frames and combines them
into a single movie file.

%We'll continue by looking at a number of parallelization patterns,
%examples of how to apply them, and situations where they might apply.

\subsection*{You're not using those bytes, are you?}
So as a first idea we might think of saving some space by considering the range of, for example, an integer. An \texttt{i32} is 4 bytes. (In C, \texttt{int} is usually 4, though only guaranteed to be at least 2). If we have an integer array of capacity $N$ that uses $N \times 4$ bytes and if we want to do something like increment each element, we iterate over the array and increment it, which is a read of 4 and write of 4. Now, if we could live with limiting our maximum value from 2,147,483,647 (signed, or 4,294,967,295 unsigned) to 32,767  (signed, or 65,535 unsigned), we could reduce in half the amount of space needed for this array and make operations like incrementing take half as much time!

Aside from the obvious tradeoff of limiting the maximum value, the other hidden cost is that of course things that were simple like \texttt{array[i] += 1} is more complicated. What do we do now?

Instead of \texttt{+=1} we need to calculate the new number to add. The interesting part is about how to represent the upper portion of the number. For just adding 1 it might be simple, and we can manually break out our calculators or draw a bit vector or think in hexadecimal about how to convert a number if it's more difficult. But you wouldn't---you would probably just use bit shift to calculate it. But one must be careful with that as well: the bit shift does sign extension which sometimes you don't want (or does unexpected things), and if we have to bit shift on every iteration of the loop, it's not clear that this is better than two assignment statements\ldots

Maybe you think this example is silly because of Rust's \texttt{i8}/C's \texttt{short} types. Which you could certainly use to reduce the size of the array. But then modifying each \texttt{short} in a different instruction defeats the purpose.

Aha! We can also take it a step farther: if it's a 64-bit processor there's no reason why you couldn't modify 8 bytes in a single instruction. The principle is the same, even if the math is a little more complex.

What we've got here is a poor person's version of Single Instruction Multiple Data (SIMD) (in NZ-speak, using No. 8 wire to implement SIMD), because we have to do our own math in advance and/or do a lot of bit shifting every time we want to use a value\ldots This is a pain. Fortunately, we don't have to\ldots

\subsection*{Data Parallelism with SIMD}
The ``typical'' boring standard uniprocessor is Single Instruction Single Data (SISD) but since the mid-1980s we've had more options than that. We'll talk about single-instruction multiple-data (SIMD) later on in
this course, but here's a quick look. Each SIMD
instruction operates on an entire vector of data. These instructions
originated with supercomputers in the 70s. More recently, GPUs; the
x86 SSE instructions; the SPARC VIS instructions; and the
Power/PowerPC AltiVec instructions all implement SIMD.

% #updateAfterCOVID
SIMD provides an advantage by using a single control unit to command multiple processing units and therefore the amount of overhead in the instruction stream. This is something that we do quite frequently in the everyday: if I asked someone to erase the board\footnote{Classrooms. How 2019.}, it's more efficient if I say ``erase these segments of the board'' (and clearly indicate which segments) than if I say ``erase this one'' and when that's done, then say ``erase that one''\ldots and so on. So we can probably get some performance benefit out of this!

There is the downside, though, that because there's only the one control unit, all the processing units are told to do the same thing. That might not be what you want, so SIMD is not something we can use in every situation. There are also diminishing returns: the more processing units you have, the less likely it is that you can use all of that power effectively (because it will be less likely to have enough identical operations)~\cite{sse}.

\paragraph{Compilation.} Let's look at an example of SIMD instructions when they are compiled.

By default your compiler will assume a particular target architecture; which one exactly is dependent on what the Rust team decided some time in the past. Choosing a too-new architecture will cause your code to fail on older machines. The choice of architecture can be overridden in your compile-time options with the \texttt{target} parameter. Let's look at some SSE code to add two slides and put the result in a third slice:

\begin{lstlisting}[language=Rust]
pub fn foo(a: &[f64], b: &[f64], c: &mut [f64]) {
    for ((a, b), c) in a.iter().zip(b).zip(c) {
        *c = *a + *b;
    }
}\end{lstlisting}

%% Compiling this without SIMD on a 32-bit x86 ({\tt --target=i586-unknown-linux-gnu}, decidedly not the default) might give this core loop contents:
%% \begin{verbatim}
%%   fld     qword ptr [ecx]
%%   fld     qword ptr [edx]
%%   faddp   st(1), st
%%   fstp    qword ptr [eax]
%% \end{verbatim}

We can compile with \texttt{rustc} defaults
and get something like this as core loop contents:
\begin{verbatim}
  movsd   xmm0, qword ptr [rcx]
  addsd   xmm0, qword ptr [rdx]
  movsd   qword ptr [rax], xmm0
\end{verbatim}
This uses the SSE\footnote{You can also compile without SIMD using \texttt{--target=i586-unknown-linux-gnu} and see the stack-based x87 instructions.} register \texttt{xmm0} and SSE2 instructions \texttt{movsd} and
\texttt{addsd}; the \texttt{sd} suffix denotes scalar double instructions, applying
only to the first 64 bits of the 128-bit \texttt{xmm0} register---this is a literal
translation of the code. If you additionally specify \texttt{-O}, the compiler generates a number of variants, including this middle one:
\begin{verbatim}
  movupd  xmm0, xmmword ptr [rdi + 8*rcx]
  movupd  xmm1, xmmword ptr [rdi + 8*rcx + 16]
  movupd  xmm2, xmmword ptr [rdx + 8*rcx]
  addpd   xmm2, xmm0
  movupd  xmm0, xmmword ptr [rdx + 8*rcx + 16]
  addpd   xmm0, xmm1
  movupd  xmmword ptr [r8 + 8*rcx], xmm2
  movupd  xmmword ptr [r8 + 8*rcx + 16], xmm0
\end{verbatim}
The \emph{packed} operations ({\tt p}) operate on multiple data
elements at a time (what kind of parallelism is this?)  The
implication is that the loop only needs to loop half as many times.
The compiler includes more variants, not shown, to handle cases where
there are odd numbers of elements in the slices.

So this is a piece of good news, for once: there's automatic use of the SSE instructions if your compiler knows the target machine architecture supports them. However, we can also explicitly invoke these instructions, or use libraries\footnote{A discussion of libraries available as of May 2020: \url{https://www.mdeditor.tw/pl/pdnr}; your choices are \texttt{packed\_simd} (nightly Rust only), \texttt{faster} (unmaintained), or \texttt{simdeez} (must use unsafe Rust).}, although we won't do that much. Instead, we'll learn more about how they work and then do some measures as to whether they really do. 

SIMD is different from the other types of parallelization we're
looking at, since there aren't multiple threads working at once.
It is complementary to using threads, and good for cases
where loops operate over vectors of data. These loops could also be
parallelized; multicore chips can do both, achieving high throughput.
SIMD instructions also work well on small data sets, where thread startup
cost is too high, while registers are just there.

In~\cite{lemire18:_multic_simd}, Daniel Lemire argues that vector
instructions are, in general, a more efficient way to parallelize code
than threads. That is, when applicable, they use less overall CPU
resources (cores and power) and run faster.

Data alignment, however, can be an issue with SIMD. According to~\cite{sse}: ``Data must be 16-byte aligned when loading to and storing from the 128-bit XMM registers used by SSE/SSE2/SSE3/SSSE3. This must be done to avoid severe performance penalties.''. That's a pretty harsh restriction. SSE4.2 lifts it, if your machine is new enough.

But in any case, Rust will generally align primitives to their sizes. Under the default representation, Rust promises nothing else about alignment. You can use the \texttt{repr(packed(N)} or \texttt{repr(align(N)} directives to express constraints on alignment, and you can specify the C representation, which allows you more control over data layout.

%% This is not all quite true for Rust.
%% If you've heard me talk about the dangers of unaligned data before, I usually give you something that looks like this as an example.
%% \begin{lstlisting}[language=Rust]
%% #[repr(C)]
%% struct Position {
%%   marked: bool,
%%   w: i32,
%%   x: i32,
%%   y: i32,
%%   z: i32
%% }
%% \end{lstlisting}
%% First of all, if you use the default representation (i.e. omit \texttt{repr}), then Rust makes no guarantees about the representation except for meeting any \texttt{align} or \texttt{packed} modifiers.

%% Now, if you use the \texttt{C} representation, Rust publishes an algorithm for determining the layout, again respecting modifiers. This algorithm potentially results in the integer values being unaligned if the boolean type is of size 1. You might be in trouble even if you deleted that boolean type out of the structure because your \texttt{w} parameter might not be 16-byte aligned even if it's 4-byte aligned.
%% %% Rust aligns primitives to their sizes, except for \texttt{u64} and \texttt{f64} on x86, which are only aligned to 4 bytes. However, you can also specify a \texttt{repr} of \texttt{C, packed(N)} or \texttt{C, align(N)} to align to N bytes.

%% Alignment headaches can be reduced by using the SSE specific types (and there are some) as the compiler will automatically see to it that the memory is allocated in accordance with this requirement. But then we're sort of back to packing our larger types ourselves, or at least writing them as a smaller type array and telling the compiler to pretend it's now a 128-bit type.

\paragraph{Worked Example.} So let's say that you actually wanted to try it out. Let's consider a \texttt{simdeez} example, which I've put in the repo's live coding subdir under \texttt{lectures/live-coding/L18}.

\begin{lstlisting}[language=Rust]
use simdeez::*;
use simdeez::scalar::*;
use simdeez::sse2::*;
use simdeez::sse41::*;
use simdeez::avx2::*;

simd_runtime_generate!(
// assumes that the input sizes are evenly divisible by VF32_WIDTH
pub fn add(a:&[f32], b: &[f32]) -> Vec<f32> {
  let len = a.len();
  let mut result: Vec<f32> = Vec::with_capacity(len);
  result.set_len(len);
  for i in (0..len).step_by(S::VF32_WIDTH) {
    let a0 = S::loadu_ps(&a[i]);
    let b0 = S::loadu_ps(&b[i]);
    S::storeu_ps(&mut result[0], S::add_ps(a0, b0));
  }
  result
});

fn main() {
  let a : [f32; 4] = [1.0, 2.0, 3.0, 4.0];
  let b : [f32; 4] = [5.0, 6.0, 7.0, 8.0];

  unsafe {
    println!("{:?}", add_sse2(&a, &b))
  }
}
\end{lstlisting}
What this does is generate an \texttt{add\_*} function for each of \texttt{scalar},
\texttt{sse2}, \texttt{sse41}, and \texttt{avx}. Then \texttt{main}
unsafely calls \texttt{add\_sse2} with two length-4 arrays of \texttt{f32}s and
gets a \texttt{Vec<f32>} back.

\texttt{simdeez} is a fairly lightweight wrapper around SIMD instructions and just
calls the \texttt{loadu\_ps} and \texttt{storeu\_ps} calls to load and store
packed single-precision numbers, and \texttt{add\_ps} to add them. Operator overloading
works too.

\section*{Case Study on SIMD: Stream VByte }

\hfill ``Can you run faster just by trying harder?''

The performance improvements we've seen to date have been leveraging parallelism
to improve throughput. Decreasing latency is trickier---it often requires domain-specific
tweaks.

Sometimes it's classic computer science: Quantum Flow found a place
where they could cache the last element of a list to reduce time
complexity for insertion from $O(n^2)$ to $O(n \log n)$.

\begin{center}
\url{https://bugzilla.mozilla.org/show_bug.cgi?id=1350770}
\end{center}

We'll also look at a more involved example of decreasing latency today, Stream VByte~\cite{LEMIRE20181}, and briefly at parts of its \CPP~implementation.
Even this example leverages parallelism---it uses vector instructions. But there
are some sequential improvements, e.g. Stream VByte takes care to be predictable
for the branch predictor.

\paragraph{Context.} We can abstract the problem to that of storing a sequence of small integers.
Such sequences are important, for instance, in the context of inverted indexes, which allow
fast lookups by term, and support boolean queries which combine terms.

Here is a list of documents and some terms that they contain:
\begin{center}
\begin{tabular}{r|l}
docid & terms \\ \hline
1 & dog, cat, cow\\
2 & cat\\
3 & dog, goat\\
4 & cow, cat, goat\\
\end{tabular}
\end{center}

The inverted index looks like this:
\begin{center}
\begin{tabular}{r|l}
term & docs \\ \hline
dog & 1, 3 \\
cat & 1, 2, 4 \\
cow & 1, 4 \\
goat & 3, 4
\end{tabular}
\end{center}

Inverted indexes contain many small integers in their lists: it is
sufficient to store the delta between a doc id and its successor, and
the deltas are typically small if the list of doc ids is sorted.
(Going from deltas to original integers takes time logarithmic
in the number of integers).

VByte is one of a number of schemes that use a variable number of
bytes to store integers.  This makes sense when most integers are
small, and especially on today's 64-bit processors.

VByte works like this:
\vspace*{-1em}
\begin{itemize}[noitemsep]
\item $x$ between 0 and $2^7-1$, e.g. $17 = 0b10001$: $0xxx xxxx$, e.g. $0001 0001$;
\item $x$ between $2^7$ and $2^{14}-1$, e.g. $1729 = 0b110 11000001$:
                   $1xxx xxxx/0xxx xxxx$, e.g. $1100 0001/0000 1101$;
\item $x$ between $2^{14}$ and $2^{21}-1$: $0xxx xxxx/1xxx xxxx/1xxx xxxx$;
\item etc.
\end{itemize}
That is, the control bit, or high-order bit, is 0 if you have finished representing the integer,
and 1 if more bits remain. (UTF-8 encodes the length, from 1 to 4, in high-order bits of the first byte.)

It might seem that dealing with variable-byte integers might be
harder than dealing fixed-byte integers, and it is. But there are performance benefits: because we are
using fewer bits, we can fit more information into our limited RAM and
cache, and even get higher throughput. Storing and reading 0s isn't an effective
use of resources. However, a naive algorithm to decode VByte also gives
lots of branch mispredictions.

Stream VByte is a variant of VByte which works using SIMD instructions.
Science is incremental, and Stream VByte builds on earlier work---masked VByte
as well as {\sc varint}-GB and {\sc varint}-G8IU. The innovation in
Stream VByte is to store the control and data streams separately.

Stream VByte's control stream uses two bits per integer to represent the size of the integer:
\begin{center}
\vspace*{-1em}
\begin{tabular}{ll@{~~~~~~~~}ll}
00 & 1 byte & 10 & 3 bytes\\
01 & 2 bytes & 11 & 4 bytes
\end{tabular}
\end{center}

Each decode iteration reads a byte from the control stream and 16 bytes of data from memory.
It uses a lookup table over the possible values of the control stream to decide how many
bytes it needs out of the 16 bytes it has read, and then uses SIMD instructions to shuffle
the bits each into their own integers. Note that, unlike VByte, Stream VByte uses all 8 bits
of each data byte as data.

For instance, if the control stream contains $0b1000~1100$, then the data stream
contains the following sequence of integer sizes: $3, 1, 4, 1$. Out of the 16 bytes read,
this iteration will use 9 bytes; it advances the data pointer by 9. It then uses the SIMD
``shuffle'' instruction to put the decoded integers from the data stream at known positions in the
128-bit SIMD register; in this case, it pads the first 3-byte integer with 1 byte, then
the next 1-byte integer with 3 bytes, etc. Let's say that the input is
{\tt 0xf823~e127~2524~9748~1b..~....~....~....}. The 128-bit output is
{\tt 0x00f8~23e1/0000~0027/2524 9748/0000~001b}, with the /s denoting separation
between outputs. The shuffle mask is precomputed and, at
execution time, read from an array.

The core of the (\CPP) implementation uses three SIMD instructions (also available in \texttt{simdeez}):
\begin{lstlisting}[language=C]
  uint8_t C = lengthTable[control];
  __m128i Data = _mm_loadu_si128 ((__m128i *) databytes);
  __m128i Shuf = _mm_loadu_si128(shuffleTable[control]);
  Data = _mm_shuffle_epi8(Data, Shuf);
  databytes += C; control++;
\end{lstlisting}

\paragraph{Discussion.} The paper~\cite{LEMIRE20181} includes a number of benchmark results
showing how Stream VByte performs better than previous techniques on a realistic input.
Let's discuss how it achieves this performance.

\begin{itemize}[noitemsep]
\item control bytes are sequential: the processor can always prefetch the next control byte, because
its location is predictable;
\item data bytes are sequential and loaded at high throughput;
\item shuffling exploits the instruction set so that it takes 1 cycle;
\item control-flow is regular (executing only the tight loop which retrieves/decodes control
and data; there are no conditional jumps).
\end{itemize}
We're exploiting SIMD, so this isn't quite strictly single-threaded performance.
Considering branch prediction and caching issues, though,
certainly improves single-threaded performance.

\section*{SIMD and Planetary Motion}

At the moment, I'm not planning to cover this, but you can read more about SIMD
in Rust here:
\begin{center}
\url{https://medium.com/@Razican/learning-simd-with-rust-by-finding-planets-b85ccfb724c3}
\end{center}

\bibliographystyle{alphaurl}
\bibliography{459}


\end{document}
