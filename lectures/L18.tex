\documentclass[letterpaper,10pt]{article}

\usepackage{enumitem}
\usepackage{titling}
\usepackage{listings,listings-rust}
\usepackage{url}
\usepackage{soul}
\usepackage{hyperref}
\usepackage{setspace}
\usepackage{subfig}
\usepackage{sectsty}
\usepackage{pdfpages}
\usepackage{colortbl}
\usepackage{multirow}
\usepackage{multicol}
\usepackage{relsize}
\usepackage{amsmath}
\usepackage{wasysym}
\usepackage{fancyvrb}
\usepackage[yyyymmdd]{datetime}
\usepackage{amsmath,amssymb,amsthm,graphicx,xspace}
\usepackage[titlenotnumbered,noend,noline]{algorithm2e}
\usepackage[compact]{titlesec}
\usepackage{XCharter}
\usepackage[T1]{fontenc}
\usepackage[scaled]{beramono}
\usepackage[normalem]{ulem}
\usepackage{booktabs}
\usepackage{tikz}
\usetikzlibrary{arrows.meta,automata,shapes,trees,matrix,chains,scopes,positioning,calc,decorations.pathreplacing}
\tikzstyle{block} = [rectangle, draw, fill=blue!20, 
    text width=2.5em, text centered, rounded corners, minimum height=2em]
\tikzstyle{bw} = [rectangle, draw, fill=blue!20, 
    text width=4em, text centered, rounded corners, minimum height=2em]

\definecolor{namerow}{cmyk}{.40,.40,.40,.40}
\definecolor{namecol}{cmyk}{.40,.40,.40,.40}
\renewcommand{\dateseparator}{-}

\let\LaTeXtitle\title
\renewcommand{\title}[1]{\LaTeXtitle{\textsf{#1}}}

\lstset{basicstyle=\footnotesize\ttfamily,breaklines=true}

\newcommand{\CPP}{C\nolinebreak\hspace{-.05em}\raisebox{.4ex}{\tiny\bf +}\nolinebreak\hspace{-.10em}\raisebox{.4ex}{\tiny\bf +}}
\def\CPP{{C\nolinebreak[4]\hspace{-.05em}\raisebox{.4ex}{\tiny\bf ++}}}

\newcommand{\handout}[5]{
  \noindent
  \begin{center}
  \framebox{
    \vbox{
      \hbox to 5.78in { {\bf ECE459: Programming for Performance } \hfill #2 }
      \vspace{4mm}
      \hbox to 5.78in { {\Large \hfill #4  \hfill} }
      \vspace{2mm}
      \hbox to 5.78in { {\em #3 \hfill \today} }
    }
  }
  \end{center}
  \vspace*{4mm}
}

\newcommand{\lecture}[3]{\handout{#1}{#2}{#3}{Lecture#1}}
\newcommand{\tuple}[1]{\ensuremath{\left\langle #1 \right\rangle}\xspace}

\addtolength{\oddsidemargin}{-1.000in}
\addtolength{\evensidemargin}{-0.500in}
\addtolength{\textwidth}{2.0in}
\addtolength{\topmargin}{-1.000in}
\addtolength{\textheight}{1.75in}
\addtolength{\parskip}{\baselineskip}
\setlength{\parindent}{0in}
\renewcommand{\baselinestretch}{1.5}
\newcommand{\term}{Winter 2020}

\singlespace


\begin{document}

\lecture{18 --- Reentrancy, Inlining, High-Level Languages}{\term}{Patrick Lam}

\section*{Reentrancy}

Recall from a bit earlier the idea of a side effect of a function call. 

Code that allows multiple concurrent invocations without affecting the outcome is called reentrant or ``pure'' (and the use of the word pure shouldn't imply any sort of moral judgement on the code). It is a desirable property to have code that is reentrant if we want to parallelize things. If a function is not reentrant, it may not be possible to make it thread safe. And furthermore, a reentrant function cannot call a non-reentrant one (and maintain its status as reentrant).

Side effects are sort of undesirable, but not necessarily bad. Printing to console is unavoidably making use of a side effect, but it's what we want. Nevertheless, it should be obvious that when printing we can't have reentrant behaviour because two threads trying to write at the same time to the console would result in jumbled output. Or alternatively, restarting the print routine might result in some doubled characters on the screen.

The trivial example of a non-reentrant C function:
\begin{lstlisting}[language=C]
int tmp;

void swap( int x, int y ) {
    tmp = y;
    y = x;
    x = tmp;
}
\end{lstlisting}

Why is this non-reentrant? Because there is a global variable \texttt{tmp} and it is changed on every invocation of the function. We can make the code reentrant by moving the declaration of \texttt{tmp} inside the function, which would mean that every invocation is independent of every other. And thus it would be thread safe, too.

Remember that in things like interrupt subroutines (ISRs) having the code be reentrant is very important. Interrupts can get interrupted by higher priority interrupts and when that happens the ISR may simply be restarted (or we're going to break off handling what we're doing and call the same ISR in the middle of the current one). Either way, if the code is not reentrant we will run into problems.

Let us also draw a distinction between thread safe code and reentrant code. A thread safe operation is one that can be performed from more than one thread at the same time. On the other hand, a reentrant operation can be invoked while the operation is already in progress, possibly from within the same thread. Or it can be re-started without affecting the outcome. See this code example from~\cite{tont:threadsafe}:
\begin{lstlisting}[language=C]
int length = 0;
char *s = NULL;

// Note: Since strings end with a 0, if we want to
// add a 0, we encode it as "\0", and encode a
// backslash as "\\".


// WARNING! This code is buggy - do not use!


void AddToString(int ch)
{
  EnterCriticalSection(&someCriticalSection);
  // +1 for the character we're about to add
  // +1 for the null terminator
  char *newString = realloc(s, (length+1) * sizeof(char));
  if (newString) {
    if (ch == '\0' || ch == '\\') {
      AddToString('\\'); // escape prefix
    }
    newString[length++] = ch;
    newString[length] = '\0';
    s = newString;
  }
  LeaveCriticalSection(&someCriticalSection);
}
\end{lstlisting}

Is it thread safe? Sure - there is a critical section protected by the mutex \texttt{someCriticalSection}. But is is re-entrant? Nope. The internal call to \texttt{AddToString} causes a problem because the attempt to use \texttt{realloc} will use a pointer to \texttt{s} that is no longer valid (because it got stomped by the earlier call to \texttt{realloc}).

\subsection*{Functional Programming and Parallelization}
Interestingly, functional programming languages (by which I do NOT mean procedural programming languages like C) such as Haskell and Scala and so on, lend themselves very nicely to being parallelized. Why? Because a purely functional program has no side effects and they are very easy to parallelize. If a function is impure, its functions signature will indicate so. Thus spake Joel\footnote{``Thus Spake Zarathustra'' is a book by Nietzsche, and this was not a spelling error.}:

\begin{quote}
\textit{Without understanding functional programming, you can't invent MapReduce, the algorithm that makes Google so massively scalable. The terms Map and Reduce come from Lisp and functional programming. MapReduce is, in retrospect, obvious to anyone who remembers from their 6.001-equivalent programming class that purely functional programs have no side effects and are thus trivially parallelizable.}~\cite{joel:functional}
\end{quote}

This assumes of course that there is no data dependency between functions. Obviously, if we need a computation result, then we have to wait. 

Object oriented programming kind of gives us some bad habits in this regard: we tend to make a lot of \texttt{void} methods. In functional programming these don't really make sense, because if it's purely functional, then there are some inputs and some outputs. If a function returns nothing, what does it do? For the most part it can only have side effects which we would generally prefer to avoid if we can, if the goal is to parallelize things. 

\paragraph{C++: the functional version?} {\tt algorithms} has been part of C++ since C++11. It provides algorithm implementations as part of the standard library. Some of the algorithms are standard: {\tt sort}, {\tt reverse}, {\tt is\_heap}\ldots

What's new in C++17, though, is parallel and vectorized {\tt
algorithms}. You can specify an execution policy for these algorithms
({\tt sequenced\_policy}, {\tt parallel\_policy}, or {\tt
parallel\_unsequenced\_policy}). The compiler and runtime make it
so. (Or, they don't. As of this writing in 2018, mainstream C++
compilers don't support C++17 yet).

As part of this process, C++17 also introduced some new algorithms, such as
{\tt for\_each\_n}, {\tt exclusive\_scan}, and {\tt reduce}.
If you know functional programming (e.g. Haskell), these are also known as
{\tt map}, {\tt scanl}, and {\tt fold1/foldl1}.

Rainer Grimm writes more about these in blogposts from February and May of 2017:
\cite{grimm17:_paral_algor_stand_templ_librar} \cite{grimm17:_c}.


\section*{Good Programming Practices: Inlining}
We have seen the notion of inlining:
  \begin{itemize}
    \item Instructs the compiler to just insert the function code in-place,
      instead of calling the function.
    \item Hence, no function call overhead!
    \item Compilers can also do better---context-sensitive---operations they couldn't
      have done before.
  \end{itemize}

OK, so inlining removes overhead. Sounds like better performance! Let's inline everything!
There are two ways of inlining in C++.

\paragraph{Implicit inlining.} (defining a function inside a class definition):
  \begin{lstlisting}[language=C]
class P {
public:
    int get_x() const { return x; }
...
private:
    int x;
};
  \end{lstlisting}

\paragraph{Explicit inlining.} Or, we can be explicit:
  \begin{lstlisting}[language=C]
inline max(const int& x, const int& y) {
    return x < y ? y : x;
}
  \end{lstlisting}

\paragraph{The Other Side of Inlining.}
Inlining has one big downside:
  \begin{itemize}
    \item Your program size is going to increase.
  \end{itemize}
   This is worse than you think:
      \begin{itemize}
        \item Fewer cache hits.
        \item More trips to memory.
      \end{itemize}
   Some inlines can grow very rapidly (C++ extended constructors).
  Just from this your performance may go down easily.

  Note also that inlining is merely a suggestion to compilers~\cite{gcc:inlining}.
  They may ignore you.
  For example:
  \begin{itemize}
    \item taking the address of an ``inline'' function and using it; or
    \item virtual functions (in C++),
  \end{itemize}
  will get you ignored quite fast.

\paragraph{Implications of inlining.} Inlining can make your life worse in two ways.
First, debugging is more difficult (e.g. you can't set a breakpoint in a function that
  doesn't actually exist).
 Most compilers simply won't inline code with debugging symbols on.
 Some do, but typically it's more of a pain.

Second, it can be a problem for library design:
  \begin{itemize}
    \item If you change any inline function in your library, any users
      of that library have to {\bf recompile} their program if the
      library updates. (Congratulations, you made a non-binary-compatible change!)
  \end{itemize}
This would not be a problem for non-inlined functions---programs execute the new function
dynamically at runtime.

\section*{High-Level Language Performance Tweaks}
So far, we've only seen C---we haven't seen anything complex, and C is
low level, which is good for learning what's really going on.

 Writing compact, readable code in C is hard, especially when \#define
macros and {\tt void *} beckon.

    C++11 has made major strides towards readability and
    efficiency---it provides light-weight abstractions. We'll look at
    a couple of examples.

\paragraph{Sorting.} Our goal is simple: we'd like to sort a bunch of integers.
In C, you would usually just use qsort from {\tt stdlib.h}.

  \begin{lstlisting}
void qsort (void* base, size_t num, size_t size,
            int (*comparator) (const void*, const void*));
  \end{lstlisting}

This is a fairly ugly definition (as usual, for generic C functions). How ugly is it?
Let's look at a usage example.
  \begin{lstlisting}[language=C]
#include <stdlib.h>

int compare(const void* a, const void* b)
{
    return (*((int*)a) - *((int*)b));
}

int main(int argc, char* argv[])
{
    int array[] = {4, 3, 5, 2, 1};
    qsort(array, 5, sizeof(int), compare);
}
  \end{lstlisting}
This looks like a nightmare, and is more likely to have bugs than what we'll see next.


C++ has a sort with a much nicer interface\footnote{\ldots well, nicer to use, after you get over templates.}:

  \begin{lstlisting}[language=C++]
template <class RandomAccessIterator>
void sort (
    RandomAccessIterator first,
    RandomAccessIterator last
);

template <class RandomAccessIterator, class Compare>
void sort (
    RandomAccessIterator first,
    RandomAccessIterator last,
    Compare comp
);
  \end{lstlisting}
It is, in fact, easier to use:
  \begin{lstlisting}[language=C++]
#include <vector>
#include <algorithm>

int main(int argc, char* argv[])
{
    std::vector<int> v = {4, 3, 5, 2, 1};
    std::sort(v.begin(), v.end());
}
  \end{lstlisting}

{\bf Note:} Your compare function can be a function or a functor. (Don't know what functors
are? In C++, they're functions with state.) By default,
  {\tt sort} uses {\tt operator$<$} on the objects being sorted.

  \begin{itemize}
    \item Which is less error prone?
    \item Which is {\bf faster}?
  \end{itemize}

The second question is empirical. Let's see. We generate an array of 2 million ints
and sort it (10 times, taking the average).

\begin{itemize}
\item qsort: 0.49 seconds
\item C++ sort: 0.21 seconds
\end{itemize}

The C++ version is {\bf twice} as fast. Why?
      \begin{itemize}
        \item The C version just operates on memory---it has no clue about the
          data.
        \item We're throwing away useful information about what's being sorted.
        \item A C function-pointer call prevents inlining of the compare function.
      \end{itemize}
OK. What if we write our own sort in C, specialized for the data?

\begin{itemize}
\item Custom C sort: 0.29 seconds
\end{itemize}

Now the C++ version is still faster (but it's close). But, this is
quickly going to become a maintainability nightmare.
      \begin{itemize}
        \item Would you rather read a custom sort or 1 line?
        \item What (who) do you trust more?
      \end{itemize}

\subsection*{Lesson}
Abstractions will not make your program slower. 

\noindent
They allow speedups and are much easier to maintain and read.

\subsection*{Vectors vs Lists}
Consider two
problems.

\begin{enumerate}
\item Generate {\bf N} random integers and insert them into (sorted)
      sequence.
      
      {\bf Example:} 3 4 2 1
      
      \begin{itemize}
        \item 3
        \item 3 4
        \item 2 3 4
        \item 1 2 3 4
      \end{itemize}

\item Remove {\bf N} elements one-at-a-time by going to a random position
      and removing the element.

      {\bf Example:} 2 0 1 0
      
      \begin{itemize}
        \item 1 2 4
        \item 2 4
        \item 2
        \item 
      \end{itemize}
\end{enumerate}

For which {\bf N} is it better to use a list than a vector (or array)?

 
\paragraph{Complexity analysis.} As good computer scientists, let's analyze
the complexity.  

{\bf Vector}:\\[-2em]
      \begin{itemize}
        \item Inserting\\[-2em]
          \begin{itemize}
            \item $O(\log n)$ for binary search
            \item $O(n)$ for insertion (on average, move half the elements)
          \end{itemize}
        \item Removing\\[-2em]
          \begin{itemize}
            \item $O(1)$ for accessing
            \item $O(n)$ for deletion (on average, move half the elements)
          \end{itemize}
      \end{itemize}

{\bf List}:\\[-2em]
      \begin{itemize}
        \item Inserting\\[-2em]
          \begin{itemize}
            \item $O(n)$ for linear search
            \item $O(1)$ for insertion
          \end{itemize}
        \item Removing\\[-2em]
          \begin{itemize}
            \item $O(n)$ for accessing
            \item $O(1)$ for deletion
          \end{itemize}
      \end{itemize}

Therefore, based on their complexity, lists should be better.

\paragraph{Reality.} OK, here's what happens. 
\begin{verbatim}
$ ./vector_vs_list 50000
Test 1
======
vector: insert 0.1s   remove 0.1s   total 0.2s
list:   insert 19.44s   remove 5.93s   total 25.37s
Test 2
======
vector: insert 0.11s   remove 0.11s   total 0.22s
list:   insert 19.7s   remove 5.93s   total 25.63s
Test 3
======
vector: insert 0.11s   remove 0.1s   total 0.21s
list:   insert 19.59s   remove 5.9s   total 25.49s
\end{verbatim}

{\bf Vectors} dominate lists, performance wise. Why?
  \begin{itemize}
    \item Binary search vs. linear search complexity dominates.
    \item Lists use far more memory.
      {\bf On 64 bit machines:}
      \begin{itemize}
        \item Vector: 4 bytes per element.
        \item List: At least 20 bytes per element.
      \end{itemize}
    \item Memory access is slow, and results arrive in blocks:
      \begin{itemize}
        \item Lists' elements are all over memory, hence many
          cache misses.
        \item A cache miss for a vector will bring a lot more usable data.
      \end{itemize}
  \end{itemize}

So, here are some tips for getting better performance.
  \begin{itemize}
    \item Don't store unnecessary data in your program.
    \item Keep your data as compact as possible.
    \item Access memory in a predictable manner.
    \item Use vectors instead of lists by default.
    \item Programming abstractly can save a lot of time.
    \item Often, telling the compiler more gives you better code.
    \item Data structures can be critical, sometimes more than complexity.
    \item {\bf Low-level code != Efficient}.
    \item Think at a low level if you need to optimize anything.
    \item Readable code is good code---different hardware needs different
      optimizations.
  \end{itemize}


\bibliographystyle{alphaurl}
\bibliography{459}


\end{document}
