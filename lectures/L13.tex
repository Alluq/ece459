\documentclass[letterpaper,10pt]{article}

\usepackage{enumitem}
\usepackage{titling}
\usepackage{listings,listings-rust}
\usepackage{url}
\usepackage{soul}
\usepackage{hyperref}
\usepackage{setspace}
\usepackage{subfig}
\usepackage{sectsty}
\usepackage{pdfpages}
\usepackage{colortbl}
\usepackage{multirow}
\usepackage{multicol}
\usepackage{relsize}
\usepackage{amsmath}
\usepackage{wasysym}
\usepackage{fancyvrb}
\usepackage[yyyymmdd]{datetime}
\usepackage{amsmath,amssymb,amsthm,graphicx,xspace}
\usepackage[titlenotnumbered,noend,noline]{algorithm2e}
\usepackage[compact]{titlesec}
\usepackage{XCharter}
\usepackage[T1]{fontenc}
\usepackage[scaled]{beramono}
\usepackage[normalem]{ulem}
\usepackage{booktabs}
\usepackage{tikz}
\usetikzlibrary{arrows.meta,automata,shapes,trees,matrix,chains,scopes,positioning,calc,decorations.pathreplacing}
\tikzstyle{block} = [rectangle, draw, fill=blue!20, 
    text width=2.5em, text centered, rounded corners, minimum height=2em]
\tikzstyle{bw} = [rectangle, draw, fill=blue!20, 
    text width=4em, text centered, rounded corners, minimum height=2em]

\definecolor{namerow}{cmyk}{.40,.40,.40,.40}
\definecolor{namecol}{cmyk}{.40,.40,.40,.40}
\renewcommand{\dateseparator}{-}

\let\LaTeXtitle\title
\renewcommand{\title}[1]{\LaTeXtitle{\textsf{#1}}}

\lstset{basicstyle=\footnotesize\ttfamily,breaklines=true}

\newcommand{\CPP}{C\nolinebreak\hspace{-.05em}\raisebox{.4ex}{\tiny\bf +}\nolinebreak\hspace{-.10em}\raisebox{.4ex}{\tiny\bf +}}
\def\CPP{{C\nolinebreak[4]\hspace{-.05em}\raisebox{.4ex}{\tiny\bf ++}}}

\newcommand{\handout}[5]{
  \noindent
  \begin{center}
  \framebox{
    \vbox{
      \hbox to 5.78in { {\bf ECE459: Programming for Performance } \hfill #2 }
      \vspace{4mm}
      \hbox to 5.78in { {\Large \hfill #4  \hfill} }
      \vspace{2mm}
      \hbox to 5.78in { {\em #3 \hfill \today} }
    }
  }
  \end{center}
  \vspace*{4mm}
}

\newcommand{\lecture}[3]{\handout{#1}{#2}{#3}{Lecture#1}}
\newcommand{\tuple}[1]{\ensuremath{\left\langle #1 \right\rangle}\xspace}

\addtolength{\oddsidemargin}{-1.000in}
\addtolength{\evensidemargin}{-0.500in}
\addtolength{\textwidth}{2.0in}
\addtolength{\topmargin}{-1.000in}
\addtolength{\textheight}{1.75in}
\addtolength{\parskip}{\baselineskip}
\setlength{\parindent}{0in}
\renewcommand{\baselinestretch}{1.5}
\newcommand{\term}{Winter 2020}

\singlespace


\begin{document}

\lecture{13 --- OpenMP}{\term}{Patrick Lam}

\section*{OpenMP}
%\paragraph{Moving on.} Now that we've seen automatic parallelization (and how that works in Solaris Studio and gcc), let's talk about manual---but compiler-aided---parallelization using OpenMP. 

\paragraph{About OpenMP.} OpenMP (Open Multiprocessing) 
is an API specification which allows you to tell the compiler how you'd 
like your program to be parallelized. Implementations of OpenMP 
include compiler support (present in Intel's compiler, Solaris's 
compiler, {\tt gcc} as of 4.2, and Microsoft Visual C++) as well as a 
runtime library.

You use OpenMP~\cite{omptutorial} by specifying
directives in the source code. In C and C++, these directives are
pragmas of the form \verb+#pragma omp ...+. There is also OpenMP
syntax for Fortran. 

Here are some benefits of the OpenMP approach:
\begin{itemize}
\item Because OpenMP uses compiler directives, you can easily tell the
  compiler to build a parallel version or a serial version (which it can do by
  ignoring the directives). This can simplify debugging---you
  have some chance of observing differences in behaviour between 
  versions.
\item OpenMP's approach also separates the parallelization
  implementation (inserted by the compiler) from the algorithm
  implementation (which you provide), making the algorithm easier to
  read. Plus, you're not responsible for dealing with thread libraries.
\item The directives apply to limited parts of the code, thus supporting
  incremental parallelization of the program, starting with the hotspots.
\end{itemize}

Let's look at a simple example:
{\small
\begin{verbatim}
  void calc (double *array1, double *array2, int length) {
    #pragma omp parallel for
    for (int i = 0; i < length; i++) {
      array1[i] += array2[i];
    }
  }
\end{verbatim}
}
This \verb+#pragma+ instructs the C compiler to parallelize the
loop. It is the responsibility of the developer to make sure that
the parallelization is safe; for instance, {\tt array1} and {\tt array2}
had better not overlap. You no longer need to supply {\tt restrict}
qualifiers, although it's still not a bad idea. (If you wanted this
to be autoparallelized without OpenMP, you would need to provide
{\tt restrict}.)

OpenMP will always start parallel threads if you tell it to, dividing
the iterations contiguously among the threads.

Let's look at the parts of this \verb+#pragma+.
\begin{itemize}
\item \verb+#pragma omp+ indicates an OpenMP directive;
\item {\tt parallel} indicates the start of a parallel region; and
\item {\tt for} tells OpenMP to run the following {\tt for} loop in parallel.
\end{itemize}
When you run the parallelized program, the runtime library starts
up a number of threads and assigns a subrange of the loop range to 
each of the threads.

\paragraph{Restrictions.} OpenMP places some restrictions on
loops that it's going to parallelize:
\begin{itemize}
\item the loop must be of the form 
\[ \mbox{\tt for (init expression; test expression; increment expression)}; \]
\item the loop variable must be integer (signed or unsigned), pointer, or a C++
random access iterator;
\item the loop variable must be initialized to one end of the range;
\item the loop increment amount must be loop-invariant (constant with respect to the loop body); 
\item the test expression must be one of {\tt >}, {\tt >=}, {\tt <}, or {\tt <=}, and the comparison value (bound) must be loop-invariant.
\end{itemize}

(These restrictions therefore also apply to automatically parallelized
loops.) If you want to parallelize a loop that doesn't meet the 
restriction, restructure it so that it does, as we saw last time.

\paragraph{Runtime effect.} When you compile a program with 
OpenMP directives, the compiler generates code to spawn a \emph{team}
of threads and automatically splits off the worker-thread code into a
separate procedure. The code uses fork-join parallelism, so when the
master thread hits a parallel region, it gives work to the worker
threads, which execute and report back. Then the master thread
continues running, while the worker threads wait for more work.

You can specify the number of threads by setting the
\verb+OMP_NUM_THREADS+ environment variable (adjustable by calling 
\verb+omp_set_num_threads()+), and you can get the
Solaris compiler to tell you what it did by giving it the
options \verb+-xopenmp -xloopinfo+.

\section*{Variable scoping}
When using multiple threads, some variables, like loop counters,
should be thread-local, or \emph{private}, while other variables
should be \emph{shared} between threads. Changes to shared variables
are visible to all threads, while changes to private variables are
visible only to the changing thread. Let's look at the defaults that
OpenMP uses to parallelize the above code.

{ 
\begin{verbatim}
$ er_src parallel-for.o
     1.   void calc (double *array1, double *array2, int length) {
        <Function: calc>
    
    Source OpenMP region below has tag R1
    Private variables in R1: i
    Shared variables in R1: array2, length, array1
     2.     #pragma omp parallel for
    
    Source loop below has tag L1
    L1 autoparallelized
    L1 parallelized by explicit user directive
    L1 parallel loop-body code placed in function _$d1A2.calc along with 0 inner loops
    L1 multi-versioned for loop-improvement:dynamic-alias-disambiguation. 
        Specialized version is L2
     3.     for (int i = 0; i < length; i++) {
     4.       array1[i] += array2[i];
     5.     }
     6.   }
\end{verbatim}
}

We can see that the loop variable {\tt i} is private, while the {\tt
  array1}, {\tt array2} and {\tt length} variables are shared.
Actually, it would be fine for the {\tt length} variable to be either
shared or private, but if it was private, then you would have to copy
in the appropriate initial value. The {\tt array} variables, though, 
need to be shared.

\paragraph{Summary of default rules.} Loop variables are private; 
variables defined in parallel code are private; and variables defined
outside the parallel region are shared.

You can disable the default rules by specifying {\tt default(none)}
on the {\tt parallel} pragma, or you can give explicit scoping:

\verb+   #pragma omp parallel for private(i) shared(length, array1, array2)+

\subsection*{Reductions}
Recall that we introduced the concept of a reduction, e.g.
{\small
\begin{verbatim}
  for (int i = 0; i < length; i++) total += array[i];
\end{verbatim}
}
What is the appropriate scope for {\tt total}? We want each thread
to be able to write to it, but we don't want race conditions.
Fortunately, OpenMP can deal with reductions as a special case:

\verb!   #pragma omp parallel for reduction (+:total)!

\noindent
specifies that the {\tt total} variable is the accumulator for a
reduction over {\tt +}. OpenMP will create local copies of {\tt total} and 
combine them at the end of the parallel region.

\subsection*{Accessing Private Data outside a Parallel Region}
A related problem with private variables is that sometimes you need 
access to them outside their parallel region. Here's some contrived code.

{\small
\begin{verbatim}
  int data=1;
  #pragma omp parallel for private(data)
  for (int i = 0; i < 100; i++)
    printf ("data=%d\n", data);
\end{verbatim}
}
Since {\tt data} is private, OpenMP is not going to copy in the
initial value {\tt 1} for it, so you'll get an undefined value.  To
make OpenMP initialize private variables with the master thread's
values for those variables, use {\tt firstprivate(data)}.  Conversely,
to get a value out from the (sequentially) last iteration of the loop,
use {\tt lastprivate(data)}.

\paragraph{Thread-private Data.} A variant on {\tt private} is
{\tt threadprivate}. Thread-private data is also local to each
thread. The main difference is that {\tt private} is for transient
data, typically local variables, declared at the start of a region,
while {\tt threadprivate} is for persistent data, e.g. declared at
file scope, which lives beyond a single parallel region. Instead of
{\tt firstprivate(local)} for {\tt private(local)}, use {\tt
  copyin(global)} for {\tt threadprivate(global)}.

\begin{verbatim}
#include <omp.h>
#include <stdio.h>

int tid, a, b;

#pragma omp threadprivate(a)

int main(int argc, char *argv[])
{
    printf("Parallel #1 Start\n");
    #pragma omp parallel private(b, tid)
    {
        tid = omp_get_thread_num();
        a = tid;
        b = tid;
        printf("T%d: a=%d, b=%d\n", tid, a, b);
    }

    printf("Sequential code\n");
    printf("Parallel #2 Start\n");
    #pragma omp parallel private(tid)
    {
        tid = omp_get_thread_num();
        printf("T%d: a=%d, b=%d\n", tid, a, b);
    }

    return 0;
}    
  \end{verbatim}
This yields something like the following output:
\begin{verbatim}
% ./a.out
Parallel #1 Start
T6: a=6, b=6
T1: a=1, b=1
T0: a=0, b=0
T4: a=4, b=4
T2: a=2, b=2
T3: a=3, b=3
T5: a=5, b=5
T7: a=7, b=7
Sequential code
Parallel #2 Start
T0: a=0, b=0
T6: a=6, b=0
T1: a=1, b=0
T2: a=2, b=0
T5: a=5, b=0
T7: a=7, b=0
T3: a=3, b=0
T4: a=4, b=0
\end{verbatim}

\paragraph{Collapsing Loops.}
Usually, when you have nested loops, it's best to parallelize the
outermost loop. {\sf Why?} \\[2em]

Sometimes, however, that just doesn't work out. The main issue is that
the outermost loop may have too low a trip count to be worth
parallelizing. You could instead parallelize the inner loop by putting
the pragma just before the inner loop, but then you pay more overhead
than you need to. OpenMP therefore supports \emph{collapsing} loops,
which creates a single loop performing all the iterations of the
collapsed loops. Consider:

{\small
\begin{verbatim}
  #include <math.h>
  int main() {
    double array[2][10000];
    #pragma omp parallel for collapse(2)
    for (int i = 0; i < 2; i++)
      for (int j = 0; j < 10000; j++)
        array[i][j] = sin(i+j);
    return 0;
  }
\end{verbatim}
}

Parallelizing the outer loop only enables the use of 2 threads.
Parallelizing both loops together enables the use of up to 20,000
threads, although the loop body is too small for that to be
worthwhile. 

{\sf Where have you seen something like a manually collapsed loop?}\\[1em]

\subsection*{Better Performance Through Scheduling}
OpenMP tries to guess how many iterations to distribute to each thread
in a team. The default mode is called \emph{static scheduling}; in
this mode, OpenMP looks at the number of iterations it needs to run,
assumes they all take the same amount of time, and distributes them
evenly. So for 100 iterations and 2 threads, the first thread gets 50
iterations and the second thread gets 50 iterations.

This assumption doesn't always hold; consider, for instance, the 
following (contrived) code:
{\small
\begin{verbatim}
  double calc(int count) {
    double d = 1.0;
    for (int i = 0; i < count*count; i++) d += d;
    return d;
  }

  int main() {
    double data[200][100];
    int i, j;
    #pragma omp parallel for private(i, j) shared(data)
    for (int i = 0; i < 200; i++) {
      for (int j = 0; j < 200; j++) {
        data[i][j] = calc(i+j);
      }
    }
    return 0;
  }
\end{verbatim}
}
This code gives sublinear scaling, because the earlier iterations 
finish faster than the later iterations, and the program needs to wait
for all iterations to complete.

Telling OpenMP to use a \emph{dynamic schedule} can enable better
parallelization: the runtime distributes work to each thread in
chunks, which results in less waiting. Just add {\tt
  schedule(dynamic)} to the pragma. Of course, this has more overhead,
since the threads need to solicit the work, and there is a potential
serialization bottleneck in soliciting work from the single work
queue. 

The default chunk size is 1, but you can specify it yourself, either
using a constant or a value computed at runtime, e.g. 
{\tt schedule(dynamic, n/50)}. Static scheduling also accepts a
chunk size.

OpenMP has an even smarter work distribution mode, {\tt guided}, where
it changes the chunk size according to the amount of work remaining.
You can specify a minimum chunk size, which defaults to 1. There
are also two meta-modes, {\tt auto}, which leaves it up to OpenMP, and
{\tt runtime}, which leaves it up to the \verb+OMP_SCHEDULE+ environment
variable.

\section*{Beyond {\tt for} Loops: OpenMP parallel sections and tasks}
The part of OpenMP we've seen so far has been strictly less powerful
than pthreads (but harder to misuse): we have only parallelized
specific forms of {\tt for} loops. This reflects OpenMP's
scientific-computation heritage, where you have huge FORTRAN matrix
calculations to parallelize. However, these days we also care about
parallelism in more general settings, so OpenMP now provides
more ways to parallelize.

\paragraph{Parallel Sections.} The first mechanism, \emph{parallel sections},
is a purely-static mechanism for specifying independent units of work
which ought to be run in parallel. For instance, you can set up two 
(and exactly two, in this example) linked lists simultaneously as follows:

{\small
\begin{verbatim}
  #include <stdlib.h>

  typedef struct s { struct s* next; } S;

  void setuplist (S* current) {
    for (int i = 0; i < 10000; i++) {
      current->next = (S*) malloc (sizeof(S));
      current = current->next;
    }
    current->next = NULL;
  }

  int main() {
    S var1, var2;
    #pragma omp parallel sections
    {
      #pragma omp section
      { setuplist (&var1); }
      #pragma omp section
      { setuplist (&var2); }
    }
    return 0;
  }
\end{verbatim}
}
Note that the structure of the parallelism is explicitly visible in the
structure of the code, and that you don't get to start an unbounded 
number of threads with the parallel sections mechanism.

{\sf What's another potential barrier to parallelism in the above code?}\\

\paragraph{Nested Parallelism.} Instead of collapsing loops, we can
specify nested parallelism; we might have, for instance, two parallel
sections, each of which contain parallel for loops. To enable
such nested parallelism, you have to call \verb+omp_set_nested+ with
a non-zero value. The runtime might refuse; call \verb+omp_get_nested+ to
find out if the runtime complied or not. You can also set the 
\verb+OMP_NESTED+ environment variable to enable nesting.

Here's an example of nested parallelism.
{\small
\begin{verbatim}
  #include <stdlib.h>
  #include <omp.h>

  int main() {
    double *array1, *array2;
    omp_set_nested(1);
    #pragma omp parallel sections shared(array1, array2)
    {
      #pragma omp section
      {
        array1 = (double*) malloc(sizeof (double)*1024*1024);
        #pragma omp parallel for shared(array1)
        for (int i = 0; i < 1024*1024; i++)
          array1[i] = i;
      }
      #pragma omp section
      {
        array2 = (double*) malloc(sizeof (double)*1024*512);
        #pragma omp parallel for shared(array2)
        for (int i = 0; i < 1024*512; i++)
          array2[i] = i;
      }
    }
  }
\end{verbatim}
}

\paragraph{Tasks: OpenMP's thread-like mechanism.}
The main new feature in OpenMP 3.0 is the notion of \emph{tasks}. When
the program executes a \verb+#pragma omp task+ statement, the code
inside the task is split off as a task and scheduled to run sometime
in the future. Tasks are more flexible than parallel sections, because
parallel sections constrain exactly how many threads are supposed to
run, and there is also always a join at the end of the parallel
section.  On the other hand, the OpenMP runtime can assign any task to
any thread that's running. Tasks therefore have lower overhead.

Two examples which show off tasks,
from~\cite{Ayguade:2009:DOT:1512157.1512430}, include a web server (with
unstructured requests) and a user interface which allows users to
start tasks that are to run in parallel.

Here's pseudocode for the Boa webserver main loop from~\cite{Ayguade:2009:DOT:1512157.1512430}.
{\small
\begin{verbatim}
#pragma omp parallel
  /* a single thread manages the connections */
  #pragma omp single nowait
  while (!end) {
    process any signals
    foreach request from the blocked queue {
      if (request dependencies are met) {
        extract from the blocked queue
        /* create a task for the request */
        #pragma omp task untied
          serve_request(request);
      }
    }
    if (new connection) {
      accept_connection();
      /* create a task for the request */
      #pragma omp task untied
        serve_request(new connection);
    }
    select();
  }
\end{verbatim}
}
The {\tt untied} qualifier lifts restrictions on the task-to-thread mapping;
we won't talk about that. The {\tt single} directive indicates that the
runtime is only to use one thread to execute the next statement; otherwise,
it could execute $N$ copies of the statement, which does belong to a
OpenMP {\tt parallel} construct.

\paragraph{More OpenMP features.} 
Random fact: you can use the {\tt flush} directive to make sure that
all values in registers or cache are written to memory. Usually, this
isn't a problem for OpenMP programs, because of the way they're
written. On a related note, the {\tt barrier} directive explicitly
instructs the runtime to wait for all threads to complete; OpenMP also
has implicit barriers at the end of parallel sections.  

OpenMP also supports critical sections (one thread at a time), atomic
sections, and typical mutex locks (\verb+omp_set_lock+,
\verb+omp_unset_lock+).

\paragraph{A Warning About Using OpenMP Directives.}
  Write your code so that simply eliding OpenMP directives gives a valid program. For instance, this is wrong:
  \begin{verbatim}
if (a != 0)
    #pragma omp barrier // wrong!
if (a != 0)
    #pragma omp taskyield // wrong!
  \end{lstlisting}

  Use this instead:
  \begin{lstlisting}
if (a != 0) {
    #pragma omp barrier
}
if (a != 0) {
    #pragma omp taskyield
}
  \end{verbatim}




\bibliographystyle{alphaurl}
\bibliography{459}


\end{document}
