\documentclass[letterpaper,10pt]{article}

\usepackage{enumitem}
\usepackage{titling}
\usepackage{listings,listings-rust}
\usepackage{url}
\usepackage{soul}
\usepackage{hyperref}
\usepackage{setspace}
\usepackage{subfig}
\usepackage{sectsty}
\usepackage{pdfpages}
\usepackage{colortbl}
\usepackage{multirow}
\usepackage{multicol}
\usepackage{relsize}
\usepackage{amsmath}
\usepackage{wasysym}
\usepackage{fancyvrb}
\usepackage[yyyymmdd]{datetime}
\usepackage{amsmath,amssymb,amsthm,graphicx,xspace}
\usepackage[titlenotnumbered,noend,noline]{algorithm2e}
\usepackage[compact]{titlesec}
\usepackage{XCharter}
\usepackage[T1]{fontenc}
\usepackage[scaled]{beramono}
\usepackage[normalem]{ulem}
\usepackage{booktabs}
\usepackage{tikz}
\usetikzlibrary{arrows.meta,automata,shapes,trees,matrix,chains,scopes,positioning,calc,decorations.pathreplacing}
\tikzstyle{block} = [rectangle, draw, fill=blue!20, 
    text width=2.5em, text centered, rounded corners, minimum height=2em]
\tikzstyle{bw} = [rectangle, draw, fill=blue!20, 
    text width=4em, text centered, rounded corners, minimum height=2em]

\definecolor{namerow}{cmyk}{.40,.40,.40,.40}
\definecolor{namecol}{cmyk}{.40,.40,.40,.40}
\renewcommand{\dateseparator}{-}

\let\LaTeXtitle\title
\renewcommand{\title}[1]{\LaTeXtitle{\textsf{#1}}}

\lstset{basicstyle=\footnotesize\ttfamily,breaklines=true}

\newcommand{\CPP}{C\nolinebreak\hspace{-.05em}\raisebox{.4ex}{\tiny\bf +}\nolinebreak\hspace{-.10em}\raisebox{.4ex}{\tiny\bf +}}
\def\CPP{{C\nolinebreak[4]\hspace{-.05em}\raisebox{.4ex}{\tiny\bf ++}}}

\newcommand{\handout}[5]{
  \noindent
  \begin{center}
  \framebox{
    \vbox{
      \hbox to 5.78in { {\bf ECE459: Programming for Performance } \hfill #2 }
      \vspace{4mm}
      \hbox to 5.78in { {\Large \hfill #4  \hfill} }
      \vspace{2mm}
      \hbox to 5.78in { {\em #3 \hfill \today} }
    }
  }
  \end{center}
  \vspace*{4mm}
}

\newcommand{\lecture}[3]{\handout{#1}{#2}{#3}{Lecture#1}}
\newcommand{\tuple}[1]{\ensuremath{\left\langle #1 \right\rangle}\xspace}

\addtolength{\oddsidemargin}{-1.000in}
\addtolength{\evensidemargin}{-0.500in}
\addtolength{\textwidth}{2.0in}
\addtolength{\topmargin}{-1.000in}
\addtolength{\textheight}{1.75in}
\addtolength{\parskip}{\baselineskip}
\setlength{\parindent}{0in}
\renewcommand{\baselinestretch}{1.5}
\newcommand{\term}{Winter 2020}

\singlespace


\begin{document}

\lecture{14 --- OpenMP Tasks, Memory Model}{\term}{Patrick Lam}

\section*{Tasks: OpenMP's thread-like mechanism.}
The main new feature in OpenMP 3.0 is the notion of \emph{tasks}. When
the program executes a \verb+#pragma omp task+ statement, the code
inside the task is split off as a task and scheduled to run sometime
in the future. Tasks are more flexible than parallel sections, because
parallel sections constrain exactly how many threads are supposed to
run, and there is also always a join at the end of the parallel
section.  On the other hand, the OpenMP runtime can assign any task to
any thread that's running. Tasks therefore have lower overhead.


  \begin{center}
    {\tt \#pragma omp }{\bf task} {\it [clause [[,] clause]*]}
  \end{center}~\\

Generates a task for a thread in the team to run.
     When a thread enters the region it may:
\begin{itemize}
        \item immediately execute the task; or
        \item defer its execution. (any other thread may be assigned the task)
\end{itemize}

  Allowed Clauses: {\bf if, final, untied, default, mergeable, private,
  firstprivate, shared}

\paragraph{{\tt if} and {\tt final} Clauses.}

  \begin{center}
  {\bf if} {\it(scalar-logical-expression)}
  \end{center}

    When expression is {\tt false}, generates an undeferred task---\\
    the generating task region is suspended until execution of the
      undeferred task finishes.\\[1em]

  \begin{center}
  {\bf final} {\it(scalar-logical-expression)}
  \end{center}

    When expression is {\tt true}, generates a final task.\\
    All tasks within a final task are {\it included}.\\
    Included tasks are undeferred and also execute immediately in the same thread.

Let's look at some examples of these clauses.
  \begin{lstlisting}[language=C]
void foo () {
    int i;
    #pragma omp task if(0) // This task is undeferred
    {
        #pragma omp task
        // This task is a regular task
        for (i = 0; i < 3; i++) {
            #pragma omp task
            // This task is a regular task
            bar();
        }
    }
    #pragma omp task final(1) // This task is a regular task
    {
        #pragma omp task // This task is included
        for (i = 0; i < 3; i++) {
            #pragma omp task
            // This task is also included
            bar();
        }
    }
}
  \end{lstlisting}

\paragraph{{\tt untied} and {\tt mergeable} Clauses.}

\begin{center}
  {\bf untied}
\end{center}
  \begin{itemize}
    \item A suspended task can be resumed by any thread.
    \item ``untied'' is ignored if used with {\bf final}.
    \item Interacts poorly with thread-private variables and {\tt gettid()}.
  \end{itemize}

\begin{center}
  {\bf mergeable}
\end{center}

  \begin{itemize}
    \item For an undeferred or included task,
    allows the implementation to generate a merged task instead.
    \item In a merged task, the implementation may re-use the environment from its generating task (as if there was no task directive).
  \end{itemize}

  For more: \url{docs.oracle.com/cd/E24457_01/html/E21996/gljyr.html}

\paragraph{Bad {\tt mergeable} Example.}

  \begin{lstlisting}[language=C]
#include <stdio.h>
void foo () {
    int x = 2;
    #pragma omp task mergeable
    {
        x++; // x is by default firstprivate
    }
    #pragma omp taskwait
    printf("%d\n",x); // prints 2 or 3
}
  \end{lstlisting}
  
    This is an incorrect usage of {\bf mergeable}: the output depends
      on whether or not the task got merged.
    Merging tasks (when safe) produces more efficient code.

\paragraph{Taskyield.}

  \begin{center}
    {\tt \#pragma omp }{\bf taskyield}
  \end{center}

This directive specifies that the current task can be suspended in favour of another task.

  Here's a good use of {\bf taskyield}.

  \begin{lstlisting}[language=C]
void foo (omp_lock_t * lock, int n) {
    int i;
    for ( i = 0; i < n; i++ )
    #pragma omp task
    {
        something_useful();
        while (!omp_test_lock(lock)) {
            #pragma omp taskyield
        }
        something_critical();
        omp_unset_lock(lock);
    }
}
  \end{lstlisting}

\paragraph{Taskwait.}
  \begin{center}
    {\tt \#pragma omp }{\bf taskwait}
  \end{center}~\\[1em]

     Waits for the completion of the current task's child tasks.

Two examples which show off tasks,
from~\cite{Ayguade:2009:DOT:1512157.1512430}, include a web server (with
unstructured requests) and a user interface which allows users to
start tasks that are to run in parallel.

Here's pseudocode for the Boa webserver main loop from~\cite{Ayguade:2009:DOT:1512157.1512430}.
{\small
\begin{lstlisting}[language=C,morekeywords={foreach,pragma,omp,parallel,single,nowait,task,untied,barrier,taskyield}]
#pragma omp parallel
  /* a single thread manages the connections */
  #pragma omp single nowait
  while (!end) {
    process any signals
    foreach request from the blocked queue {
      if (request dependencies are met) {
        extract from the blocked queue
        /* create a task for the request */
        #pragma omp task untied
          serve_request(request);
      }
    }
    if (new connection) {
      accept_connection();
      /* create a task for the request */
      #pragma omp task untied
        serve_request(new connection);
    }
    select();
  }
\end{lstlisting}
}
The {\tt untied} qualifier lifts restrictions on the task-to-thread mapping.
All it means is that a task that's been started by one thread could be picked up
by another and carried forward. This can have bad side effects if there's
thread-private data involved. The {\tt single} directive indicates that the
runtime is only to use one thread to execute the next statement; otherwise,
it could execute $N$ copies of the statement, which does belong to a
OpenMP {\tt parallel} construct.

\paragraph{Tree Traversal.}
  \begin{lstlisting}[language=C]
struct node {
    struct node *left;
    struct node *right;
};
extern void process(struct node *);

void traverse(struct node *p) {
    if (p->left) {
        #pragma omp task
        // p is firstprivate by default
        traverse(p->left);
    }
    if (p->right) {
        #pragma omp task
        // p is firstprivate by default
        traverse(p->right);
    }
    process(p);
}    
  \end{lstlisting}

If we want to guarantee a post-order traversal,
we simply need to insert an explicit \verb+#pragma omp taskwait+
after the two calls to {\tt traverse} and before the
call to {\tt process}.


\paragraph{Parallel Linked List Processing.} We can spawn 
tasks to process linked list entries. It's hard to
use two threads to traverse the list, though.

  \begin{lstlisting}[language=C]
// node struct with data and pointer to next
extern void process(node* p);

void increment_list_items(node* head) {
    #pragma omp parallel
    {
        #pragma omp single
        {
            node * p = head;
            while (p) {
                #pragma omp task
                {
                    process(p);
                }
                p = p->next;
            }
        }
    }
}
  \end{lstlisting}

\paragraph{Using Lots of Tasks.} Let's see what happens
if we spawn lots of tasks in a {\tt single} directive.

  \begin{lstlisting}[language=C]
#define LARGE_NUMBER 10000000
double item[LARGE_NUMBER];
extern void process(double);

int main() {
    #pragma omp parallel
    {
        #pragma omp single
        {
            int i;
            for (i=0; i<LARGE_NUMBER; i++) {
                #pragma omp task
                // i is firstprivate, item is shared
                process(item[i]);
            }
        }
    }
}
  \end{lstlisting}

In this case, the main loop (which executes in one thread only, due to {\tt single}) generates tasks and queues them for execution. When too many tasks get generated and are waiting, OpenMP suspends the main thread, runs some tasks, then resumes the loop in the main thread.

Any thread may pick up a task and execute it. Without {\tt untied}, a thread that starts a task has to finish running that task.

\paragraph{Improved code.} If we {\tt untied} the spawned tasks, that would enable the tasks to
migrate between threads when suspended. Just make sure that there is no threadprivate data that
is going to be wrong after a thread migration.

\section*{More Scoping Clauses}
Besides the {\tt shared}, {\tt private} and {\tt threadprivate}, OpenMP also 
supports {\tt firstprivate} and {\tt lastprivate}, which work like this.


Pthreads pseudocode for {\tt firstprivate} clause:
  \begin{lstlisting}
int x;

void* run(void* arg) {
    int thread_x = x;
    // use thread_x
}
  \end{lstlisting}

Pthread pseudocode for the {\tt lastprivate} clause:
  \begin{lstlisting}
int x;

void* run(void* arg) {
    int thread_x;
    // use thread_x
    if (last_iteration) {
        x = thread_x;
    }
}
  \end{lstlisting}
In other words, {\tt lastprivate} makes sure that the variable {\tt x}
has the same value as if the loop executed sequentially.

{\tt copyin} is like firstprivate, but for threadprivate variables.

Pthreads pseudocode for {\tt copyin}:
  \begin{lstlisting}[language=C]
int x;
int x[NUM_THREADS];

void* run(void* arg) {
  x[thread_num] = x;
  // use x[thread_num]
}
  \end{lstlisting}

The {\tt copyprivate} clause is only used with {\tt single}.
It copies the specified private variables from the thread to all other
threads. It cannot be used with {\tt nowait}.

\paragraph{Defaults.} {\bf default(shared)} makes all variables shared; 
{\bf default(none)} prevents sharing by default (creating compiler errors if
you treat a variable as shared.)



\paragraph{Catch Me Outside...}
A related problem with private variables is that sometimes you need 
access to them outside their parallel region. Here's some contrived code.

\begin{lstlisting}[language=C,morekeywords={foreach,pragma,omp,parallel,single,nowait,task,untied,barrier,taskyield}]
#include <omp.h>
#include <stdio.h>

int tid, a, b;

#pragma omp threadprivate(a)

int main(int argc, char *argv[])
{
    printf("Parallel #1 Start\n");
    #pragma omp parallel private(b, tid)
    {
        tid = omp_get_thread_num();
        a = tid;
        b = tid;
        printf("T%d: a=%d, b=%d\n", tid, a, b);
    }

    printf("Sequential code\n");
    printf("Parallel #2 Start\n");
    #pragma omp parallel private(tid)
    {
        tid = omp_get_thread_num();
        printf("T%d: a=%d, b=%d\n", tid, a, b);
    }

    return 0;
}    
  \end{lstlisting}
This yields something like the following output:
\begin{lstlisting}[language=C,morekeywords={foreach,pragma,omp,parallel,single,nowait,task,untied,barrier,taskyield}]
% ./a.out
Parallel #1 Start
T6: a=6, b=6
T1: a=1, b=1
T0: a=0, b=0
T4: a=4, b=4
T2: a=2, b=2
T3: a=3, b=3
T5: a=5, b=5
T7: a=7, b=7
Sequential code
Parallel #2 Start
T0: a=0, b=0
T6: a=6, b=0
T1: a=1, b=0
T2: a=2, b=0
T5: a=5, b=0
T7: a=7, b=0
T3: a=3, b=0
T4: a=4, b=0
\end{lstlisting}

\section*{OpenMP Memory Model, Its Pitfalls, and How to Mitigate Them}
OpenMP uses a {\bf relaxed-consistency, shared-memory} model. This almost certainly doesn't 
do what you want. Here are its properties:

\begin{itemize}
    \item All threads share a single store called
      \emph{memory}---this store may not actually represent RAM.
    \item Each thread can have its own {\it temporary} view of memory.
    \item A thread's {\it temporary} view of memory is not required to be
      consistent with memory.
\end{itemize}

We'll talk more about memory models later. Now we're going to talk about 
the OpenMP model and why it's a problem.

\paragraph{Memory Model Pitfall.} Consider this code.

  \begin{lstlisting}[language=C]
                    a = b = 0
/* thread 1 */                      /* thread 2 */

atomic(b = 1) // [1]                atomic(a = 1) // [3]
atomic(tmp = a) // [2]              atomic(tmp = b) // [4]
if (tmp == 0) then                  if (tmp == 0) then
    // protected section                // protected section
end if                              end if
  \end{lstlisting}

Does this code actually prevent simultaneous execution?
Let's reason about possible states.

  \begin{center}
  \begin{tabular}{r r r r | r r}
    \multicolumn{4}{c|}{Order} & t1 tmp & t2 tmp\\
    \hline
    1 & 2 & 3 & 4 & 0 & 1\\
    1 & 3 & 2 & 4 & 1 & 1\\
    1 & 3 & 4 & 2 & 1 & 1\\
    3 & 4 & 1 & 2 & 1 & 0\\
    3 & 1 & 2 & 4 & 1 & 1\\
    3 & 1 & 4 & 2 & 1 & 1\\
  \end{tabular}
  \end{center}

Looks like it (at least intuitively).

Sorry! With OpenMP's memory model, no guarantees:
the update from one thread may not be seen by the other.

\paragraph{Restoring Sanity with Flush.} We do rely on 
shared memory working ``properly'', but that's expensive.
So OpenMP provides the {\bf flush} directive.

  \begin{center}
    {\tt \#pragma omp }{\bf flush} {\it[(list)]}
  \end{center}

This directive makes the thread's temporary view of memory consistent with main
      memory; it:
\begin{itemize}
    \item enforces an order on the memory operations of the variables.
\end{itemize}

The variables in the list are called the {\it flush-set}. 
If you give no variables, the compiler will determine them for you.

Enforcing an order on the memory operations means:
\begin{itemize}
    \item All read/write operations on the {\it flush-set} which happen
      before the {\bf flush} complete before the flush executes.
    \item All read/write operations on the {\it flush-set} which happen
      after the {\bf flush} complete after the flush executes.
    \item Flushes with overlapping {\it flush-sets} can not be reordered.
\end{itemize}

To show a consistent value for a variable between two threads, OpenMP
must run statements in this order:

  \begin{enumerate}
    \item $t_1$ writes the value to $v$;
    \item $t_1$ flushes $v$; 
    \item $t_2$ flushes $v$ also;
    \item $t_2$ reads the consistent value from $v$.
  \end{enumerate}

Let's revise the example again.
  \begin{lstlisting}[language=C]
                    a = b = 0
/* thread 1 */                      /* thread 2 */

atomic(b = 1)                       atomic(a = 1)
flush(b)                            flush(a)
flush(a)                            flush(b)
atomic(tmp = a)                     atomic(tmp = b)
if (tmp == 0) then                  if (tmp == 0) then
    // protected section                // protected section
end if                              end if
  \end{lstlisting}

OK. Will this now prevent simultaneous access?

Well, no.

The compiler can reorder the {\tt flush(b)} in thread 1 or {\tt
  flush(a)} in thread 2. If {\tt flush(b)} gets reordered to after the
protected section, we will not get our intended operation.

\paragraph{Correct Example.} We have to provide a list of variables
to {\tt flush} to prevent re-ordering:
  \begin{lstlisting}[language=C]
                    a = b = 0
/* thread 1 */                      /* thread 2 */

atomic(b = 1)                       atomic(a = 1)
flush(a, b)                         flush(a, b)
atomic(tmp = a)                     atomic(tmp = b)
if (tmp == 0) then                  if (tmp == 0) then
    // protected section                // protected section
end if                              end if
  \end{lstlisting}

\paragraph{Where There Is Implicit Flush:}
  \begin{itemize}[noitemsep]
    \item {\tt omp barrier}
    \item at entry to, and exit from, {\bf omp critical};
    \item at exit from {\bf omp parallel}; 
    \item at exit from {\bf omp for};
    \item at exit from {\bf omp sections};
    \item at exit from {\bf omp single}.
  \end{itemize}

\paragraph{Where There's No Implicit Flush:}
  \begin{itemize}[noitemsep]
    \item at entry to {\bf for};
    \item at entry to, or exit from, {\bf master};
    \item at entry to {\bf sections};
    \item at entry to {\bf single};
    \item at exit from {\bf for}, {\bf single} or {\bf sections} with a {\bf nowait}
      \begin{itemize}
        \item {\bf nowait} removes implicit flush along with the implicit barrier
      \end{itemize}
  \end{itemize}

This is not true for OpenMP versions before 2.5, so be careful.

\paragraph{Final thoughts on flush.} We've seen that it's very difficult to use flush properly. Really, you should be using mutexes or other synchronization instead of flush~\cite{flush}, because you'll probably just get it wrong. But now you know what flush means.


\section*{Why Your Code is Slow}
OpenMP code too slow? Avoid these pitfalls:
  \begin{enumerate}
    \item Unnecessary flush directives.
    \item Using critical sections or locks instead of atomic.
    \item Unnecessary concurrent-memory-writing protection:
      \begin{itemize}
        \item No need to protect local thread variables.
        \item No need to protect if only accessed in {\bf single} or
          {\bf master}.
      \end{itemize}
    \item Too much work in a critical section.
    \item Too many entries into critical sections.
  \end{enumerate}

\paragraph{Example: Too Many Entries into Critical Sections.}~

  \begin{lstlisting}[language=C]
#pragma omp parallel for
for (i = 0; i < N; ++i) { 
    #pragma omp critical
    {
        if (arr[i] > max) max = arr[i];
    } 
}
  \end{lstlisting}

would be better as:

  \begin{lstlisting}[language=C]
#pragma omp parallel for
for (i = 0 ; i < N; ++i) { 
    #pragma omp flush(max)
    if (arr[i] > max) {
          #pragma omp critical
          {
                if (arr[i] > max) max = arr[i];
          }
    }
}
\end{lstlisting}


\bibliographystyle{alphaurl}
\bibliography{459}


\end{document}
