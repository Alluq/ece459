\documentclass[letterpaper,10pt]{article}

\usepackage{enumitem}
\usepackage{titling}
\usepackage{listings,listings-rust}
\usepackage{url}
\usepackage{soul}
\usepackage{hyperref}
\usepackage{setspace}
\usepackage{subfig}
\usepackage{sectsty}
\usepackage{pdfpages}
\usepackage{colortbl}
\usepackage{multirow}
\usepackage{multicol}
\usepackage{relsize}
\usepackage{amsmath}
\usepackage{wasysym}
\usepackage{fancyvrb}
\usepackage[yyyymmdd]{datetime}
\usepackage{amsmath,amssymb,amsthm,graphicx,xspace}
\usepackage[titlenotnumbered,noend,noline]{algorithm2e}
\usepackage[compact]{titlesec}
\usepackage{XCharter}
\usepackage[T1]{fontenc}
\usepackage[scaled]{beramono}
\usepackage[normalem]{ulem}
\usepackage{booktabs}
\usepackage{tikz}
\usetikzlibrary{arrows.meta,automata,shapes,trees,matrix,chains,scopes,positioning,calc,decorations.pathreplacing}
\tikzstyle{block} = [rectangle, draw, fill=blue!20, 
    text width=2.5em, text centered, rounded corners, minimum height=2em]
\tikzstyle{bw} = [rectangle, draw, fill=blue!20, 
    text width=4em, text centered, rounded corners, minimum height=2em]

\definecolor{namerow}{cmyk}{.40,.40,.40,.40}
\definecolor{namecol}{cmyk}{.40,.40,.40,.40}
\renewcommand{\dateseparator}{-}

\let\LaTeXtitle\title
\renewcommand{\title}[1]{\LaTeXtitle{\textsf{#1}}}

\lstset{basicstyle=\footnotesize\ttfamily,breaklines=true}

\newcommand{\CPP}{C\nolinebreak\hspace{-.05em}\raisebox{.4ex}{\tiny\bf +}\nolinebreak\hspace{-.10em}\raisebox{.4ex}{\tiny\bf +}}
\def\CPP{{C\nolinebreak[4]\hspace{-.05em}\raisebox{.4ex}{\tiny\bf ++}}}

\newcommand{\handout}[5]{
  \noindent
  \begin{center}
  \framebox{
    \vbox{
      \hbox to 5.78in { {\bf ECE459: Programming for Performance } \hfill #2 }
      \vspace{4mm}
      \hbox to 5.78in { {\Large \hfill #4  \hfill} }
      \vspace{2mm}
      \hbox to 5.78in { {\em #3 \hfill \today} }
    }
  }
  \end{center}
  \vspace*{4mm}
}

\newcommand{\lecture}[3]{\handout{#1}{#2}{#3}{Lecture#1}}
\newcommand{\tuple}[1]{\ensuremath{\left\langle #1 \right\rangle}\xspace}

\addtolength{\oddsidemargin}{-1.000in}
\addtolength{\evensidemargin}{-0.500in}
\addtolength{\textwidth}{2.0in}
\addtolength{\topmargin}{-1.000in}
\addtolength{\textheight}{1.75in}
\addtolength{\parskip}{\baselineskip}
\setlength{\parindent}{0in}
\renewcommand{\baselinestretch}{1.5}
\newcommand{\term}{Winter 2020}

\singlespace


\begin{document}

\lecture{19 --- Single-Thread Performance}{\term}{Patrick Lam}

\section*{Single-Thread Performance}

\hfill ``Can you run faster just by trying harder?''

The performance improvements we've seen to date have been leveraging parallelism
to improve throughput. Decreasing latency is trickier---it often requires domain-specific
tweaks. We'll look at one example of decreasing latency today, Stream VByte~\cite{LEMIRE20181}.
Even this example leverages parallelism---it uses vector instructions. But there
are some sequential improvements, e.g. Stream VByte takes care to be predictable
for the branch predictor.

\paragraph{Context.} We can abstract the problem to that of storing a sequence of small integers.
Such sequences are important, for instance, in the context of inverted indexes, which allow
fast lookups by term, and support boolean queries which combine terms.

Here is a list of documents and some terms that they contain:
\begin{center}
\begin{tabular}{r|l}
docid & terms \\ \hline
1 & dog, cat, cow\\
2 & cat\\
3 & dog, goat\\
4 & cow, cat, goat\\
\end{tabular}
\end{center}

The inverted index looks like this:
\begin{center}
\begin{tabular}{r|l}
term & docs \\ \hline
dog & 1, 3 \\
cat & 1, 2, 4 \\
cow & 1, 4 \\
goat & 3, 4
\end{tabular}
\end{center}

Inverted indexes contain many small integers in their lists: it is
sufficient to store the delta between a doc id and its successor, and
the deltas are typically small if the list of doc ids is sorted.
(Going from deltas to original integers takes time logarithmic
in the number of integers).

VByte is one of a number of schemes that use a variable number of
bytes to store integers.  This makes sense when most integers are
small, and especially on today's 64-bit processors.

VByte works like this:
\vspace*{-1em}
\begin{itemize}[noitemsep]
\item $x$ between 0 and $2^7-1$, e.g. $17 = 0b10001$: $0xxx xxxx$, e.g. $0001 0001$;
\item $x$ between $2^7$ and $2^{14}-1$, e.g. $1729 = 0b110 11000001$:
                   $1xxx xxxx/0xxx xxxx$, e.g. $1100 0001/0000 1101$;
\item $x$ between $2^{14}$ and $2^{21}-1$: $0xxx xxxx/1xxx xxxx/1xxx xxxx$;
\item etc.
\end{itemize}
That is, the control bit, or high-order bit, is 0 if you have finished representing the integer,
and 1 if more bits remain. (UTF-8 encodes the length, from 1 to 4, in high-order bits of the first byte.)

It might seem that dealing with variable-byte integers might be
harder than dealing fixed-byte integers, and it is. But there are performance benefits: because we are
using fewer bits, we can fit more information into our limited RAM and
cache, and even get higher throughput. Storing and reading 0s isn't an effective
use of resources. However, a naive algorithm to decode VByte also gives
lots of branch mispredictions.

Stream VByte is a variant of VByte which works using SIMD instructions.
Science is incremental, and Stream VByte builds on earlier work---masked VByte
as well as {\sc varint}-GB and {\sc varint}-G8IU. The innovation in
Stream VByte is to store the control and data streams separately.

Stream VByte's control stream uses two bits per integer to represent the size of the integer:
\begin{center}
\vspace*{-1em}
\begin{tabular}{ll@{~~~~~~~~}ll}
00 & 1 byte & 10 & 3 bytes\\
01 & 2 bytes & 11 & 4 bytes
\end{tabular}
\end{center}

Each decode iteration reads a byte from the control stream and 16 bytes of data from memory.
It uses a lookup table over the possible values of the control stream to decide how many
bytes it needs out of the 16 bytes it has read, and then uses SIMD instructions to shuffle
the bits each into their own integers. Note that, unlike VByte, Stream VByte uses all 8 bits
of each data byte as data.

For instance, if the control stream contains $0b1000~1100$, then the data stream
contains the following sequence of integer sizes: $3, 1, 4, 1$. Out of the 16 bytes read,
this iteration will use 9 bytes; it advances the data pointer by 9. It then uses the SIMD
``shuffle'' instruction to put the decoded integers from the data stream at known positions in the
128-bit SIMD register; in this case, it pads the first 3-byte integer with 1 byte, then
the next 1-byte integer with 3 bytes, etc. Let's say that the input is
{\tt 0xf823~e127~2524~9748~1b..~....~....~....}. The 128-bit output is
{\tt 0x00f8~23e1/0000~0027/2524 9748/0000/001b}, with the /s denoting separation
between outputs. The shuffle mask is precomputed and, at
execution time, read from an array.

The core of the implementation uses three SIMD instructions:
\begin{lstlisting}[language=C]
  uint8_t C = lengthTable[control];
  __m128i Data = _mm_loadu_si128 ((__m128i *) databytes);
  __m128i Shuf = _mm_loadu_si128(shuffleTable[control]);
  Data = _mm_shuffle_epi8(Data, Shuf);
  databytes += C; control++;
\end{lstlisting}

\paragraph{Discussion.} The paper~\cite{LEMIRE20181} includes a number of benchmark results
showing how Stream VByte performs better than previous techniques on a realistic input.
Let's discuss how it achieves this performance.

\begin{itemize}[noitemsep]
\item control bytes are sequential: the processor can always prefetch the next control byte, because
its location is predictable;
\item data bytes are sequential and loaded at high throughput;
\item shuffling exploits the instruction set so that it takes 1 cycle;
\item control-flow is regular (executing only the tight loop which retrieves/decodes control
and data; there are no conditional jumps).
\end{itemize}
We're exploiting SIMD, so this isn't quite strictly single-threaded performance.
Considering branch prediction and caching issues, though,
certainly improves single-threaded performance.


\bibliographystyle{alphaurl}
\bibliography{459}


\end{document}
