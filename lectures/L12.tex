\documentclass[letterpaper,10pt]{article}

\usepackage{enumitem}
\usepackage{titling}
\usepackage{listings,listings-rust}
\usepackage{url}
\usepackage{soul}
\usepackage{hyperref}
\usepackage{setspace}
\usepackage{subfig}
\usepackage{sectsty}
\usepackage{pdfpages}
\usepackage{colortbl}
\usepackage{multirow}
\usepackage{multicol}
\usepackage{relsize}
\usepackage{amsmath}
\usepackage{wasysym}
\usepackage{fancyvrb}
\usepackage[yyyymmdd]{datetime}
\usepackage{amsmath,amssymb,amsthm,graphicx,xspace}
\usepackage[titlenotnumbered,noend,noline]{algorithm2e}
\usepackage[compact]{titlesec}
\usepackage{XCharter}
\usepackage[T1]{fontenc}
\usepackage[scaled]{beramono}
\usepackage[normalem]{ulem}
\usepackage{booktabs}
\usepackage{tikz}
\usetikzlibrary{arrows.meta,automata,shapes,trees,matrix,chains,scopes,positioning,calc,decorations.pathreplacing}
\tikzstyle{block} = [rectangle, draw, fill=blue!20, 
    text width=2.5em, text centered, rounded corners, minimum height=2em]
\tikzstyle{bw} = [rectangle, draw, fill=blue!20, 
    text width=4em, text centered, rounded corners, minimum height=2em]

\definecolor{namerow}{cmyk}{.40,.40,.40,.40}
\definecolor{namecol}{cmyk}{.40,.40,.40,.40}
\renewcommand{\dateseparator}{-}

\let\LaTeXtitle\title
\renewcommand{\title}[1]{\LaTeXtitle{\textsf{#1}}}

\lstset{basicstyle=\footnotesize\ttfamily,breaklines=true}

\newcommand{\CPP}{C\nolinebreak\hspace{-.05em}\raisebox{.4ex}{\tiny\bf +}\nolinebreak\hspace{-.10em}\raisebox{.4ex}{\tiny\bf +}}
\def\CPP{{C\nolinebreak[4]\hspace{-.05em}\raisebox{.4ex}{\tiny\bf ++}}}

\newcommand{\handout}[5]{
  \noindent
  \begin{center}
  \framebox{
    \vbox{
      \hbox to 5.78in { {\bf ECE459: Programming for Performance } \hfill #2 }
      \vspace{4mm}
      \hbox to 5.78in { {\Large \hfill #4  \hfill} }
      \vspace{2mm}
      \hbox to 5.78in { {\em #3 \hfill \today} }
    }
  }
  \end{center}
  \vspace*{4mm}
}

\newcommand{\lecture}[3]{\handout{#1}{#2}{#3}{Lecture#1}}
\newcommand{\tuple}[1]{\ensuremath{\left\langle #1 \right\rangle}\xspace}

\addtolength{\oddsidemargin}{-1.000in}
\addtolength{\evensidemargin}{-0.500in}
\addtolength{\textwidth}{2.0in}
\addtolength{\topmargin}{-1.000in}
\addtolength{\textheight}{1.75in}
\addtolength{\parskip}{\baselineskip}
\setlength{\parindent}{0in}
\renewcommand{\baselinestretch}{1.5}
\newcommand{\term}{Winter 2020}

\singlespace


\begin{document}

\lecture{12 --- Autoparallelization and OpenMP}{\term}{Patrick Lam}

\section*{Automatic Parallelization and OpenMP}

\paragraph{Summary of conditions for automatic parallelization.} Here's what I 
can figure out; you may also refer to Chapter 3 of the Solaris Studio
\emph{C User's Guide}, but it doesn't spell out the exact conditions
either. To parallelize a loop, it must:
\begin{itemize}
\item have a recognized loop style, e.g. {\tt for} loops with
bounds that don't vary per iteration;
\item have no dependencies between data accessed in loop bodies for
  each iteration;
\item not conditionally change scalar variables read after the loop
  terminates, or change any scalar variable across iterations;
\item have enough work in the loop body to make parallelization profitable.
\end{itemize}

\paragraph{Reductions.} The concept behind a 
reduction (as made ``famous'' in MapReduce, which we'll talk about
later) is reducing a set of data to a smaller set which somehow
summarizes the data. For us, reductions are going to reduce
arrays to a single value. Consider, for instance, this function, which
calculates the sum of an array of numbers:

{
\begin{verbatim}
double sum (double *array, int length)
{
  double total = 0;

  for (int i = 0; i < length; i++)
    total += array[i];
  return total;
}
\end{verbatim}
}

There are two barriers: 1) the value of {\tt total} depends on what
gets computed in previous iterations; and 2) addition is actually
non-associative for floating-point values. ({\sf Why? When is it
appropriate to parallelize non-associative operations?})

Nevertheless, the Solaris C compiler will explicitly recognize
some reductions and can parallelize them for you:

{
\begin{verbatim}
$ cc -O3 -xautopar -xreduction -xloopinfo sum.c
"sum.c", line 5: PARALLELIZED, reduction, and serial version generated
\end{verbatim}
}

{\bf Note:}  If we try to do the reduction on {\tt fploop.c} with {\tt restrict}s added, we'll get the following:

\begin{verbatim}
$ cc -O3 -xautopar -xloopinfo  -xreduction -c fploop.c
"fploop.c", line 5: PARALLELIZED, and serial version generated
"fploop.c", line 8: not parallelized, not profitable
\end{verbatim}

\paragraph{Dealing with function calls.} Generally, function calls
can have arbitrary side effects. Production compilers will usually
avoid parallelizing loops with function calls; research compilers try
to ensure that functions are pure and then parallelize them.
(This is why functional languages are nice for parallel
programming: impurity is visible in type signatures.)

For builtin functions, like {\tt sin()}, you can promise to the 
compiler that you didn't replace them with your own implementations
({\tt -xbuiltin}), and then the compiler will parallelize the loop.

Another option is to crank up the optimization level ({\tt -xO4}), or
to explicitly tell the compiler to inline certain functions ({\tt
  -xinline=}), thereby enabling parallelization. This doesn't work as
well as one might hope; using macros will always work, but is less maintainable.

\paragraph{Helping the compiler parallelize.} Let's summarize what we've
seen. To help the compiler, we can use the {\tt restrict} qualifier on
pointers (possibly copying a pointer to a {\tt restrict}-qualified
pointer: {\tt int * restrict p = s->p;}); and, we can make sure that
loop bounds don't change in the loop (e.g. by using temporary
variables). Some compilers can automatically create different versions
for the alias-free case and the (parallelized) aliased case; at
runtime, the program runs the aliased case if the inputs permit.

\paragraph{What happened last time?} There was some confusion about manual parallelization. Recall that we manually parallelized three ways:

\begin{tabular}{ll}
  \begin{minipage}{5em} --- --- --- ---\\[-.8em] --- --- --- ---\\[-.8em] --- --- --- --- \end{minipage}& horizontal good: \\
      & \qquad create 4 threads to do 1000 iterations on sub-arrays.\\
      \begin{minipage}{5em} --- --- --- ---\\[-.8em] --- --- --- ---\\[-.8em] --- --- --- --- \end{minipage}& horizontal bad: \\
      & \qquad 1000 times, create 4 threads to iterate on sub-array. \\
      $ \mid \mid \mid\mid \: \mid \mid \mid \mid \: \mid \mid \mid \mid\: \mid \mid \mid \mid$& vertical:\\
      & \qquad create 4 threads, handle 1 element at a time.\\[1em]
\end{tabular}

Timings were inconclusive. I tried harder and got these
timings (in seconds) with {\tt perf -r 5}:

      \begin{center}
      \begin{tabular}{lrrrr}
        & H good & H bad & V & auto \\
        gcc, no opt & 2.794 & 2.953 & 2.799\\
        gcc, -O3 & 0.588 & 1.490 & 0.980\\
        solaris, no opt & 3.175 & 3.291 & 2.966 \\
        solaris, -xO4 & 0.494 & 1.453 & 2.739 & 0.688\\
      \end{tabular}
      \end{center}

{\tt perf} also told me other fun facts about the executions:

\begin{itemize}
\item fast executions had 3 to 7 cpu-migrations, slow ones had 4000 cpu-migrations.
\item branch misses varied from 8k (gcc -O3, H good) to 208k (gcc -O3, bad).
\item \# cycles varied from 2B (gcc -O3, H good) to 9.7B (unopt).\\
\end{itemize}

Turns out that these stats don't perfectly predict the runtime. Frontend cycles was really
high for solaris autoparallelization (which was quite fast). 
      

\paragraph{Moving on.} Now that we've seen automatic parallelization (and how that works in Solaris Studio
and gcc), let's talk about manual---but compiler-aided---parallelization using OpenMP. 

\paragraph{About OpenMP.} OpenMP (Open Multiprocessing) 
is an API specification which allows you to tell the compiler how you'd 
like your program to be parallelized. Implementations of OpenMP 
include compiler support (present in Intel's compiler, Solaris's 
compiler, {\tt gcc} as of 4.2, and Microsoft Visual C++) as well as a 
runtime library.

You use OpenMP\footnote{More information:
  \url{https://computing.llnl.gov/tutorials/openMP/}} by specifying
directives in the source code. In C and C++, these directives are
pragmas of the form \verb+#pragma omp ...+. There is also OpenMP
syntax for Fortran. 

Here are some benefits of the OpenMP approach:
\begin{itemize}
\item Because OpenMP uses compiler directives, you can easily tell the
  compiler to build a parallel version or a serial version (which it can do by
  ignoring the directives). This can simplify debugging---you
  have some chance of observing differences in behaviour between 
  versions.
\item OpenMP's approach also separates the parallelization
  implementation (inserted by the compiler) from the algorithm
  implementation (which you provide), making the algorithm easier to
  read. Plus, you're not responsible for dealing with thread libraries.
\item The directives apply to limited parts of the code, thus supporting
  incremental parallelization of the program, starting with the hotspots.
\end{itemize}

Let's look at a simple example:
{\small
\begin{verbatim}
  void calc (double *array1, double *array2, int length) {
    #pragma omp parallel for
    for (int i = 0; i < length; i++) {
      array1[i] += array2[i];
    }
  }
\end{verbatim}
}
This \verb+#pragma+ instructs the C compiler to parallelize the
loop. It is the responsibility of the developer to make sure that
the parallelization is safe; for instance, {\tt array1} and {\tt array2}
had better not overlap. You no longer need to supply {\tt restrict}
qualifiers, although it's still not a bad idea. (If you wanted this
to be autoparallelized without OpenMP, you would need to provide
{\tt restrict}.)

OpenMP will always start parallel threads if you tell it to, dividing
the iterations contiguously among the threads.

Let's look at the parts of this \verb+#pragma+.
\begin{itemize}
\item \verb+#pragma omp+ indicates an OpenMP directive;
\item {\tt parallel} indicates the start of a parallel region; and
\item {\tt for} tells OpenMP to run the following {\tt for} loop in parallel.
\end{itemize}
When you run the parallelized program, the runtime library starts
up a number of threads and assigns a subrange of the loop range to 
each of the threads.

\paragraph{Restrictions.} OpenMP places some restrictions on
loops that it's going to parallelize:
\begin{itemize}
\item the loop must be of the form 
\[ \mbox{\tt for (init expression; test expression; increment expression)}; \]
\item the loop variable must be integer (signed or unsigned), pointer, or a C++
random access iterator;
\item the loop variable must be initialized to one end of the range;
\item the loop increment amount must be loop-invariant (constant with respect to the loop body); 
\item the test expression must be one of {\tt >}, {\tt >=}, {\tt <}, or {\tt <=}, and the comparison value (bound) must be loop-invariant.
\end{itemize}

(These restrictions therefore also apply to automatically parallelized
loops.) If you want to parallelize a loop that doesn't meet the 
restriction, restructure it so that it does, as we saw last time.

\paragraph{Runtime effect.} When you compile a program with 
OpenMP directives, the compiler generates code to spawn a \emph{team}
of threads and automatically splits off the worker-thread code into a
separate procedure. The code uses fork-join parallelism, so when the
master thread hits a parallel region, it gives work to the worker
threads, which execute and report back. Then the master thread
continues running, while the worker threads wait for more work.

You can specify the number of threads by setting the
\verb+OMP_NUM_THREADS+ environment variable (adjustable by calling 
\verb+omp_set_num_threads()+), and you can get the
Solaris compiler to tell you what it did by giving it the
options \verb+-xopenmp -xloopinfo+.

\section*{Variable scoping}
When using multiple threads, some variables, like loop counters,
should be thread-local, or \emph{private}, while other variables
should be \emph{shared} between threads. Changes to shared variables
are visible to all threads, while changes to private variables are
visible only to the changing thread. Let's look at the defaults that
OpenMP uses to parallelize the above code.

{ 
\begin{verbatim}
$ er_src parallel-for.o
     1.   void calc (double *array1, double *array2, int length) {
        <Function: calc>
    
    Source OpenMP region below has tag R1
    Private variables in R1: i
    Shared variables in R1: array2, length, array1
     2.     #pragma omp parallel for
    
    Source loop below has tag L1
    L1 autoparallelized
    L1 parallelized by explicit user directive
    L1 parallel loop-body code placed in function _$d1A2.calc along with 0 inner loops
    L1 multi-versioned for loop-improvement:dynamic-alias-disambiguation. 
        Specialized version is L2
     3.     for (int i = 0; i < length; i++) {
     4.       array1[i] += array2[i];
     5.     }
     6.   }
\end{verbatim}
}

We can see that the loop variable {\tt i} is private, while the {\tt
  array1}, {\tt array2} and {\tt length} variables are shared.
Actually, it would be fine for the {\tt length} variable to be either
shared or private, but if it was private, then you would have to copy
in the appropriate initial value. The {\tt array} variables, though, 
need to be shared.

\paragraph{Summary of default rules.} Loop variables are private; 
variables defined in parallel code are private; and variables defined
outside the parallel region are shared.

You can disable the default rules by specifying {\tt default(none)}
on the {\tt parallel} pragma, or you can give explicit scoping:

\verb+   #pragma omp parallel for private(i) shared(length, array1, array2)+


\bibliographystyle{alphaurl}
\bibliography{459}


\end{document}
