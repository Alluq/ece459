\documentclass[letterpaper,10pt]{article}

\usepackage{titling}
\usepackage{listings}
\usepackage{url}
\usepackage{setspace}
\usepackage{subfig}
\usepackage{sectsty}
\usepackage{pdfpages}
\usepackage{colortbl}
\usepackage{multirow}
\usepackage{multicol}
\usepackage{relsize}
\usepackage{amsmath}
\usepackage{wasysym}
\usepackage{fancyvrb}
\usepackage{amsmath,amssymb,amsthm,graphicx,xspace}
\usepackage[titlenotnumbered,noend,noline]{algorithm2e}
\usepackage[compact]{titlesec}
\usepackage{XCharter}
\usepackage[T1]{fontenc}
\usepackage{tikz}
\usetikzlibrary{arrows,automata,shapes,trees,matrix,chains,scopes,positioning,calc}
\tikzstyle{block} = [rectangle, draw, fill=blue!20, 
    text width=2.5em, text centered, rounded corners, minimum height=2em]
\tikzstyle{bw} = [rectangle, draw, fill=blue!20, 
    text width=4em, text centered, rounded corners, minimum height=2em]

\definecolor{namerow}{cmyk}{.40,.40,.40,.40}
\definecolor{namecol}{cmyk}{.40,.40,.40,.40}

\let\LaTeXtitle\title
\renewcommand{\title}[1]{\LaTeXtitle{\textsf{#1}}}


\newcommand{\handout}[5]{
  \noindent
  \begin{center}
  \framebox{
    \vbox{
      \hbox to 5.78in { {\bf ECE459: Programming for Performance } \hfill #2 }
      \vspace{4mm}
      \hbox to 5.78in { {\Large \hfill #4  \hfill} }
      \vspace{2mm}
      \hbox to 5.78in { {\em #3 \hfill} }
    }
  }
  \end{center}
  \vspace*{4mm}
}

\newcommand{\lecture}[3]{\handout{#1}{#2}{#3}{Lecture #1}}
\newcommand{\tuple}[1]{\ensuremath{\left\langle #1 \right\rangle}\xspace}

\addtolength{\oddsidemargin}{-1.000in}
\addtolength{\evensidemargin}{-0.500in}
\addtolength{\textwidth}{2.0in}
\addtolength{\topmargin}{-1.000in}
\addtolength{\textheight}{1.75in}
\addtolength{\parskip}{\baselineskip}
\setlength{\parindent}{0in}
\renewcommand{\baselinestretch}{1.5}
\newcommand{\term}{Winter 2017}

\singlespace

\title{\bf ECE 459: Programming for Performance\\Assignment 3}
\author{Patrick Lam}
\date{\today ~ (Due: March 13, 2017)}

\lstset{basicstyle=\scriptsize, frame=single}



\begin{document}

\maketitle

In this assignment, you will write OpenCL code to implement parts of
an $N$-body simulation.  The first part consists of converting some
sequential C code into OpenCL, while the second part requires you to
improve the algorithm by making approximations for points that are far
away.

We'll be starting with the code from GPU Gems 3\footnote{\url{http://http.developer.nvidia.com/GPUGems3/gpugems3_ch31.html}}.
I also found an OpenCL N-body simulation code on the Internet\footnote{\url{http://www.browndeertechnology.com/docs/BDT_OpenCL_Tutorial_NBody-rev3.html}}. Also, you can read Andrew Cooke's notes; they are similar to my OpenCL lecture, but they may make more sense to you\footnote{\url{http://www.acooke.org/cute/APractical0.html}}.

Using OpenCL on ece459-1 should be fairly straightforward as long as
you have the right compiler options. The Makefile I've provided
should work for you. You are also free to use OpenCL on your own
computer, if you choose.

\paragraph{Getting started.} Fork the provided git repository:
\begin{center}
{\tt ssh git@ecgit.uwaterloo.ca fork ece459/1171/a3 ece459/1171/USERNAME/a3}
\end{center}
\noindent and then clone the provided files.

To submit, push C++ files containing OpenCL versions (naive and optimized) of
the $N$-body code, along with a brief report (about half a page).

\section*{Part 1: Brute-force approach (50 points)}
The easiest way to do an $N$-body simulation is to compute the effects
of all points on each other. I've posted some sequential code (based
on the GPU Gems code) to calculate the forces for one time-step. Your
first task is to convert this code to OpenCL. We will evaluate the
correctness and efficiency of your conversion. OpenCL can assign
workgroups automatically.

\section*{Part 2: Far-field approximations (50 points)}

Algorithmic improvements are important. In this part, you will
speed up the force calculation by crudely estimating the forces
exerted by faraway points. The idea is to divide the points into a
number of bins. (Real codes would use quadtrees or octrees to store,
and thus find, the closest points\footnote{For a readable summary
of the Barnes-Hut algorithm, see \url{http://arborjs.org/docs/barnes-hut}.}
). You will then compute the center
of mass for each bin and add the force exerted by the center of mass
for faraway bins to the force exerted by individual particles for
nearby particles.

\paragraph{Computing centers-of-masses for bins (20 points).}
Instead of finding the $k$ nearest points, we will divide space into a
fixed number of bins and compute the center of mass of each of these
bins. For nearby bins, we do the $O(n^2)$ calculation; for further
bins, we compute forces for each bin. Conveniently, for this assignment
I've divided space into $[0, 1000]^3$, so we can take bins which are
cubes of length 100. This gives 1000 bins.

I recommend that you create a 3-dimensional array {\tt cm} of {\tt
  float4}s\footnote{When I did the assignment, a {\tt float} versus {\tt float4}
mismatch took me a long time to debug.} to store centers-of-mass. The {\tt x}, {\tt y} and {\tt z}
components contain the average position of the center of mass of a
bin, while the {\tt w} component stores the total mass. Compute all of
the masses in parallel: create one thread per bin, and add a point's
position if it belongs to the bin, e.g.

{\small
\begin{verbatim}
    int xbin, ybin, zbin; // initialize with bin coordinates
    int b;
    for (i = 0; i < POINTS; i++) {
        if (pts[i] in bin coordinates) {
            cm[b].x += pts[i].x; // y, z too
            cm[b].w += 1.0f;
        }
    }
    cm[b].x /= cm[b].w; // etc
\end{verbatim}
}

\noindent
Note that this parallelizes with the number of bins.

\paragraph{Bin Contents (20 points).}
For the next step, you'll also need to keep track of the points in
each bin. Fortunately, you've collected the number of points in each
bin, so you can allocate the appropriate amount of memory to store the
points in a two-dimensional array {\tt binPts}. In a second phase,
iterate over all bins again, this time putting coordinates into the
proper element of {\tt binPts}.

\paragraph{Computing Forces (10 points).}
The payoff from all these calculations is to save time while
calculating forces. Let's arbitrarily say that we'll compute exact
forces for the points in the same bin and the directly-adjacent bins
in each direction (think of a Rubik's Cube; that makes 27 bins in all,
with 6 bins sharing a square, 12 bins sharing an edge, and 8 bins
sharing a point with the center bin). If there is no adjacent bin,
then there are no points in that bin. 

Using the data that you've
computed so far, write OpenCL code to estimate forces for each point.
This has two parts. In the first part, compute forces directly for the 
points in the 27 adjacent bins. In the second part, 
sum the forces from the centers of mass. 

There is a caveat: it's easier to add forces from all centers of mass,
whether nearby or far away. I recommend that you add forces from centers
of mass, and then subtract away the forces that you're double-counting:

\begin{verbatim}
  // add negative forces to not double-count adjacent bins
  negBin.x = 2*myPosition.x-globalCM[bin].x;
  negBin.y = 2*myPosition.y-globalCM[bin].y;
  negBin.z = 2*myPosition.z-globalCM[bin].z;
  negBin.w = globalCM[bin].w;
  bodyBodyInteraction(myPosition, negBin, pacc);
\end{verbatim}

Finally, compare the performance of part 1 and part 2.

\paragraph{What to hand in.} Commit and push your code.
For part 2,
write about your design choices and results (about half a page), and
hand that in along with your OpenCL code.

\section*{Timings}
Here's some timings from the past.

\paragraph{With 500 points.}
\begin{itemize}
\item    OpenCL, no approximations (1 kernel): 0.182s
\item    OpenCL, with approximations (3 kernels): 0.168s
\end{itemize}

\paragraph{With 5000 points.}
\begin{itemize}
\item    OpenCL, no approximations (1 kernel): 6.131s
\item    OpenCL, with approximations (3 kernels): 3.506s
\end{itemize}


\end{document}
