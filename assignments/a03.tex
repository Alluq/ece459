\documentclass[letterpaper,10pt]{article}

\usepackage{titling}
\usepackage{listings}
\usepackage{url}
\usepackage{setspace}
\usepackage{subfig}
\usepackage{sectsty}
\usepackage{pdfpages}
\usepackage{colortbl}
\usepackage{multirow}
\usepackage{multicol}
\usepackage{relsize}
\usepackage{amsmath}
\usepackage{wasysym}
\usepackage{fancyvrb}
\usepackage{amsmath,amssymb,amsthm,graphicx,xspace}
\usepackage[titlenotnumbered,noend,noline]{algorithm2e}
\usepackage[compact]{titlesec}
\usepackage{XCharter}
\usepackage[T1]{fontenc}
\usepackage{tikz}
\usetikzlibrary{arrows,automata,shapes,trees,matrix,chains,scopes,positioning,calc}
\tikzstyle{block} = [rectangle, draw, fill=blue!20, 
    text width=2.5em, text centered, rounded corners, minimum height=2em]
\tikzstyle{bw} = [rectangle, draw, fill=blue!20, 
    text width=4em, text centered, rounded corners, minimum height=2em]

\definecolor{namerow}{cmyk}{.40,.40,.40,.40}
\definecolor{namecol}{cmyk}{.40,.40,.40,.40}

\let\LaTeXtitle\title
\renewcommand{\title}[1]{\LaTeXtitle{\textsf{#1}}}


\newcommand{\handout}[5]{
  \noindent
  \begin{center}
  \framebox{
    \vbox{
      \hbox to 5.78in { {\bf ECE459: Programming for Performance } \hfill #2 }
      \vspace{4mm}
      \hbox to 5.78in { {\Large \hfill #4  \hfill} }
      \vspace{2mm}
      \hbox to 5.78in { {\em #3 \hfill} }
    }
  }
  \end{center}
  \vspace*{4mm}
}

\newcommand{\lecture}[3]{\handout{#1}{#2}{#3}{Lecture #1}}
\newcommand{\tuple}[1]{\ensuremath{\left\langle #1 \right\rangle}\xspace}

\addtolength{\oddsidemargin}{-1.000in}
\addtolength{\evensidemargin}{-0.500in}
\addtolength{\textwidth}{2.0in}
\addtolength{\topmargin}{-1.000in}
\addtolength{\textheight}{1.75in}
\addtolength{\parskip}{\baselineskip}
\setlength{\parindent}{0in}
\renewcommand{\baselinestretch}{1.5}
\newcommand{\term}{Winter 2017}

\singlespace

\title{\bf ECE 459: Programming for Performance\\Assignment 3}
\author{Jeff Zarnett \& Patrick Lam}
\date{\today ~ (Due: March 12, 2018)}

\lstset{basicstyle=\scriptsize, frame=single}



\begin{document}

\maketitle

In this assignment, you will convert your code that performs the Coulomb's Law 
simulation from assignment 2 to use OpenCL. This problem is a great candidate for
OpenCL because computation of the force on each point $p$ at a given time is
independent of the computation of any other point at that same time. 


OpenCL is covered in the lecture notes, but you can read Andrew Cooke's notes; they 
are similar to the OpenCL lecture, but they may make more sense to you\footnote{\url{http://www.acooke.org/cute/APractical0.html}}.

Using OpenCL on \texttt{ecetesla0} should be fairly straightforward as long as
you have the right compiler options. The Makefile I've provided
should work for you. You are also free to use OpenCL on your own
computer, if you choose. There are C++ as well as C bindings for OpenCL but you 
should use \textbf{only the C++ bindings}. The C binding has a nasty habit of 
crashing the machine (and/or rendering OpenCL unusable). If you use the C bindings, 
you will get 0 marks and we will be slightly displeased.

\paragraph{Getting started.} 
Check \url{git.uwaterloo.ca} for your repository which should 
be automatically created for you and populated with the starter files. The starter 
files are just the Makefile and some OpenCL boilerplate that will be necessary.
Plus also some test cases.

Then you can copy in your source files from assignment 2, question 3. You'll need to 
modify them extensively but it's probably much better to start with those than to 
start from scratch. But it's up to you!

\paragraph{What to submit.} Push C++ files as well as your kernel file(s), containing 
the OpenCL version of your code, along with a brief report (max 1 page).

\section*{Coulomb's Law Problem: Again!}

For a recap of the algorithm is supposed to work, please refer to the assignment 2 PDF (repeating it here only introduces the possibility of inconsistency or error). 
The input file format and the argument order will be the same at the command line. 
The arguments mean the same thing as they did in assignment2. 
The compiled executable file is named slightly differently to note that it's OpenCL. 


The output file produced by the 
OpenCL version should be exactly the same as an output file produced by the OpenMP or
sequential version for the same input file.

The goal of the assignment is to correctly convert the CPU code from assignment 2 into GPU (OpenCL) code. Correctness counts the most, but efficiency is also a good thing (you don't want it to be massively slower). It might now be clear why we used \texttt{float} in the previous assignment instead of \texttt{double}: because GPUs are very, very good at working with \texttt{float}s. 

\section*{Converting Your Code to OpenCL}

Step one would be to remove all OpenMP annotations as your code should not contain
OpenMP directives (the goal is, after all, to do work in OpenCL).

The OpenCL kernel is written in a language that is very C-like but it does have some
significant limitations compared to C. This makes it difficult to use 
structures like the \texttt{Vec3} that you were provided in assignment 2. But rest 
assured that we didn't trick you into doing something that would force a massive 
rewrite later. OpenCL gives you the types \texttt{float3} and \texttt{float4} (and other variants like 2, 8, 16...), which can take the place of \texttt{Vec3}.

A \texttt{float4}, for example, is a grouping of 4 single-precision floating point numbers. It is
equivalent to a \texttt{struct} composed of 4 \texttt{float} variables. It has
properties (like \texttt{.x}) to access the components, and there are already
defined arithmetic and comparison operations on them. So you should be able to drop
these types into your program, replacing \texttt{Vec3}, without great difficulty.
The links below should be helpful in getting familiar with the vector types and how 
to use them. They are both from the same article; the whole article might be worth a 
read but you should definitely read at least these two pages to understand the 
types and how the operations on them work:

\begin{center}

\url{http://www.informit.com/articles/article.aspx?p=1732873&seqNum=3}

\url{http://www.informit.com/articles/article.aspx?p=1732873&seqNum=10}

\end{center}

Beware, of course, of mismatches such as writing \texttt{float} when you mean 
\texttt{float4} or similar. Beware also that if you use the \texttt{w} property 
of \texttt{float4} for things like determining if something is a proton or electron 
then you might not get the desired behaviour for the default math operations unless
you use them carefully. 
Note that the types have slightly different names 
depending on where you use them: in the OpenCL kernel it's called
\texttt{float4} but in your C++ code it is \texttt{cl\_float4}, for example. 

Once you have converted to using the OpenCL vector types, it's probably a good idea 
to test with some test cases and make sure that nothing was broken by this. Then make 
a commit to save your work and you'll then go on to the next part, which is handing 
work over to the GPU. You do set-up and configuration in the host (CPU, C++) code, before giving the work to the GPU and collecting the results back to the host code.

You need to do at least the following operations in GPU (OpenCL kernel). You may \textbf{ not } do these things in CPU code because it defeats the purpose of the assignment: 

\begin{enumerate}
\item the $y_{1}$ computation
\item the $z_{1}$ computation
\item checking for error larger than the maximum allowed.
\end{enumerate} 

Hint for item 3:  You might think of this as either a reduction, or as finding the max value...

Your solution should probably consist of three kernels (but could be slightly more or less, depending).

\section*{Report}
In the report, write about your design choices and results (max half a page). Also 
write about what worked well in this assignment and where you had difficulties, as
we could use this in the future to give more guidance (also max half a page). Please use LaTeX for this; a template is provided for you in the base code that you cloned. 

\section*{Grading}
60\% of the marks for correctness of conversion; 30\% for efficiency; 10\% for report. Efficiency in this case means (1) clean code conversion, (2) leverages the GPU well (i.e., the code isn't executed in a very inefficient way), (3) doesn't do unnecessary work. Your code will be evaluated on \texttt{ecetesla0} so make sure it
works as expected there. 

\end{document}
