\documentclass[letterpaper,10pt]{article}

\usepackage{titling}
\usepackage{listings}
\usepackage{url}
\usepackage{setspace}
\usepackage{subfig}
\usepackage{sectsty}
\usepackage{pdfpages}
\usepackage{colortbl}
\usepackage{multirow}
\usepackage{multicol}
\usepackage{relsize}
\usepackage{amsmath}
\usepackage{wasysym}
\usepackage{fancyvrb}
\usepackage{amsmath,amssymb,amsthm,graphicx,xspace}
\usepackage[titlenotnumbered,noend,noline]{algorithm2e}
\usepackage[compact]{titlesec}
\usepackage{XCharter}
\usepackage[T1]{fontenc}
\usepackage{tikz}
\usetikzlibrary{arrows,automata,shapes,trees,matrix,chains,scopes,positioning,calc}
\tikzstyle{block} = [rectangle, draw, fill=blue!20, 
    text width=2.5em, text centered, rounded corners, minimum height=2em]
\tikzstyle{bw} = [rectangle, draw, fill=blue!20, 
    text width=4em, text centered, rounded corners, minimum height=2em]

\definecolor{namerow}{cmyk}{.40,.40,.40,.40}
\definecolor{namecol}{cmyk}{.40,.40,.40,.40}

\let\LaTeXtitle\title
\renewcommand{\title}[1]{\LaTeXtitle{\textsf{#1}}}


\newcommand{\handout}[5]{
  \noindent
  \begin{center}
  \framebox{
    \vbox{
      \hbox to 5.78in { {\bf ECE459: Programming for Performance } \hfill #2 }
      \vspace{4mm}
      \hbox to 5.78in { {\Large \hfill #4  \hfill} }
      \vspace{2mm}
      \hbox to 5.78in { {\em #3 \hfill} }
    }
  }
  \end{center}
  \vspace*{4mm}
}

\newcommand{\lecture}[3]{\handout{#1}{#2}{#3}{Lecture #1}}
\newcommand{\tuple}[1]{\ensuremath{\left\langle #1 \right\rangle}\xspace}

\addtolength{\oddsidemargin}{-1.000in}
\addtolength{\evensidemargin}{-0.500in}
\addtolength{\textwidth}{2.0in}
\addtolength{\topmargin}{-1.000in}
\addtolength{\textheight}{1.75in}
\addtolength{\parskip}{\baselineskip}
\setlength{\parindent}{0in}
\renewcommand{\baselinestretch}{1.5}
\newcommand{\term}{Winter 2020}

\singlespace


\title{\bf ECE 459: Programming for Performance\\Assignment 1}
\author{Jeff Zarnett}
%\date{\today ~ (Due: January 26, 2018)}
\date{\today ~ (Due: January 27, 2020 at 11:59PM Eastern Time)}

\begin{document}

\maketitle

In this assignment, you'll work with a program that solves sudokus 
(see description below if you're unfamiliar with it) and 
also verifies against an external webservice. We provide a starter
single-threaded implementation. You will modify this program in several
ways and discuss what worked well, what didn't, and why.
In part 2, you'll use nonblocking I/O to speed up the verification process.

\section*{Setup}


The course will use \url{https://git.uwaterloo.ca/} which is the university-
provided GitLab service. It is likely at this point you have already used 
either GitLab or GitHub (they are similar). You may need to set up your 
account but this is easy to do. If you're reading this and you haven't set up 
your account, do so immediately. Otherwise we can't give you the assignment 
starter code.

If you've previously configured your account at {\tt git.uwaterloo.ca}, you 
should find already that we've created a repository for you for this 
assignment (we are nice) and it will contain the starter files. Future 
assignments will work the same way.

You should do this assignment using the ECE-provided Ubuntu servers 
(eceubuntu). That's the environment that we tested the
assignment using. If you are having trouble with your environment or setup
the first question I will probably ask is ``are you running it locally?''.
Due to the diversity of possible setups, the course staff cannot spend
time figuring out library incompatibilities and personal laptop issues.
Sorry. If you are a non-ECE student (e.g., SYDE) you should have
access to the ECE Ubuntu servers due to being registered in an ECE course.

Remember that access to most UW server resources is restricted by the firewall 
and that you may need to enable your VPN connection to do this assignment, 
especially if you are working from off-campus.

To submit, simply push your fork of the git repository back to {\tt git.uwaterloo.ca}.

\section*{Sudoku}

Sudoku is a puzzle challenge. 
Here's the brief description from Wikipedia\footnote{
\url{https://en.wikipedia.org/wiki/Sudoku}}:
it is a number puzzle based on a $9\times9$ grid. The goal is to fill the
grid with digits so that each column, each row, and each of the nine
$3\times3$ subgrids contain all the digits from 1 through 9. Zero is not
a valid value and there can be no repeats within a row, column, or subgrid. 
The initial condition is always a partially-filled-in matrix that needs to 
be completed. A well-designed matrix has a single valid solution. 
It's easier to explain visually, so read the wikipedia page.
If you would like to generate
more test cases and their solutions, see
\url{https://qqwing.com/generate.html}


This sort of puzzle is pretty easy for computers to solve. There are clever
algorithms and strategies for humans but computers are perfectly happy to
brute force it. This is a parallelizable task!

\newpage

\section*{Part 1: Parallelization, Three Ways}

There's more than one way to parallelize your program, and in this
assignment you'll do it in three different ways. Each of these is a
strategy discussed in the course. This assignment gives you an opportunity
to practice them and reflect on their use. There are three different
starter files and you should modify each of them according to their named
strategy. In all cases, you will use \texttt{pthread}s to implement the 
functionality. If you need a refresher on how \texttt{pthreads} work,
there's a pdf in the course repository that should help! 

The program provides a makefile that is used to compile the code (and the
\LaTeX~report). The makefile isn't super robust but it means you can easily
compile the program using the command \texttt{make}. You can make changes
to the makefile if you need (but you might get frustrated because makefiles
are a bit difficult to work with). But anyway.

Your program takes a parameter \texttt{-t N} that specifies the number 
of threads. It's in the starter code but does not do anything at the start. 
The program also takes the parameter \texttt{-i inputfile} to specify the
input file. Output always appears in \texttt{output.txt}. It is always 
required to provide an input file, otherwise the program won't run.
A sample invocation then: \texttt{./sudoku -t 4 -i inputfile.txt}

Note that you are allowed to 
modify the code as needed to accomplish your strategy (e.g., add some
concurrency control or change a function's signature and/or implementation).
Just be sure you both (1) preserve the behaviour (solving puzzles) and
(2) use the strategy as specified below.

\paragraph{Strategy One.}
One thread per puzzle (\texttt{sudoku\_threads.c}). In this implementation 
each thread takes an entire puzzle from start to finish: read it, process it, 
write it to a file. This is pretty straightforward.

\paragraph{Strategy Two.}

One puzzle, multiple threads(\texttt{sudoku\_multi.c}). In this strategy, 
multiple threads contribute to solving a single puzzle. In particular, 
the \texttt{solve()} function is what should be parallelized. You decide how.

\paragraph{Strategy Three.} Workers with specific jobs (\texttt{sudoku\_workers.c}). In this implementation, different threads 
have different jobs: some threads read the input file, some process puzzles
(that is, execute the \texttt{solve}) function), 
and some write the output.
	
There's no reason why you must have equal numbers of each kind of thread.
You can choose whatever distribution of threads makes the most sense. You
should probably experiment to find out what works best. In this version,
if the argument specifying the number of threads is less than 3 (i.e., you
cannot have at least one thread of each type), it is okay to exit with an
error message.

\paragraph{General Notes.}
A quick way to get a look at how long your code has executed is the 
\texttt{time} command, which you put in front of the executable you would
like to run. This command tells you the user, system, and real time. The
user time is how much CPU time your code was executing; system is how much
was spent in system calls; real is the wall-clock time between the start of
execution and the end of execution. In this case, you are trying to reduce
the real time value (and this may be done by increasing the user time).


For all strategies, your code should be free of race conditions and other
concurrency problems. It should also not have memory leaks. Use Valgrind
to check things (both the memory check and the helgrind check). 
Make sure any library calls you use are thread-safe (you
can find out by looking at the man pages). Memory that is ``still reachable'' 
should be deallocated where possible. Remember that Valgrind may report
issues that you cannot fix because they are in a library not under your
control. You may also have benign race conditions. If either of those 
is the case, just make a note of that in your report for the marker to read.

You will want to avoid repeatedly creating \& destroying threads.
Think of strategy for how threads get work.

Also in the repository is a verifier tool. This tool can be used to validate
output files against an external service that checks the solution. You can
also of course compare the single-thread version's output against the output
of your modified code, but you might not write all puzzles in the same order
in the output file as they were in the input file. That is okay, you do not
have to maintain the order. For that reason, there is the checking tool
which would allow you to validate all of them efficiently.

\paragraph{Report, Number One.} In your report, you should show timing data 
for different runs of each version of the solver, comparing against the
provided starter code for different sizes of input (see the provided files).
Test with the number of threads $N \in \{3, 4, 16, 32\}$
Your data will make it clear which strategy or strategies work best under
what circumstances; you should explain a bit about \textbf{why} 
a strategy works
better or worse than the others. You should also include some discussion
of the level of difficulty of implementing each of the strategies and whether
it was worth it for the speedup over an easier strategy's speedup.

The report is made automatically by the makefile. \LaTeX~might seem scary
but it's really not. It is very much like programming a document and it's
a nice skill to learn. For the most part, you can just type in some text and
it compiles to a nice pdf. If you're having trouble with something, google
is good at telling you the answer. Also, we mark it for content, not for
formatting, so it doesn't have to be the prettiest document ever. 

\section*{Part 2: Nonblocking I/O}

In this part, you will modify the verifier tool so that it uses non-blocking
I/O. Its current (simple) implementation uses blocking I/O to verify the
output file from the previous part against an external service. 
Your solution should \emph{not} use pthreads. However, it should have multiple
concurrent connections to servers open. This is accomplished with the
\texttt{curl\_multi} interface.
In this case, the {\tt -t} commandline option indicates the number of 
concurrent connections to the server. The \texttt{-i} option is still used
to specify the input file.

The verifier reads the provided file and for each puzzle, it 
creates a JSON object and sends it to the server with an HTTP POST command. 
The endpoint on the report server is \texttt{verify}.
The endpoint will return HTTP 400 if anything is wrong with the provided data 
(including if the body is missing). It will return HTTP 200 if it is capable 
of parsing and understanding the data; if it is a valid Sudoku solution 
then a body of 1 is sent back; otherwise 0 is returned. The verifier counts
the number of successes and tells you that $x$ of $y$ were correct.

The implementation has a 50 ms delay built into it to simulate sending data 
over the actual internet. It is likely that for the assignment all the 
servers and clients will be in the same network (or at least geographically
nearby) and the delay makes the scenario more ``realistic''.

You should also use valgrind on part two. See the general notes of part one
for guidance about that.

\paragraph{Report.} Again, benchmark your work and report results as compared to the baseline (unmodified) program, for the number of concurrent connections $N \in \{3, 4, 16, 32\}$. Is the amount of the performance increase as expected? Why or why not?


\section*{Rubric}
The general principle is that correct solutions earn full marks.
However, it is your responsibility to demonstrate to the TA
that your solution is correct. Well-designed, clean solutions 
are more likely to be recognized as correct. 
Solutions that do not compile will earn at most 39\% of the available
marks for that part. Solutions that crash earn
at most 49\%. Grading is done by compiling, running it, checking the 
output, and code inspection. 

\paragraph{Part 1: Parallelization (60marks, 3x20)}
20 marks for each of the three strategies for a total of 60.
Each of the three is considered separately in the compilation/crash 
clauses of the general grading notes.
Your code needs to produce correct sudoku output (although the order of the
output can vary) and use the strategy specified.

\paragraph{Part 2: Nonblocking I/O (20 marks)}
20 marks for implementation; it must properly use \texttt{curl\_multi} and 
have the correct number of concurrent connections.

\paragraph{Part 3: Report (20 marks)}~\\
12 marks for discussing the strategies of part 1.\\
5 marks for discussion of part 2.\\
3 marks for clarity.


\end{document}
