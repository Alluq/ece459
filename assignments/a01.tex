\documentclass[letterpaper,10pt]{article}

\usepackage{titling}
\usepackage{listings}
\usepackage{url}
\usepackage{setspace}
\usepackage{subfig}
\usepackage{sectsty}
\usepackage{pdfpages}
\usepackage{colortbl}
\usepackage{multirow}
\usepackage{multicol}
\usepackage{relsize}
\usepackage{amsmath}
\usepackage{wasysym}
\usepackage{fancyvrb}
\usepackage{amsmath,amssymb,amsthm,graphicx,xspace}
\usepackage[titlenotnumbered,noend,noline]{algorithm2e}
\usepackage[compact]{titlesec}
\usepackage{XCharter}
\usepackage[T1]{fontenc}
\usepackage{tikz}
\usetikzlibrary{arrows,automata,shapes,trees,matrix,chains,scopes,positioning,calc}
\tikzstyle{block} = [rectangle, draw, fill=blue!20, 
    text width=2.5em, text centered, rounded corners, minimum height=2em]
\tikzstyle{bw} = [rectangle, draw, fill=blue!20, 
    text width=4em, text centered, rounded corners, minimum height=2em]

\definecolor{namerow}{cmyk}{.40,.40,.40,.40}
\definecolor{namecol}{cmyk}{.40,.40,.40,.40}

\let\LaTeXtitle\title
\renewcommand{\title}[1]{\LaTeXtitle{\textsf{#1}}}


\newcommand{\handout}[5]{
  \noindent
  \begin{center}
  \framebox{
    \vbox{
      \hbox to 5.78in { {\bf ECE459: Programming for Performance } \hfill #2 }
      \vspace{4mm}
      \hbox to 5.78in { {\Large \hfill #4  \hfill} }
      \vspace{2mm}
      \hbox to 5.78in { {\em #3 \hfill} }
    }
  }
  \end{center}
  \vspace*{4mm}
}

\newcommand{\lecture}[3]{\handout{#1}{#2}{#3}{Lecture #1}}
\newcommand{\tuple}[1]{\ensuremath{\left\langle #1 \right\rangle}\xspace}

\addtolength{\oddsidemargin}{-1.000in}
\addtolength{\evensidemargin}{-0.500in}
\addtolength{\textwidth}{2.0in}
\addtolength{\topmargin}{-1.000in}
\addtolength{\textheight}{1.75in}
\addtolength{\parskip}{\baselineskip}
\setlength{\parindent}{0in}
\renewcommand{\baselinestretch}{1.5}
\newcommand{\term}{Winter 2017}

\singlespace


\title{\bf ECE 459: Programming for Performance\\Assignment 1}
\author{Patrick Lam}
\date{\today ~ (Due: January 26, 2018)}

\begin{document}

\maketitle

In this assignment, you'll work with a program which requests a
resource across the network. I've provided 
a single-threaded implementation which uses blocking I/O to get the
resource. You will reduce the latency of this operation by sending out
multiple requests simultaneously (to different machines). In part 1,
you'll use pthreads to do this, while in part 2, you'll use
nonblocking I/O.

\section*{Setup}

%Download the provided assignment code from
%\url{http://patricklam.ca/p4p/files/provided-assignment-01.tar.gz}
%and untar it using the {\tt tar xzvf assignment-01.tar.gz} command.

The course will use \url{https://git.uwaterloo.ca/} which is the university-provided
GitLab service. It is likely at this point you have already used either GitLab or GitHub (they are similar). You may need to set up your account but this is easy to do.

After configuring your account at {\tt git.uwaterloo.ca}, you should find already that we've created a repository for you for this assignment (we are nice) and it will contain the starter files. Future assignments will work the same way, but you'll only need to set up your account the first time

You should do this assignment in Linux (for example, ecelinux), 
as the provided {\tt Makefile} was only tested on Linux and isn't very robust.
Use a virtual machine at your peril. It might work OK since the program's
bottleneck is the response time of the remote execution. But virtual machines come
with other hassles and you should be able to easily log into a Linux system at this
point.

Remember that access to most UW server resources is restricted by the firewall and that you may need to enable your VPN connection to do this assignment, especially if you are working from off-campus.

%You may log into {\tt ece459-1.uwaterloo.ca} with the ssh key that you
%set up at {\tt ecgit}. Use your uwuserid with
%that ssh key. I'll be testing your solutions on that machine.

%I tested everything on this system, so everything should
%be working. Just in case you want to make sure you setup is the same,
%I used {\tt gcc 4.6.2} and {\tt glibc 2.15}. Make sure to use the
%optimization flag {\tt -O2}, or you may get some weird results.

\section*{Assignment code walkthrough}

You will find the file {\tt paster.c} in the provided file. This code uses
{\tt libcurl} to fetch a set of {\tt PNG} files from the network and {\tt libpng}
to paste the files together.

I've provided a web API which returns portions of some pictures that I took.
You can see this in a browser by visiting 
\url{http://machine:4590/image?img=N},
where {\tt machine} is one of {\tt ece459-1.patricklam.ca}, {\tt
  ece459-2.patricklam.ca} and {\tt ece459-3.patricklam.ca}, and where $\mbox{\tt N} \in [1,3]$. This API returns a $200 \times 3000$ vertical strip, and uses
an HTTP response header to tell you which strip you got. You get back a random
segment each time you make a request, so it can take a variable amount of time to get all the pieces that you need.

The provided code repeatedly fetches image segments until it has them
all, puts them in an array, and then produces an output file, {\tt
  output.png}, with the pasted-together image.

\newpage

\section*{Part 0: Resource Leaks (5 marks)}

I inadvertently left a resource leak in the provided code. Resource
leaks sap performance. Find it (valgrind helps), fix it, and document
it in your report.

\section*{Part 1: Pthreads (45 marks)}

Use the {\tt pthread} library, create a threaded version of the
provided program.  Your program should create as many threads as the
{\tt num\_threads} variable (which reads the value from the {\tt -t}
command line option) and distribute the work between the 3 provided
servers. Make sure all of your library (standard glibc, libcurl, and
libpng) calls are {\it thread-safe} (for glibc, e.g. {\tt man 3 rand} to
look at the documentation).  

We will look at your code to ensure that it uses {\tt pthread} calls 
properly, and we will execute your code to verify that it produces the
correct output. Code that doesn't compile on {\tt ece459-1} will get at most
39\%.

\vspace*{1em} \noindent (10 points) In your report, describe how you know that your threaded
code uses only thread-safe calls, with pointers to the appropriate
documents, and why your code is free of race conditions.

Also, time your executions with the serial version and parallel version 
(take an average of 3 runs each; for the parallel version, investigate $N \in \{4, 64\}$) and discuss how well parallelization works.

\section*{Part 2: Nonblocking I/O (45 marks)}

In this part, you will write a single-threaded version of {\tt paster}
which uses nonblocking I/O to request multiple versions of the image
simultaneously.  You will need to use the {\tt curl\_multi} API as
well as either {\tt select} or {\tt epoll}. Once again, distribute the
work between the 3 provided servers.

Your solution should \emph{not} use pthreads. However, it should have multiple
concurrent connections to servers open. In this case, the {\tt -t} command
line option indicates the number of connections to keep open at once.
The {\tt -i} option always indicates which image to fetch.

\vspace*{1em} Again benchmark your work and report comparative results.
Discuss the performance of all three versions. Is it what you expected?


\section*{Part 3: Applying Amdahl's Law and Gustafson's Law (5 marks)}
The {\tt paster\_parallel} code clearly has a parallel and a serial part.
In your report, estimate the number of seconds typically spent in the serial part,
and explain how you arrived at that number. Discuss why Amdahl's Law and Gustafson's Law
apply, or don't apply, to {\tt paster\_parallel}.

\section*{Submitting}

To submit, simply push your fork of the git repository back to {\tt git.uwaterloo.ca}.
We will
be marking {\tt Makefile}, {\tt src/paster.c}, {\tt
  src/paster\_parallel.c}, {\tt src/paster\_nbio.c}, and {\tt
  report.pdf}. (You can modify the provided {\tt report/report.tex}
and create it with {\tt make report}; do not submit a {\tt doc}
file!). Running {\tt make} in the {\tt assignment-01} folder should
produce three files: {\tt bin/paster}, {\tt bin/paster\_parallel}, and
{\tt bin/paster\_nbio}. 

\newpage

\section*{Rubric}
The general principle is that correct solutions earn full marks.
However, it is your responsibility to demonstrate to the TA
that your solution is correct. Well-designed, clean solutions 
are therefore more likely to be recognized as correct. 

Solutions that do not compile will earn at most 39\% of the available
marks for that part. Segfaulting or otherwise crashing solutions earn
at most 49\%.

\paragraph{Part 0 (5 marks):} Self-explanatory.

\paragraph{Part 1 (45 marks):} (35 marks for implementation) A correct solution must:
\begin{itemize}
\item start the appropriate number of threads (5 points);
\item have each thread do work, distributed among the 3 servers (10 points);
\item code safety: prevent buffer overflows and clean up all allocated resources, as verified by valgrind (10 points); and
\item avoid data races and produce the correct output (10 points).
\end{itemize}

\noindent (10 marks for report) 8 marks for including the necessary information; 2 marks
for clarity.

\paragraph{Part 2 (45 marks):} (40 marks for implementation) A correct solution must:
\begin{itemize}
\item properly initialize the {\tt curl\_multi} handle with the appropriate number of individual {\tt curl\_easy} handles (10 points);
\item process results from the {\tt curl\_multi} handle ({\tt select}/{\tt multi\_perform}/{\tt multi\_info\_read} or {\tt multi\_perform\_socket}) (10 points); 
\item replace finished handles with new requests while requests remain (10 points);
\item code safety: prevent buffer overflows and clean up all allocated resources (5 points); and
\item produce the correct output (5 points).
\end{itemize}

\noindent (5 marks for report) 4 marks for information, 1 for clarity of exposition.

\paragraph{Part 3 (5 marks):} Self-explanatory.

\end{document}
