\documentclass[letterpaper,10pt]{article}

\usepackage{titling}
\usepackage{listings}
\usepackage{url}
\usepackage{setspace}
\usepackage{subfig}
\usepackage{sectsty}
\usepackage{pdfpages}
\usepackage{colortbl}
\usepackage{multirow}
\usepackage{multicol}
\usepackage{relsize}
\usepackage{amsmath}
\usepackage{wasysym}
\usepackage{fancyvrb}
\usepackage{amsmath,amssymb,amsthm,graphicx,xspace}
\usepackage[titlenotnumbered,noend,noline]{algorithm2e}
\usepackage[compact]{titlesec}
\usepackage{XCharter}
\usepackage[T1]{fontenc}
\usepackage{tikz}
\usetikzlibrary{arrows,automata,shapes,trees,matrix,chains,scopes,positioning,calc}
\tikzstyle{block} = [rectangle, draw, fill=blue!20, 
    text width=2.5em, text centered, rounded corners, minimum height=2em]
\tikzstyle{bw} = [rectangle, draw, fill=blue!20, 
    text width=4em, text centered, rounded corners, minimum height=2em]

\definecolor{namerow}{cmyk}{.40,.40,.40,.40}
\definecolor{namecol}{cmyk}{.40,.40,.40,.40}

\let\LaTeXtitle\title
\renewcommand{\title}[1]{\LaTeXtitle{\textsf{#1}}}


\newcommand{\handout}[5]{
  \noindent
  \begin{center}
  \framebox{
    \vbox{
      \hbox to 5.78in { {\bf ECE459: Programming for Performance } \hfill #2 }
      \vspace{4mm}
      \hbox to 5.78in { {\Large \hfill #4  \hfill} }
      \vspace{2mm}
      \hbox to 5.78in { {\em #3 \hfill} }
    }
  }
  \end{center}
  \vspace*{4mm}
}

\newcommand{\lecture}[3]{\handout{#1}{#2}{#3}{Lecture #1}}
\newcommand{\tuple}[1]{\ensuremath{\left\langle #1 \right\rangle}\xspace}

\addtolength{\oddsidemargin}{-1.000in}
\addtolength{\evensidemargin}{-0.500in}
\addtolength{\textwidth}{2.0in}
\addtolength{\topmargin}{-1.000in}
\addtolength{\textheight}{1.75in}
\addtolength{\parskip}{\baselineskip}
\setlength{\parindent}{0in}
\renewcommand{\baselinestretch}{1.5}
\newcommand{\term}{Winter 2017}

\singlespace

\title{\bf ECE 459: Programming for Performance\\Assignment 2}
\author{Patrick Lam \& Jeff Zarnett}
\date{\today ~ (Due: February 26, 2018)}

\lstset{basicstyle=\scriptsize, frame=single}

\begin{document}

\maketitle
% http://gurmeet.net/2008/09/20/latex-tips-n-tricks-for-conference-papers/
\newcommand{\squishlist}{
 \begin{list}{$\bullet$}
  { \setlength{\itemsep}{0pt}
     \setlength{\parsep}{3pt}
     \setlength{\topsep}{3pt}
     \setlength{\partopsep}{0pt}
     \setlength{\leftmargin}{1.5em}
     \setlength{\labelwidth}{1em}
     \setlength{\labelsep}{0.5em} } }
\newcommand{\squishend}{
  \end{list}  }

\noindent
Important Notes:

\squishlist
  \item {\bf Make sure you run your program on {\tt ece-tesla0.uwaterloo.ca}.}
  \item {\bf Use the command ``{\tt OMP\_NUM\_THREADS=14;export OMP\_NUM\_THREADS}'' to set 14 threads.}
  \item {\bf Run ``{\tt make report}'' and push your fork of the {\tt a2} directory.}
\squishend

\noindent
The repository will be created for you and initialized with starter files 
at \texttt{git.uwaterloo.ca}; look for your 
assignment 2 repo and then clone the provided files.

Grading will be done by running {\tt make}, running your programs,
looking at the source code and reading the report.

\section{Automatic Parallelization (15 marks)}
Ray tracing is, in principle, easy to automatically parallelize. You do
a separate computation for each point. In this part, you will convince a
parallelizing compiler (I recommend Oracle's Solaris Studio) to parallelize
a simple raytracing computation.

For this question, you will work with {\tt raytrace\_simple.c} and {\tt
  raytrace\_auto.c} in the {\tt q1} directory.  I've bumped up the
height of the image to 60000 pixels so that the compiler will find it
profitable to parallelize. Benchmark the sequential ({\tt raytrace})
and optimized sequential ({\tt raytrace\_opt}) versions. Note that the
compiler does manage to significantly optimize the computation of the sequential
raytrace. Report the speedup due to the compiler and
speculate why that is the case. Compare all subsequent numbers
to the optimized version.

Your first programming task is to modify your program so you can take
advantage of automatic parallelization. Determine what why it won't
parallelize as is, and make any changes necessary. Preserve behaviour
and make all your changes to {\tt raytrace\_auto.c}.

Solaris Studio 12.3 is available on {\tt ece-tesla0}. The provided {\tt
  Makefile} calls that compiler with the relevant flags. Your compiler
output should look something like the following (the line numbers
don't have to match, but you {\bf must} parallelize the critical loop):

\newpage
\begin{lstlisting}
Compiling Part 1 Automatic Parallelization
/opt/oracle/solarisstudio12.3/bin/cc  -fast -xautopar -xloopinfo -xreduction -xbuiltin -xO4 \
    src/raytrace_auto.c -o bin/raytrace_auto
"raytrace_auto.c", line 217: PARALLELIZED
"raytrace_auto.c", line 218: not parallelized, not profitable
"raytrace_auto.c", line 233: not parallelized, loop has multiple exits
"raytrace_auto.c", line 241: not parallelized, not a recognized for loop
"raytrace_auto.c", line 264: not parallelized, not a recognized for loop
\end{lstlisting}

Justify each change you make and explain why:
\begin{itemize}
\item the existing code does not parallelize;
\item your changes improve parallelization and preserve the behaviour of the sequential version
\item your changes adversely impact maintainability
\end{itemize}

Run your benchmark again and calculate your speedup. Speculate about why you got your speedup.

\squishlist
  \item {\bf Minimum expected speedup:} 1.1
  \item {\bf (my) initial solution speedup:} 1.1
\squishend

\paragraph{Totally unrelated hints.} Consider this page:
\begin{center}
  \scriptsize \url{http://stackoverflow.com/questions/321143/good-programming-practices-for-macro-definitions-define-in-c}
\end{center}
Also, let's say that you want a macro to return a struct of type {\tt struct foo} with
two fields. You can create such a struct on-the-fly like so: \verb!(struct foo){1,2}!.

\section{Using OpenMP Tasks (30 marks)}

We saw briefly how OpenMP tasks allow us to easily express some parallelism.
In this question, you will apply OpenMP tasks to the $n$-queens
problem\footnote{\url{http://jsomers.com/nqueen_demo/nqueens.html}}.
Benchmark the provided sequential version with a number that executes in approximately
15 seconds under {\tt -O2} (14 on {\tt ece-tesla0} takes roughly 13.4s, but 15 takes about 1m 25s. 
So choose 14 in this scenario.).

\noindent {\bf Notes:} Use {\tt er\_src} to get more detail about what the Oracle
Solaris Studio compiler did.  You may change the
Makefile's compilation flags if needed. Report speedups over the
compiler-optimized sequential version.  You can use any compiler, but
say which one you used. OpenMP tips:
\begin{center}
  \url{www.viva64.com/en/a/0054/}
\end{center}

Modify the code to use OpenMP tasks. Benchmark your modified program
and calculate the speedup. Explain why your
changes improved performance. Write a couple of sentences explaining how
you could further improve performance.

\vspace*{1em}
\noindent {\bf Hints:} 1) Be sure to get the right variable scoping, or
you'll get race conditions. 2) Just adding the task annotation is going
to make your code way slower. 3) You will have to implement a cutoff to
get speedup. See, for instance, the Google results for ``openmp
fibonacci tasks''. 4) My solution includes 4 annotations and some
cutting-and-pasting of code. 5) Be sure to check the output of the
OpenMP program for a given input against the non-OpenMP program to be
sure that your results are consistent.

\squishlist
  \item {\bf Minimum expected speedup:} 1.5
  \item {\bf Initial solution speedup:} 1.7
\squishend

\newpage

\section{Manual Parallelization with OpenMP (55 marks)}

This time rather than just apply OpenMP directives to an existing program, you will write the program according to what is written below and verify its correctness with some provided sample files.

The program does a simulation of Coulomb's Law: there are proton and electron particles (that have the standard masses and charges). The protons are kept fixed in place via mechanical forces, but the electrons will move. Electrons move according to classical physics: they are attracted to protons and repelled by other electrons. The program will perform just one step of the simulation. The plan is to use Heun's method from ECE 204A (if you took that course).

The program takes parameters:
\begin{enumerate}
\item $h$ -- initial size of simulation step (a measure of time) 
\item $e$ -- epsilon, the amount of error allowed
\item An input file of initial positions (comma separated value file).
\end{enumerate}

The input file is in csv format, first column is whether it is a proton (indicated by \texttt{p}) or electron (indicated by \texttt{e}), second is x coordinate, third is y coordinate, fourth is z coordinate. The input values are floating point numbers (of \texttt{float} type) and they have a precision of 6 digits and are written using scientific notation such as \texttt{3.14159e-05}. 

The program produces as output the new positions of the electrons and protons (the protons should not move). The output file format should be the same as the input file format, and the order of each of the particles must be the same as the order of the input file.

Some physics facts you may need to know to get this done:

\begin{itemize}
\item Coulomb's law calculates the force of attraction as $\dfrac{k (q_{1} q_{2})}{r^{2}}$.  Where $k$ is Coulomb's constant, $q$ the charge, and $r$ the distance between the two points.

\item The value for $k$ is Coulomb's constant $8.99 \times 10^{9} N m^{2} C^{-2}$. 

\item The charge of an electron is the same as the charge of a proton which is $1.60217662 \times 10^{-19}$ Coulombs. Electrons have negative charge and protons have positive charge. 

\item The mass of an electron is $9.10938356 \times 10^{-31}$ kg.
\end{itemize}


The simulation algorithm is:

\begin{enumerate}
\item The initial vector $y_{0}$ contains the positions of the electrons and protons.

\item Calculate a vector $k_{0}$ being the sum on each electron (and zero force on the protons), and use that to approximate the new positions vector $y_{1}$ = $y_{0} + h \times k_{0}$.

To calculate new positions, remember $F = m \times a$;  acceleration will be considered constant in the range we're talking about ($h$ represents a very small unit of time, or it will after the value of $h$ has been divided multiple times after we find the error has been too large). 

With the acceleration then we compute velocity $v$ for any point as $h \times a$; with the velocity in hand, the position is updated as simply $h \times v$ (again worth noting that the times are very small here so we don't have wild swings in $v$). 

Or to break it down in math notation: $F = ma \rightarrow \dfrac{F}{m} = a \rightarrow \dfrac{hF}{m} = v \rightarrow \dfrac{h^{2}F}{m} = d$

\item Calculate a second vector $k_{1}$ being the sum of the forces on each point at the new position $y_{1}$, again having zero forces on the protons. You don't have to calculate the next set of positions $y_{2}$ because it is not needed for the next step.

\item Using the two force vectors,  compute $z_{1}$ as $y_{0} + h \times \dfrac{( k_{0} + k_{1} )}{2}$ (the average of the two forces). This produces a second, more accurate position vector.

\item If $||z_{1} - y_{1}|| > e$ at any position (i.e., the error at any one position is larger than the tolerance), then the simulation is too coarse, and we need to go back to step 2, this time with $h$ divided by 2.

\item Otherwise, you are ready to produce the output file. The output file should be in the same format as the input file, and the positions $z_{1}$ are the final results that should be put in the output. The output file name should be the same as the input file name, with the extension changed to \texttt{.out}. So if the input file to the program is \texttt{testcase1.in} then the output file name should be \texttt{testcase1.out}. 

\end{enumerate}

Once the sequential version is written, you will apply OpenMP directives, one at a time (or in a tightly integrated group) and judge their impact on the runtime of your program as a way to assess what areas benefit most from parallelization and the impact of various OpenMP directives. Note down for your report what is effective and what is ineffective, and what produces invalid results. 

To produce the data we want for the report, the easiest way is just to take notes about what about OpenMP directives you have used. Each time you add some OpenMP directive(s), note down what it was and what effect it had, if any. By writing down what was successful and what was not, as well as what made a big difference, you will have at hand all the data you need.

You should also try to achieve the maximum speedup you can while preserving behaviour. The usage of the compiled output is: \texttt{./protons h e inputfile}. So a sample call might be: \texttt{./protons 1 1.000e-05 example.in} ; this behaviour needs to be preserved for your solution to be tested.

Note that for testing there are some provided input and output files. The sample output files are created with parameters $h=0.001$ and $e=0.00001$. You should, however, also do some tests with other parameters as well. The test cases provided
are not a complete test suite.

Submit the final OpenMP-annotated version of your code. Your report will contain the impact of various OpenMP directives, walking the reader through the process of applying them and testing out their effectiveness.

\squishlist
  \item {\bf Minimum expected speedup:} 8x
  \item {\bf Initial solution speedup:} 12x
\squishend


\section*{Rubric}

The general principle is that correct solutions earn full marks.
However, it is your responsibility to demonstrate to the TA
that your solution is correct. Well-designed, clean solutions 
are therefore more likely to be recognized as correct. 

Solutions that do not compile will earn at most 39\% of the available
marks for that part. Segfaulting or otherwise crashing solutions earn
at most 49\%.

\paragraph{Part 1, Automatic Parallelization (15 marks):}  
\begin{itemize}
\item 10 marks for implementation: A correct solution must:
\begin{itemize}
	\item preserve the behaviour (5 points); and
	\item enable additional parallelization (5 points).
\end{itemize}
 
\item 5 marks for report: include the necessary information
(describing the experiments and results, reasonably speculating about
the cause, and explaining why you preserve behaviour)
\end{itemize}

\paragraph{Part 2, OpenMP Tasks (30 marks):} 

\begin{itemize}
\item 20 marks for implementation: A correct solution must:
\begin{itemize}
	\item properly use OpenMP tasks to get a speedup;
	\item be free of obvious race conditions.
\end{itemize}
\item 10 marks for report: 
\begin{itemize}
\item 7 marks for analyzing the performance
of the provided version, describing the speedup due to your
changes, explaining why your changes improved performance, and
speculating reasonably about further changes. 
\item 3 marks for clarity.
\end{itemize}
\end{itemize}

\paragraph{Part 3, Manual Parallelization (55 marks):} 
\begin{itemize}
\item 35 marks for the single-threaded implementation. 

\item 10 marks for the use of OpenMP pragmas and minor code changes to parallelize the code and get speedup.

\item 10 marks for report: Explain which OpenMP directives helped. Try them out individually and determine the impact of each, and identify which ones work synergistically with others. 3 marks for clarity.
\end{itemize}



\end{document}

